% ============================================================
\section{Район исследования и данные}
\label{sec:study_area}
% ============================================================

\subsection{Район исследования}

Исследование проведено на сельскохозяйственных угодьях трёх областей Северного Казахстана --- Костанайской, Акмолинской и Северо-Казахстанской (50°--54°~с.ш., 64°--72°~в.д.) (Рис.~\ref{fig:study_area}).

\begin{figure}[H]
  \centering
  \includegraphics[width=0.90\textwidth]{fig1_study_area.png}
  \caption{Карта района исследования. Северный Казахстан (Костанайская, Акмолинская и Северо-Казахстанская области). Точки обозначают расположение 186~опробованных полей (81~хозяйство, 1085~образцов, 2020--2023~гг.).}
  \label{fig:study_area}
\end{figure}

\textbf{Климат.} Резко континентальный (Dfb/BSk по Кёппену): среднегодовая температура $+0.5 \ldots +3.0$\,°C, амплитуда от $-40$ до $+35$\,°C. Осадки 250--400~мм/год (до 40\% в июне--августе). Вегетационный период 130--150~дней, сумма активных температур 2000--2600\,°C$\cdot$дней~\citep{Fick2017}.

\textbf{Почвы.} Чернозёмы обыкновенные и южные (Haplic Chernozems, WRB) на севере, каштановые почвы (Kastanozems) на юге. Содержание гумуса 2--6\% (закономерно убывает с севера на юг). Гранулометрический состав --- средний и тяжёлый суглинок. Длительное освоение (с 1954~г.) привело к дегумификации и повышению дефляционной уязвимости.

\textbf{Рельеф.} Слабоволнистая равнина, 180--420~м, уклоны 0--3°. Преобладание ветровой эрозии над водной.

\textbf{Агрохозяйственный контекст.} Яровая пшеница, ячмень, масличные. Доля no-till: $\sim$45\% (2023). Нормы удобрений: от нуля до 90~кг/га N + 40~кг/га P (д.в.) --- значительная пространственная гетерогенность на уровне хозяйств.

\subsection{Почвенные данные}
\label{sec:soil_data}

Набор данных включает \textbf{1085~образцов} из пахотного горизонта (0--25~см), собранных за четыре полевых сезона (2020--2023~гг.) на \textbf{186~полевых участках} (81~хозяйство). Пробоотбор проведён в послеуборочный период (сентябрь--октябрь) по ГОСТ~28168-89: смешанная проба из 15--20 точечных уколов на площади 20--50~га. Координаты центроидов --- GNSS (±3~м).

Лабораторные определения (аккредитованная лаборатория Республики Казахстан):
\begin{itemize}[nosep]
  \item \textbf{pH\textsubscript{KCl}} --- потенциометрия в 1М KCl (ГОСТ~26483-85);
  \item \textbf{SOC, \%} --- метод Тюрина (ГОСТ~26213-91), пересчёт по Ван~Беммелену~\citep{Pribyl2010}: SOC~$=$~HU$\times$0.58;
  \item \textbf{NO$_3$, мг/кг} --- ионометрическое определение (ГОСТ~26951-86);
  \item \textbf{P$_2$O$_5$ и K$_2$O, мг/кг} --- метод Мачигина, 1\% (NH$_4$)$_2$CO$_3$ (ГОСТ~26205-91);
  \item \textbf{S, мг/кг} --- турбидиметрия (KCl-вытяжка).
\end{itemize}

Полный мастер-датасет (включая все годы) содержит 1215~строк; после удаления записей с пропусками в целевых свойствах (94~записи, 7.7\%, исключительно SOC/NO$_3$/P$_2$O$_5$/K$_2$O/S; pH не имеет пропусков) остаётся \textbf{1085~полных записей}, использованных в анализе.
