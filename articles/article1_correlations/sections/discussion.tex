% ============================================================
\section{Обсуждение}
\label{sec:discussion}
% ============================================================

\subsection{Физическая интерпретация ключевых взаимосвязей}

\subsubsection{pH --- GNDVI\textsubscript{весна} ($\rho = -0.670$): наиболее сильная связь}

Это наиболее сильная из обнаруженных связей. GNDVI~$= (\text{NIR} - \text{Green})/(\text{NIR} + \text{Green})$ чувствителен к содержанию хлорофилла $a+b$, которое определяется доступностью азота и микроэлементов, контролируемой pH~\citep{Gitelson1996}. На кислых почвах (pH~5.3--6.5) северной части региона с более высоким содержанием гумуса и лучшей влагообеспеченностью ранневесенняя вегетация развивается интенсивнее, давая высокие значения GNDVI. Отрицательный знак корреляции отражает инверсию: кислые почвы (низкий pH) более продуктивны в условиях данного региона, где pH~$= 5.3$--$6.5$ попадает в оптимальный диапазон для злаковых.

Высокая сила связи ($|\rho| = 0.67$) существенно превышает типичные значения для аридных зон ($|\rho| = 0.3$--$0.5$)~\citep{Wadoux2020} и обусловлена, вероятно, широтным масштабом исследования ($\sim$500~км), создающим выраженный климатический градиент, одновременно влияющий на pH (через выщелачивание) и на вегетацию (через увлажнение). Это одновременно демонстрирует силу и потенциальную ловушку крупномасштабных исследований: корреляция частично опосредована общим пространственным трендом, а не прямой каузальной связью pH$\to$спектральный сигнал.

\subsubsection{pH --- MAP ($\rho = +0.659$) и уклон ($\rho = +0.609$)}

Среднегодовые осадки отражают фундаментальный почвообразовательный процесс: нарастание увлажнённости к северу интенсифицирует выщелачивание карбонатов Ca и Mg, снижая pH~\citep{Jenny1941}. Уклон контролирует латеральное перемещение влаги: даже в условиях слабохолмистого рельефа Северного Казахстана (уклоны~$< 3°$) перепады определяют интенсивность выноса карбонатов. Данные закономерности согласуются с SCORPAN-моделью~\citep{McBratney2003}, где климат (c) и рельеф (r) --- ведущие факторы почвообразования.

\subsubsection{K$_2$O --- BSI\textsubscript{весна} ($\rho = -0.478$)}

BSI~$= ((B_6 + B_4) - (B_8 + B_2))/((B_6 + B_4) + (B_8 + B_2))$ отражает яркость обнажённой почвы в Red и SWIR. Высокие значения характерны для светлых, карбонатных почв южной части с низким калийным фондом. Обратная связь указывает на генетическую зависимость обменного K от степени выветрелости глинистых минералов (иллитов), которые заодно определяют тёмный цвет богатых гумусом почв~\citep{Barré2014}. BSI чувствителен к обоим компонентам: яркость (карбонатность) и структура (глинистость).

\subsubsection{SOC --- MSI\textsubscript{std} ($\rho = -0.350$)}

Moisture Stress Index ($\text{MSI} = \text{SWIR}/\text{NIR}$) характеризует водный стресс. Его высокая временна\'{я} вариабельность ($\sigma$ по сезонам) указывает на почвы с низким SOC, обладающие худшей водоудерживающей способностью и подверженные резким колебаниям влажности~\citep{Rawls2003}. Богатые органическим веществом почвы обеспечивают стабильное водоснабжение, стабилизируя MSI. Примечательно, что не абсолютное значение MSI, а именно его \textit{временна\'{я} вариабельность} даёт наибольшую корреляцию --- это указывает на ценность мультитемпоральных характеристик~\citep{Zizala2022}.

\subsubsection{P$_2$O$_5$ --- GS\_temp ($\rho = +0.476$) и~S --- слабые корреляции ($|\rho| < 0.28$)}

Положительная связь P$_2$O$_5$ с температурой вегетации может отражать: (i)~интенсификацию минерализации органического фосфора; (ii)~различия в агропрактиках --- южные тёплые районы специализируются на интенсивном земледелии с более высокими нормами фосфорных удобрений. Разделить эти факторы без данных об удобрениях невозможно.

Подвижная сера не имеет прямого спектрального отклика в 400--2500~нм и определяется антропогенными факторами (удобрения, атмосферные выпадения) и микробиологическими процессами. Аналогичную проблему отмечали Vaudour et al.~\citep{Vaudour2019}, исключившие S из анализа.

\subsection{Пространственная структура и её следствия}

Высокие значения Moran~$I$ (0.51--0.86) имеют двойное значение:
\begin{enumerate}[nosep]
  \item \textbf{Корреляционное}: пространственная структура усиливает наблюдаемые корреляции с ДЗЗ, поскольку и почвенные свойства, и спектральные данные определяются общими пространственными факторами (климат, геоморфология);
  \item \textbf{Методологическое}: при построении прогностических моделей случайное разбиение train/test \textbf{систематически завышает} оценки на 10--40\%~\citep{Roberts2017, Wadoux2021}; необходимы пространственные стратегии валидации (Field-LOFO-CV, Farm-LOFO-CV).
\end{enumerate}

Выявленный широтный градиент ($\rho_{\text{lat}} = -0.41$ для pH, $+0.29$ для SOC) количественно подтверждает классическую зональную последовательность: с увеличением широты растёт увлажнённость $\to$ усиливается выщелачивание (pH$\downarrow$) и гумусонакопление (SOC$\uparrow$)~\citep{Jenny1941}. Эта зональность --- ключевой механизм, обеспечивающий высокую корреляцию pH с климатическими и спектральными ковариатами.

Парадокс S (высокий Moran~$I = 0.767$ при низком ICC~$= 0.166$) заслуживает отдельного внимания. Пространственная автокорреляция S обусловлена \emph{между-региональной} кластеризацией агропрактик (группы хозяйств с различными нормами серосодержащих удобрений), а не зональными почвообразовательными градиентами. Это различие между <<географической>> и <<педогенетической>> пространственной структурой имеет прямые следствия для моделирования: высокий Moran~$I$ не гарантирует высокую предсказуемость, если структура антропогенная.

\subsection{Декомпозиция дисперсии: теоретический потолок предсказания}

Различие в долях между-полевой дисперсии (73.2\% для pH vs 22.6\% для S) определяет \textbf{ceiling effect} для точности предсказания: спутниковые данные с разрешением 10--30~м опосредуют преимущественно крупномасштабную (между-полевую) вариабельность. Следовательно:
\begin{itemize}[nosep]
  \item Для pH (ICC~=~0.71): $>$70\% изменчивости теоретически доступно для предсказания;
  \item Для S (ICC~=~0.17): лишь $\sim$23\% потенциально объяснимо ковариатами ДЗЗ, остальное --- стохастическая внутриполевая (и внутрихозяйственная) неоднородность.
\end{itemize}

Это согласуется с иерархией SCORPAN~\citep{McBratney2003}: свойства, определяемые зональными факторами почвообразования (климат, рельеф, материнская порода), лучше предсказываются по данным ДЗЗ, чем свойства с доминирующим антропогенным компонентом. Практическое следствие: для S и (частично) NO$_3$ необходимы дополнительные ковариаты, не извлекаемые из спутниковых данных (история удобрений, агрохимические карты предыдущих лет, данные о севооборотах).

\subsection{Конфаундинг pH: предостережение для цифрового почвенного картирования}

Обнаруженное опосредующее влияние pH (\textbf{41.9\%} корреляции SOC--NDVI\textsubscript{лето}) имеет критические импликации для DSM-исследований, использующих вегетационные индексы как предикторы SOC. Если pH не включён в набор предикторов или не контролируется как конфаундер, модель фактически предсказывает \textbf{pH, а не SOC}, что может привести к:
\begin{enumerate}[nosep]
  \item Завышению оценки точности при пространственной автокорреляции pH;
  \item Физически некорректной интерпретации <<хорошей>> корреляции SOC--NDVI;
  \item Деградации работы модели при переносе на территории с иным соотношением pH--SOC.
\end{enumerate}

Рекомендации: (i)~включать pH как явный предиктор SOC; (ii)~при невозможности --- использовать позднелетние индексы ($C_{\text{pH}} = -13.7\%$), где конфаундинг минимален; (iii)~верифицировать модели частными корреляциями. Данный результат особенно актуален для степной зоны, где pH и SOC имеют выраженную антиординатную зональность ($\rho_{\text{pH--SOC}} = -0.227$).

\subsection{Роль масштаба признакового пространства}

Анализ 536 мультимодальных + 110~композитных + 9~производных признаков позволил выявить неочевидные связи, которые были бы пропущены при стандартном наборе из 20--40 индексов:
\begin{itemize}[nosep]
  \item Текстурные GLCM-признаки (IDM\textsubscript{NIR, лето}) вошли в топ-предикторы NO$_3$ --- пространственная однородность NIR-отражения маркирует однородные посевы с хорошим N-питанием;
  \item Мультисезонные дельты ($\Delta$NDVI\textsubscript{лето$-$весна}) --- лучшие среди чисто спектральных RS-признаков для NO$_3$;
  \item Производный P$_2$O$_5$/K$_2$O-баланс ($\rho = -0.555$ с уклоном) --- сильнейший топо-предиктор среди составных;
  \item SoilGrids как фоновый предиктор: SOC\textsubscript{SG, 0--5~см} ($\rho = 0.27$ для SOC) --- слабый, но значимый, что указывает на потенциал ансамблевых подходов.
\end{itemize}

Это обосновывает \textbf{мультиисточниковый подход} к формированию признакового пространства моделей цифрового почвенного картирования и показывает ценность расширенного скрининга ковариат.

\subsection{Сопоставление с литературой}

Полученная максимальная корреляция для pH ($|\rho| = 0.67$) превышает типичные значения для аридных зон ($R^2 = 0.4$--$0.5$), что объясняется масштабом территории и выраженностью зонального градиента. Для SOC наши $|\rho| = 0.35$ ($R^2_{\text{equiv}} \approx 0.12$) ниже, чем $R^2 = 0.70$ у Castaldi et al.~\citep{Castaldi2019}, однако их исследование проводилось на \textit{обнажённых почвах} (без растительности), что принципиально повышает информативность спектральных данных в VNIR/SWIR. В нашем случае данные получены преимущественно на вегетирующих посевах, что является более реалистичным, но менее информативным сценарием.

Значения Moran~$I = 0.51$--$0.86$ существенно выше, чем обычно сообщается в DSM-литературе ($I = 0.1$--$0.4$)~\citep{Kerry2010}, что подчёркивает специфику крупномасштабных региональных исследований и обосновывает повышенные требования к стратегии валидации.

\subsection{Ограничения}

\begin{enumerate}[nosep]
  \item \textbf{Монотонность}: ранговая корреляция оценивает монотонные зависимости; ML-модели могут обнаруживать нелинейные связи;
  \item \textbf{Четыре года наблюдений}: климатические аномалии отдельных лет могут смещать результаты;
  \item \textbf{Масштаб пробоотбора}: элементарный участок 20--50~га усредняет внутриполевую вариабельность, ограничивая интерпретацию на масштабе $<$200~м;
  \item \textbf{Отсутствие данных об удобрениях}: история внесения не учтена, что ограничивает интерпретацию P$_2$O$_5$, K$_2$O и S;
  \item \textbf{Один регион}: закономерности специфичны для чернозёмно-каштановой зоны;
  \item \textbf{Каузальность}: корреляции не доказывают причинность; необходимы структурные модели (SEM) и естественные эксперименты.
\end{enumerate}
