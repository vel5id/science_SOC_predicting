% ============================================================
\section{Заключение}
\label{sec:conclusion}
% ============================================================

В данной работе впервые проведён комплексный количественный анализ взаимосвязей между шестью агрохимическими свойствами пахотного горизонта почв степной зоны Северного Казахстана и \textbf{536 мультимодальными признаками ДЗЗ}, извлечёнными из шести спутниковых и вспомогательных источников. На основании 1085~полевых образцов (186~участков, 81~хозяйство, 2020--2023~гг.) установлено:

\begin{enumerate}
  \item \textbf{Наиболее сильные корреляции} (Спирмен, BH FDR) обнаружены для pH: GNDVI\textsubscript{L8, весна} ($\rho = -0.670$, $p = 1.6 \times 10^{-142}$), MAP ($\rho = 0.659$) и уклон ($\rho = 0.609$). Иерархия предсказуемости по $|\rho|_{\max}$: pH $\gg$ K$_2$O $\approx$ P$_2$O$_5$ $>$ SOC $>$ NO$_3$ $>$ S.

  \item \textbf{Пространственная автокорреляция} (Moran $I = 0.51$--$0.86$) является наивысшей для pH ($I = 0.861$) и анализируемых территорий. Выявлен значимый широтный тренд ($\rho_{\text{lat}} = -0.41$ для pH, $+0.29$ для SOC), количественно подтверждающий зональную последовательность. Это обуславливает \textbf{обязательность пространственных стратегий валидации} в прогностических моделях.

  \item \textbf{Декомпозиция дисперсии} показала, что между-полевой компонент составляет 73.2\% для pH (ICC~=~0.71) и лишь 22.6\% для S (ICC~=~0.17), определяя теоретический потолок спутникового предсказания.

  \item \textbf{Конфаундинг pH}: 41.9\% наблюдаемой корреляции SOC--NDVI\textsubscript{лето} опосредовано pH. Это предостережение для DSM-исследований, использующих вегетационные индексы как предикторы SOC без контроля pH.

  \item \textbf{Композитные признаки} (110~шт.) в целом не превосходят одиночные индексы, за исключением отдельных комбинаций (GNDVI$\times$BSI для K$_2$O: $\Delta|\rho| = +0.01$; GNDVI$-$NDRE для NO$_3$: $|\rho| = 0.416$).

  \item \textbf{Производные индикаторы} (P$_2$O$_5$/K$_2$O) демонстрируют более сильные корреляции с рельефом ($\rho = -0.555$), чем индивидуальные свойства, указывая на синергетические эффекты.

  \item \textbf{Сезонная дифференциация}: весенние данные оптимальны для pH и K$_2$O (почва частично обнажена); летние --- для NO$_3$ (корреляция с пиком вегетации).

  \item \textbf{Сера не предсказуема} из мультиспектральных данных ($|\rho|_{\max} = 0.28$, ICC~=~0.17): и корреляционная база, и структура дисперсии фундаментально ограничивают потенциал ДЗЗ.
\end{enumerate}

Результаты формируют \textbf{эмпирическую и методологическую базу} для проектирования признакового пространства моделей цифрового почвенного картирования степной зоны Центральной Азии.

\subsection*{Связь с Частью~2}

Выявленные в настоящей работе корреляционные зависимости и структурные закономерности непосредственно используются в сопутствующей публикации (Часть~2), где:
\begin{itemize}[nosep]
  \item иерархия <<предсказуемости>> ($|\rho|_{\max}$: pH $\gg$ K$_2$O $\approx$ P$_2$O$_5$ $>$ SOC $>$ NO$_3$ $>$ S) верифицируется через предиктивное моделирование 12~ML-алгоритмов и 3~DL-архитектур;
  \item рекомендации по сезонному выбору признаков (весна для pH и K$_2$O, лето для NO$_3$) учтены при формировании признакового пространства и мультисезонных CNN-композитов;
  \item результаты декомпозиции дисперсии (ICC) объясняют расхождения между стратегиями пространственной валидации (Field-LOFO vs Farm-LOFO);
  \item предостережение о конфаундинге pH мотивирует включение pH как явного предиктора SOC.
\end{itemize}

\noindent Перспективные направления дальнейших исследований: (i)~построение прогностических ML/DL-моделей на основе выявленных связей с LOFO-CV (реализовано в Части~2); (ii)~интеграция гиперспектральных данных (EnMAP, PRISMA) для улучшения SOC; (iii)~привлечение данных об агропрактиках для P$_2$O$_5$, K$_2$O, S; (iv)~структурное моделирование (SEM) для каузальных выводов.
