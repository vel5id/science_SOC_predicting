% ============================================================
\section*{Аннотация}
\addcontentsline{toc}{section}{Аннотация}
% ============================================================

\noindent\textit{Данная работа является первой из двух сопутствующих публикаций, посвящённых цифровому почвенному картированию степной зоны Северного Казахстана. В Части~1 выполнен комплексный корреляционный анализ, формирующий эмпирическую и методологическую базу. В Части~2 результаты используются для построения и сравнения предиктивных моделей машинного и глубокого обучения.}

\medskip
Цифровое почвенное картирование (ЦПК) базируется на статистически обоснованных зависимостях между свойствами почв и ковариатами окружающей среды, однако для степной зоны Центральной Азии эти зависимости исследованы недостаточно.
В настоящей работе выполнен комплексный корреляционный анализ между шестью агрохимическими свойствами пахотного горизонта (pH\textsubscript{KCl}, SOC, NO$_3$, P$_2$O$_5$, K$_2$O, S; $n = 1085$, 186~полей, 81~хозяйство, 2020--2023~гг.) и набором из \textbf{536 мультимодальных признаков}, извлечённых через Google Earth Engine из шести источников: Sentinel-2, Landsat-8, Sentinel-1~SAR, SRTM~DEM, SoilGrids~v2.0 и ERA5-Land. Признаковое пространство включает сезонные медианы спектральных каналов и индексов, временны\'{е} характеристики (межсезонные дельты, амплитуды), текстурные GLCM-метрики, топографические, климатические и педологические переменные, а также \textbf{110 композитных спектральных признаков} (попарные произведения индексов, мультисезонные дельты, нормализованные разности полос).

Для каждой из $>3200$ пар <<свойство -- признак>> вычислен коэффициент Спирмена ($\rho$) с поправкой Бенджамини--Хохберга (FDR, $\alpha = 0.05$). Установлены наиболее сильные связи: pH --- GNDVI\textsubscript{L8, весна} ($\rho = -0.670$, $p = 1.6 \times 10^{-142}$), среднегодовые осадки MAP ($\rho = 0.659$), уклон ($\rho = 0.609$); K$_2$O --- BSI\textsubscript{S2, весна} ($\rho = -0.478$); P$_2$O$_5$ --- средняя температура вегетационного периода ($\rho = 0.476$). Глобальный индекс Морана подтвердил выраженную пространственную автокорреляцию для всех свойств ($I = 0.51$--$0.86$, $p < 10^{-15}$); выявлен значимый широтный градиент pH ($\rho_{\text{lat}} = -0.41$) и SOC ($\rho_{\text{lat}} = +0.29$). Иерархическая декомпозиция дисперсии показала, что между-полевой компонент составляет 22.6\% (S) -- 73.2\% (pH), определяя пределы внутриполевого предсказания. Конфаундинг-анализ установил, что pH опосредует 41.9\% наблюдаемой корреляции SOC--NDVI\textsubscript{лето}. Композитный признак GNDVI$\times$BSI\textsubscript{весна} обеспечивает прирост $\Delta|\rho| = +0.010$ для K$_2$O по сравнению с одиночным BSI.

Результаты формируют научную базу для проектирования признакового пространства предиктивных моделей ЦПК, что реализовано в сопутствующей работе (Часть~2).

\medskip
\noindent\textbf{Ключевые слова:} корреляционный анализ; дистанционное зондирование; спектральные индексы; пространственная автокорреляция; индекс Морана; декомпозиция дисперсии; конфаундинг; чернозёмы; Sentinel-2; Северный Казахстан
