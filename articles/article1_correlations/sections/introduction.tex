% ============================================================
\section{Введение}
\label{sec:introduction}
% ============================================================

\subsection{Контекст и проблема}

Почвенные ресурсы --- фундаментальная основа сельскохозяйственного производства и продовольственной безопасности. Деградация почв затрагивает $\sim$33\% мировых земельных ресурсов, а связанные экономические потери оцениваются в 10.6~трлн~USD ежегодно~\citep{Montanarella2016}. Оперативный и пространственно-детальный мониторинг агрохимических свойств --- необходимое условие парадигмы точного земледелия, направленной на оптимизацию внесения удобрений, минимизацию экологического воздействия и повышение урожайности~\citep{Gebbers2010}. Между тем, традиционные методы опираются на отбор проб с последующим лабораторным анализом (15--50~USD за образец, 2--4~недели)~\citep{ViscarraRossel2006}, что делает прямое обследование непрактичным для регулярного мониторинга обширных территорий.

Проблема особенно актуальна для Северного Казахстана --- одного из крупнейших зернопроизводящих регионов мира ($>$20~млн~га пашни)~\citep{Swinnen2017}. Регион характеризуется резко континентальным климатом, коротким вегетационным периодом и высокой уязвимостью почв к ветровой эрозии~\citep{Kraemer2015}. Несмотря на стратегическое значение, число работ, где систематически анализировались бы взаимосвязи между физико-химическими свойствами почв и мультиисточниковыми данными ДЗЗ конкретно для данной зоны, крайне невелико по сравнению с хорошо изученными территориями Европы, Китая и Южной Америки.

\subsection{Физические основы дистанционного зондирования почв}

Спектральный отклик почвенной поверхности и растительного покрова содержит информацию о физико-химических свойствах подстилающих почв. Отражательная способность в VNIR (400--1000~нм) чувствительна к содержанию органического вещества, оксидов железа и минералогическому составу; SWIR-область (1000--2500~нм) отражает влажность и глинистые минералы~\citep{BenDor2009, ViscarraRossel2010}. Спутниковые системы Sentinel-2~\citep{Drusch2012} и Landsat-8~\citep{Roy2014} обеспечивают мультиспектральные данные с разрешением 10--30~м и временным повторением 5--16~дней. Радарные данные Sentinel-1~SAR~\citep{Torres2012}, работающие независимо от облачности, несут информацию о диэлектрических свойствах и влажности поверхности (прирост точности 8--15\% при интеграции с оптическими данными)~\citep{Bauer2019}. Цифровые модели рельефа SRTM определяют геоморфологические ковариаты (уклон, экспозиция, топографический индекс влажности), контролирующие перераспределение влаги и эрозионные потоки~\citep{Farr2007}. Глобальные карты SoilGrids~v2.0~\citep{Hengl2017} предоставляют фоновую педологическую информацию (250~м), а ERA5-Land~\citep{MunozSabater2021} --- климатические характеристики.

\subsection{Существующие исследования и разрыв знаний}

Castaldi et al.~\citep{Castaldi2019} показали, что Sentinel-2 на обнажённых почвах Бельгии обеспечивает $R^2$ до~0.70 для SOC. Vaudour et al.~\citep{Vaudour2019} продемонстрировали потенциал S2 для картирования pH, SOC и текстуры во Франции. Мета-анализ Wadoux et al.~\citep{Wadoux2020} установил медианные $R^2 = 0.50$--$0.65$ для SOC и $0.55$--$0.75$ для pH в задачах цифрового почвенного картирования. \v{Z}\'{i}\v{z}ala et al.~\citep{Zizala2022} показали ценность мультисезонных данных S2 для SOC в Чехии. Однако:

\begin{enumerate}[nosep]
  \item \textbf{Географический пробел}: систематический мультиисточниковый корреляционный анализ для степной зоны Центральной Азии (чернозёмно-каштановые почвы, резко континентальный климат) отсутствует;
  \item \textbf{Масштаб признакового пространства}: большинство работ анализируют ограниченный набор из 20--80 признаков; систематический скрининг $>$500 мультимодальных ковариат, включая композитные спектральные индексы, ранее не проводился;
  \item \textbf{Компоненты вариации}: роль между-полевой vs внутриполевой дисперсии, пространственной автокорреляции и конфаундинга между почвенными свойствами редко количественно оценивается в контексте ДЗЗ;
  \item \textbf{Целевой разрыв}: предсказание NO$_3$, P$_2$O$_5$, K$_2$O и S значительно менее изучено, чем SOC и pH.
\end{enumerate}

\subsection{Цели и задачи}

Настоящая работа является \textbf{первой из двух сопутствующих публикаций}, посвящённых цифровому почвенному картированию степной зоны Северного Казахстана. В Части~1 (данная статья) проводится комплексный количественный анализ взаимосвязей между шестью агрохимическими свойствами пахотного горизонта и мультимодальными данными ДЗЗ, формирующий научную и методологическую основу для последующего предиктивного моделирования. В Части~2 (сопутствующая статья) выявленные корреляционные зависимости, структура дисперсии и рекомендации по выбору оптимальных признаков используются для построения и систематического сравнения 12~моделей машинного обучения и 3~архитектур глубокого обучения с трёхуровневой пространственной валидацией.

Конкретные задачи Части~1:

\begin{enumerate}[nosep]
  \item Количественная оценка ранговых корреляций 6~почвенных свойств с 536~ковариатами ДЗЗ из 6~источников (Спирмен, BH~FDR);
  \item Сравнительный анализ 110~композитных спектральных признаков (попарные произведения, нормализованные разности, мультисезонные дельты) с одиночными индексами;
  \item Идентификация оптимальных фенологических окон для каждого свойства;
  \item Выявление пространственной структуры почвенных свойств (Moran~$I$, широтный градиент);
  \item Иерархическая декомпозиция дисперсии (между-полевая vs внутриполевая) и оценка ICC;
  \item Конфаундинг-анализ: оценка опосредующей роли pH в корреляциях <<SOC --- спектральные индексы>> методом частных корреляций;
  \item Конструирование производных почвенных индикаторов и оценка их корреляции с данными ДЗЗ.
\end{enumerate}
