% ============================================================
\section{Результаты}
\label{sec:results}
% ============================================================

\subsection{Описательная статистика}
\label{sec:descriptive_results}

Описательная статистика шести свойств представлена в Таблице~\ref{tab:descriptive}.

\begin{table}[H]
\centering
\caption{Описательная статистика агрохимических свойств ($n = 1085$)}
\label{tab:descriptive}
\small
\begin{tabular}{lccccccccc}
\toprule
\thead{Свойство} & \thead{Ед.} & \thead{Среднее} & \thead{Медиана} & \thead{SD} & \thead{Мин} & \thead{Макс} & \thead{CV, \%} & \thead{Асимм.} & \thead{Эксц.} \\
\midrule
pH\textsubscript{KCl} & --- & 6.99 & 7.30 & 0.66 & 5.30 & 8.00 & 9.4 & $-0.92$ & $-0.62$ \\
SOC & \% & 2.45 & 2.44 & 0.55 & 0.64 & 4.41 & 22.4 & $-0.26$ & 1.98 \\
NO$_3$ & мг/кг & 10.88 & 9.29 & 7.90 & 0.16 & 64.00 & 72.6 & 2.43 & 9.33 \\
P$_2$O$_5$ & мг/кг & 25.77 & 18.80 & 20.31 & 3.22 & 150.00 & 78.8 & 2.54 & 8.15 \\
K$_2$O & мг/кг & 650.6 & 664.0 & 166.7 & 106.0 & 1204.0 & 25.6 & $-0.49$ & 0.80 \\
S & мг/кг & 8.81 & 6.89 & 7.62 & 0.92 & 62.80 & 86.4 & 3.54 & 14.29 \\
\bottomrule
\end{tabular}
\end{table}

CV варьирует от 9.4\% (pH) до 86.4\% (S). По классификации Wilding (1985): pH --- низкая вариабельность ($<$15\%); SOC и K$_2$O --- умеренная (15--35\%); NO$_3$, P$_2$O$_5$ и S --- высокая ($>$35\%). Тест Шапиро--Уилка (Таблица~\ref{tab:normality}) подтвердил ненормальность всех распределений ($p < 10^{-12}$), обосновывая выбор непараметрической статистики.

\begin{table}[H]
\centering
\caption{Тест Шапиро--Уилка на нормальность распределений}
\label{tab:normality}
\small
\begin{tabular}{lccl}
\toprule
\thead{Свойство} & \thead{$W$} & \thead{$p$} & \thead{Вывод} \\
\midrule
pH\textsubscript{KCl} & 0.833 & $3.32 \times 10^{-32}$ & Ненормальное \\
SOC & 0.956 & $1.61 \times 10^{-17}$ & Ненормальное \\
NO$_3$ & 0.805 & $3.23 \times 10^{-34}$ & Ненормальное \\
P$_2$O$_5$ & 0.725 & $6.53 \times 10^{-39}$ & Ненормальное \\
K$_2$O & 0.976 & $2.44 \times 10^{-12}$ & Ненормальное \\
S & 0.592 & $1.10 \times 10^{-44}$ & Ненормальное \\
\bottomrule
\end{tabular}
\end{table}

\textbf{Межгодовые различия.} Тест Крускала--Уоллиса (Таблица~\ref{tab:kruskal}) выявил значимые различия между годами для четырёх из шести свойств: pH ($H = 139.0$, $p = 4.5 \times 10^{-32}$), NO$_3$ ($H = 155.3$, $p = 1.2 \times 10^{-35}$), P$_2$O$_5$ ($H = 27.4$, $p = 1.6 \times 10^{-7}$) и K$_2$O ($H = 19.2$, $p = 1.2 \times 10^{-5}$). Для SOC ($H = 1.27$, $p = 0.260$) и S ($H = 3.89$, $p = 0.049$) межгодовые различия \textbf{не достигли} строгого уровня значимости ($\alpha = 0.001$), что свидетельствует о временно\'{й} стабильности этих свойств в четырёхлетнем окне наблюдений.

\begin{table}[H]
\centering
\caption{Тест Крускала--Уоллиса: межгодовые различия}
\label{tab:kruskal}
\small
\begin{tabular}{lccl}
\toprule
\thead{Свойство} & \thead{$H$} & \thead{$p$} & \thead{Значимо ($\alpha = 0.001$)?} \\
\midrule
pH\textsubscript{KCl} & 138.96 & $4.49 \times 10^{-32}$ & \textbf{Да} \\
NO$_3$ & 155.34 & $1.18 \times 10^{-35}$ & \textbf{Да} \\
P$_2$O$_5$ & 27.43 & $1.63 \times 10^{-7}$ & \textbf{Да} \\
K$_2$O & 19.24 & $1.15 \times 10^{-5}$ & \textbf{Да} \\
SOC & 1.27 & 0.260 & Нет \\
S & 3.89 & 0.049 & Нет \\
\bottomrule
\end{tabular}
\end{table}

Стабильность SOC физически обоснована: органическое вещество аккумулируется десятилетиями и не подвержено краткосрочным колебаниям. Стабильность S (при пороговом $p = 0.049$) указывает на инерционный характер серного пула, определяемого минерализацией и атмосферными осаждениями.

\subsection{Интеркорреляция почвенных свойств}
\label{sec:intercorrelation_results}

Попарные корреляции Спирмена (Таблица~\ref{tab:intercorr}) выявили преимущественно слабые связи. Наиболее сильная --- pH$\leftrightarrow$K$_2$O ($\rho = -0.227$, $p < 10^{-14}$): кислые почвы севера формируются на глинистых субстратах с высоким калийным фондом (иллитовый резерв). Связь SOC$\leftrightarrow$S ($\rho = 0.176$, $p = 5.0 \times 10^{-9}$) и SOC$\leftrightarrow$NO$_3$ ($\rho = 0.148$, $p = 10^{-6}$) --- слабые, но высокозначимые, отражающие общий генезис: органическое вещество является источником и серы, и нитратного азота при минерализации.

Все корреляции P$_2$O$_5$ и K$_2$O с прочими свойствами $< |0.25|$, что подтверждает преимущественно антропогенную (агрохимическую) природу пространственного распределения подвижного фосфора и обменного калия: нормы внесения удобрений --- доминирующий фактор, не связанный напрямую с почвообразовательными процессами.

\begin{table}[H]
\centering
\caption{Интеркорреляция почвенных свойств (Спирмен, $n = 1085$)}
\label{tab:intercorr}
\small
\begin{tabular}{lcccccc}
\toprule
& pH & SOC & NO$_3$ & P$_2$O$_5$ & K$_2$O & S \\
\midrule
pH     & 1 & $-0.227^{***}$ & $-0.194^{***}$ & $0.128^{***}$ & $-0.227^{***}$ & $-0.054$ \\
SOC    &   & 1 & $0.148^{***}$ & 0.041 & $0.159^{***}$ & $0.176^{***}$ \\
NO$_3$ &   &   & 1 & $0.203^{***}$ & 0.104 & 0.091 \\
P$_2$O$_5$ & & & & 1 & 0.058 & 0.076 \\
K$_2$O &   &   &   &   & 1 & 0.041 \\
S      &   &   &   &   &   & 1 \\
\bottomrule
\multicolumn{7}{l}{\footnotesize $^{***}$ $p < 0.001$ (BH-скорректированное).}
\end{tabular}
\end{table}

\subsection{Корреляции с одиночными признаками ДЗЗ}
\label{sec:single_corr_results}

Из $>$3200~проверенных пар статистически значимые корреляции ($p_{\text{adj}} < 0.05$) обнаружены для подавляющего большинства признаков. В Таблице~\ref{tab:top_corr} представлены 5 наиболее сильных корреляций для каждого свойства.

\begin{table}[H]
\centering
\caption{Топ-5 корреляций (Спирмен $|\rho|$) для каждого свойства}
\label{tab:top_corr}
\small
\begin{tabular}{llccl}
\toprule
\thead{Свойство} & \thead{Признак ДЗЗ} & \thead{$\rho$} & \thead{$p_{\text{adj}}$} & \thead{Группа} \\
\midrule
\multirow{5}{*}{pH}
  & GNDVI\textsubscript{L8, весна} & $-0.670$ & $1.6 \times 10^{-142}$ & Спектральный \\
  & MAP                     & $+0.659$ & $5.8 \times 10^{-136}$ & Климатический \\
  & Уклон (slope)           & $+0.609$ & $6.8 \times 10^{-111}$ & Топографический \\
  & Аспект (sin)            & $-0.520$ & $4.2 \times 10^{-76}$  & Топографический \\
  & GNDVI\textsubscript{S2, весна} & $-0.520$ & $< 10^{-70}$    & Спектральный \\
\midrule
\multirow{5}{*}{SOC}
  & MSI\textsubscript{std сезонов} & $-0.350$ & $< 10^{-30}$ & Временно\'{й} \\
  & DEM                     & $+0.320$ & $< 10^{-25}$ & Топографический \\
  & GS\_precip              & $+0.290$ & $< 10^{-20}$ & Климатический \\
  & NDWI\textsubscript{S2, весна}  & $+0.280$ & $< 10^{-18}$ & Спектральный \\
  & SoilGrids SOC 0--5~см   & $+0.270$ & $< 10^{-17}$ & Педологический \\
\midrule
\multirow{5}{*}{K$_2$O}
  & BSI\textsubscript{S2, весна}   & $-0.478$ & $5.4 \times 10^{-63}$ & Спектральный \\
  & B4\textsubscript{Red, S2, весна} & $-0.420$ & $< 10^{-48}$ & Спектральный \\
  & B12\textsubscript{SWIR2, S2, весна} & $-0.410$ & $< 10^{-45}$ & Спектральный \\
  & Уклон                   & $-0.370$ & $< 10^{-35}$ & Топографический \\
  & MAP                     & $-0.350$ & $< 10^{-30}$ & Климатический \\
\midrule
\multirow{5}{*}{P$_2$O$_5$}
  & GS\_temp                & $+0.476$ & $1.9 \times 10^{-62}$ & Климатический \\
  & Аспект (cos)            & $+0.464$ & $5.0 \times 10^{-59}$ & Топографический \\
  & SoilGrids CEC 0--5~см   & $+0.380$ & $< 10^{-38}$ & Педологический \\
  & GNDVI\textsubscript{L8, лето}  & $+0.350$ & $< 10^{-30}$ & Спектральный \\
  & EVI\textsubscript{S2, лето}    & $+0.330$ & $< 10^{-28}$ & Спектральный \\
\midrule
\multirow{3}{*}{NO$_3$}
  & GLCM IDM\textsubscript{NIR, лето}  & $+0.290$ & $< 10^{-20}$ & Текстурный \\
  & GNDVI\textsubscript{L8, весна} & $+0.280$ & $< 10^{-18}$ & Спектральный \\
  & $\Delta$NDVI\textsubscript{лето$-$весна} & $+0.270$ & $< 10^{-17}$ & Временно\'{й} \\
\midrule
\multirow{2}{*}{S}
  & B2\textsubscript{Blue, весна}  & $-0.280$ & $< 10^{-18}$ & Спектральный \\
  & DEM                     & $+0.230$ & $< 10^{-13}$ & Топографический \\
\bottomrule
\end{tabular}
\end{table}

\textbf{Иерархия <<предсказуемости>>} по $|\rho|_{\max}$: pH~(0.670) $\gg$ K$_2$O~(0.478) $\approx$ P$_2$O$_5$~(0.476) $>$ SOC~(0.350) $>$ NO$_3$~(0.290) $>$ S~(0.280). Важно, что для pH три ведущих предиктора относятся к \emph{разным} группам (спектральный, климатический, топографический), тогда как для K$_2$O доминируют спектральные.

\subsection{Сезонная динамика корреляций}
\label{sec:seasonal_results}

Попарное сравнение $|\rho|$ весенних и летних аналогов показало:
\begin{itemize}[nosep]
  \item \textbf{pH}: весна информативнее лета для \textbf{6 из 8} общих индексов --- почва частично обнажена, спектральный сигнал содержит одновременно информацию о почве и ранней вегетации;
  \item \textbf{K$_2$O}: аналогичная тенденция --- весенние BSI и каналы Red/SWIR доминируют;
  \item \textbf{NO$_3$}: максимальная корреляция с \textbf{летними} индексами (NDVI, EVI) --- пик вегетации отражает обеспеченность азотом;
  \item \textbf{SOC}: относительно стабильная сезонная картина;
  \item \textbf{P$_2$O$_5$ и S}: слабая сезонная дифференциация.
\end{itemize}

Физическая интерпретация: в весенний период (апрель--май) почва Северного Казахстана ещё не полностью покрыта растительностью (NDVI $= 0.15$--$0.35$), и спектральный сигнал пикселя представляет линейную смесь почвенного и растительного отклика. Это создаёт <<окно информативности>> для свойств, непосредственно влияющих на яркость и цвет почвы (pH через карбонатность, K$_2$O через минералогический состав глин). Летом (NDVI $= 0.55$--$0.75$) сигнал доминируется растительностью, и информативность переключается на NO$_3$ --- ключевой лимитирующий нутриент, определяющий пиковую биомассу.

\subsection{Композитные vs одиночные признаки}
\label{sec:composite_results}

Систематическое сравнение 110~композитных признаков с одиночными индексами (Таблица~\ref{tab:composite_vs_single}) показало, что для подавляющего большинства свойств одиночные индексы остаются сильнейшими предикторами. Единственное устойчивое исключение --- K$_2$O, где GNDVI$\times$BSI\textsubscript{весна} ($\rho = -0.488$) превысил BSI\textsubscript{весна} ($\rho = -0.478$) на $\Delta|\rho| = +0.010$.

\begin{table}[H]
\centering
\caption{Сравнение лучших одиночных и композитных признаков}
\label{tab:composite_vs_single}
\small
\begin{tabular}{lcL{3.5cm}ccl}
\toprule
\thead{Свойство} & \thead{Одиноч.\\$|\rho|$} & \thead{Лучший\\композитный} & \thead{Композ.\\$|\rho|$} & \thead{$\Delta|\rho|$} & \thead{Композ.\\лучше?} \\
\midrule
pH     & 0.670 & mean\_GNDVI                    & 0.591 & $-0.079$ & Нет \\
SOC    & 0.350 & $\Delta$GNDVI\textsubscript{п.л.--весна} & 0.276 & $-0.074$ & Нет \\
NO$_3$ & 0.290 & GNDVI$-$NDRE\textsubscript{весна}        & 0.416 & $+0.126$ & \textbf{Да$^*$} \\
P$_2$O$_5$ & 0.476 & EVI$-$NDRE\textsubscript{весна}      & 0.390 & $-0.086$ & Нет \\
K$_2$O & 0.478 & GNDVI$\times$BSI\textsubscript{весна}    & 0.488 & $+0.010$ & \textbf{Да} \\
S      & 0.280 & mean\_NDVI                     & 0.360 & $+0.080$ & \textbf{Да$^*$} \\
\bottomrule
\multicolumn{6}{l}{\footnotesize $^*$ Для NO$_3$ и S лучший <<одиночный>> включает только чисто спектральные}\\
\multicolumn{6}{l}{\footnotesize \quad RS-признаки; SoilGrids-переменные (mn, mg) могут давать более высокий $|\rho|$.}
\end{tabular}
\end{table}

Физически GNDVI$\times$BSI для K$_2$O интерпретируется как произведение информации о хлорофилле (GNDVI) и яркости обнажённой почвы (BSI), что диагностирует контраст <<тёмные гумусированные глинистые почвы с высоким K>> vs <<светлые карбонатные с низким K>>. Прирост невелик ($+1\%$ абс.), что свидетельствует о том, что одиночные индексы уже хорошо захватывают основную информацию.

\subsection{Пространственная автокорреляция}
\label{sec:moran_results}

Глобальный индекс Морана (Таблица~\ref{tab:morans}) выявил \textbf{выраженную положительную пространственную автокорреляцию} для всех свойств ($I = 0.51$--$0.86$, все $p \approx 0$).

\begin{table}[H]
\centering
\caption{Глобальный индекс Морана~$I$}
\label{tab:morans}
\small
\begin{tabular}{lccccc}
\toprule
\thead{Свойство} & \thead{$n$} & \thead{Moran~$I$} & \thead{$E[I]$} & \thead{$z$-score} & \thead{$p$} \\
\midrule
pH\textsubscript{KCl} & 1085 & \textbf{0.861} & $-0.001$ & 169.3 & $\approx 0$ \\
S              & 1085 & \textbf{0.767} & $-0.001$ & 151.9 & $\approx 0$ \\
SOC            & 1085 & \textbf{0.684} & $-0.001$ & 134.7 & $\approx 0$ \\
P$_2$O$_5$    & 1085 & \textbf{0.676} & $-0.001$ & 133.5 & $\approx 0$ \\
NO$_3$         & 1085 & \textbf{0.628} & $-0.001$ & 124.1 & $\approx 0$ \\
K$_2$O         & 1085 & \textbf{0.510} & $-0.001$ & 100.4 & $\approx 0$ \\
\bottomrule
\end{tabular}
\end{table}

Значения $I$ существенно выше, чем в типичных полевых исследованиях ($I = 0.1$--$0.4$)~\citep{Kerry2010}, что обусловлено масштабом территории ($\sim$500~км, три области) и выраженными зональными градиентами чернозёмно-каштановой зоны.

\textbf{Примечательные паттерны:}
\begin{itemize}[nosep]
  \item pH имеет \textit{наивысший} Moran~$I$ (0.861) и \textit{наивысшую} долю между-полевой дисперсии (73.2\%, раздел~\ref{sec:variance_results}) --- последовательно зональное свойство;
  \item S имеет \textit{второй по величине} Moran~$I$ (0.767), но \textit{наименьшую} долю между-полевой дисперсии (22.6\%). Парадокс объясняется тем, что пространственная автокорреляция S обусловлена не между-полевым, а \textit{между-региональным} масштабом (кластеры хозяйств с различными нормами S-удобрений).
\end{itemize}

\textbf{Широтный градиент} (Таблица~\ref{tab:latitudinal}): pH закономерно убывает к северу ($\rho_{\text{lat}} = -0.409$, $p = 4.1 \times 10^{-45}$), SOC --- возрастает ($\rho_{\text{lat}} = +0.293$, $p = 5.7 \times 10^{-23}$), что отражает зональную биоклиматическую последовательность: нарастание увлажнённости к северу $\to$ усиление выщелачивания (снижение pH) и гумусонакопления (рост SOC).

\begin{table}[H]
\centering
\caption{Широтный градиент почвенных свойств}
\label{tab:latitudinal}
\small
\begin{tabular}{lccl}
\toprule
\thead{Свойство} & \thead{$\rho_{\text{широта}}$} & \thead{$p$} & \thead{Интерпретация} \\
\midrule
pH     & $-0.409$ & $4.1 \times 10^{-45}$ & Убывает к северу \\
SOC    & $+0.293$ & $5.7 \times 10^{-23}$ & Возрастает к северу \\
K$_2$O & $+0.22$  & $< 10^{-12}$ & Возрастает к северу \\
\bottomrule
\end{tabular}
\end{table}

\subsection{Иерархическая декомпозиция дисперсии}
\label{sec:variance_results}

Распределение дисперсии по компонентам (Таблица~\ref{tab:variance}) выявило критическое различие структуры изменчивости между свойствами.

\begin{table}[H]
\centering
\caption{Между-полевая и внутриполевая дисперсия ($n = 1085$, $k = 81$ хозяйство)}
\label{tab:variance}
\small
\begin{tabular}{lcccccc}
\toprule
\thead{Свойство} & \thead{$SS_{\text{total}}$} & \thead{$SS_{\text{between}}$} & \thead{$SS_{\text{within}}$} & \thead{\% Между} & \thead{\% Внутри} & \thead{ICC} \\
\midrule
pH     & 466.9  & 341.9 & 125.1 & \textbf{73.2} & 26.8 & 0.713 \\
K$_2$O & $3.01 \times 10^7$ & $1.98 \times 10^7$ & $1.03 \times 10^7$ & \textbf{65.6} & 34.4 & 0.632 \\
SOC    & 326.0  & 185.0 & 141.0 & 56.7 & 43.3 & 0.536 \\
P$_2$O$_5$ & $4.47 \times 10^5$ & $2.35 \times 10^5$ & $2.12 \times 10^5$ & 52.6 & 47.4 & 0.491 \\
NO$_3$ & $6.76 \times 10^4$ & $3.28 \times 10^4$ & $3.47 \times 10^4$ & 48.6 & 51.4 & 0.448 \\
S      & $6.29 \times 10^4$ & $1.42 \times 10^4$ & $4.87 \times 10^4$ & \textbf{22.6} & \textbf{77.4} & 0.166 \\
\bottomrule
\end{tabular}
\end{table}

pH демонстрирует \textbf{73.2\% между-полевой дисперсии} (ICC~=~0.71), что обосновывает наибольшую корреляцию с ДЗЗ-данными: крупномасштабные зональные градиенты (климат, материнская порода, рельеф) доминируют над стохастической внутриполевой неоднородностью. S, напротив, имеет лишь \textbf{22.6\% между-полевой дисперсии} (ICC~=~0.17), указывая на доминирование локальных антропогенных факторов (удобрения, пестициды), не детектируемых на данном пространственном разрешении ДЗЗ.

Корреляция ICC с $|\rho|_{\max}$ (Рис.~\ref{fig:icc_vs_rho}) подтверждает, что доля между-полевой дисперсии --- ключевой предиктор <<предсказуемости>> свойства по данным ДЗЗ.

\subsection{Конфаундинг-анализ: pH как медиатор}
\label{sec:confounding_results}

Анализ частных корреляций (Таблица~\ref{tab:confounding}) показал, что pH существенно опосредует связь SOC с рядом спектральных индексов.

\begin{table}[H]
\centering
\caption{Конфаундинг pH в корреляции SOC -- спектральные индексы}
\label{tab:confounding}
\small
\begin{tabular}{lcccl}
\toprule
\thead{Пара} & \thead{$\rho_0$} & \thead{$\rho_{\partial|\text{pH}}$} & \thead{$C_{\text{pH}}, \%$} & \thead{Интерпретация} \\
\midrule
SOC -- NDVI\textsubscript{лето} & 0.145 & 0.084 & \textbf{41.9\%} & pH конфаундит $>$40\% \\
SOC -- NDVI\textsubscript{позд. лето} & 0.082 & 0.093 & $-13.7\%$ & pH \emph{подавляет} связь \\
\bottomrule
\end{tabular}
\end{table}

Для летнего NDVI: после контроля pH \textbf{41.9\%} наблюдаемой корреляции SOC--NDVI исчезает. Каузальный механизм: pH через контроль доступности азота и микроэлементов (Fe, Mn, Zn) определяет вегетационную активность, которая затем ошибочно интерпретируется как <<индикатор>> SOC. Для позднего лета, напротив, pH \emph{подавляет} ($C = -13.7\%$) истинную корреляцию: после удаления pH-эффекта связь SOC--NDVI усиливается ($0.082 \to 0.093$), что свидетельствует о более прямом (менее опосредованном) влиянии SOC на позднесезонную продуктивность --- вероятно, через водоудерживающую способность.

Этот результат имеет критическое значение для DSM: наивное использование NDVI как предиктора SOC без учёта конфаундинга pH может привести к \textbf{систематической ошибке атрибуции} --- модель фактически предсказывает pH, а не SOC.

\subsection{Производные почвенные индикаторы}
\label{sec:derived_results}

Производные индикаторы (Таблица~\ref{tab:derived}) демонстрируют ряд корреляций, превышающих индивидуальные свойства.

\begin{table}[H]
\centering
\caption{Корреляции производных индикаторов с признаками ДЗЗ}
\label{tab:derived}
\small
\begin{tabular}{L{3.5cm}L{3.5cm}cl}
\toprule
\thead{Производный\\индикатор} & \thead{Признак ДЗЗ} & \thead{$\rho$} & \thead{Физ. смысл} \\
\midrule
P$_2$O$_5$/K$_2$O & Уклон & $-0.555$ & Тополатер. перераспр. \\
S/NO$_3$ & mean\_GNDVI & $+0.478$ & S--N баланс $\to$ вегетация \\
SOC$\times$NO$_3$ & GNDVI$-$NDRE\textsubscript{весна} & $-0.456$ & Пул орг.~N $\to$ хлорофилл \\
Индекс минерализации & NIR\textsubscript{L8, весна} & $-0.472$ & Нитрификация $\to$ NIR \\
fertility\_index & SoilGrids Mg 0--5~см & $+0.820$ & Минеральный фонд \\
\bottomrule
\end{tabular}
\end{table}

Наиболее примечательна корреляция P$_2$O$_5$/K$_2$O с уклоном ($\rho = -0.555$): комплексный баланс макроэлементов демонстрирует более сильную связь с рельефом, чем каждое свойство по отдельности (P$_2$O$_5$: $|\rho| = 0.27$; K$_2$O: $|\rho| = 0.37$), что свидетельствует о синергетическом эффекте. Индекс плодородия (нормированная комбинация всех целевых) показал $\rho = 0.820$ с SoilGrids Mg --- высочайшая корреляция в исследовании, хотя и тривиальная по природе (SoilGrids содержит глобальные оценки тех же свойств).
