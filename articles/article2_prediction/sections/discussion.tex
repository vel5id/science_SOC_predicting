% ============================================================
\section{Обсуждение}
\label{sec:discussion}
% ============================================================

\subsection{Превосходство ансамблевых моделей и сравнение с SoilGrids}
\label{sec:disc_ensemble}

GBDT и RF стабильно доминируют по метрикам Field-LOFO и пространственного split.
Разрыв между CART и RF ($\Delta\rho \approx 0.50$ для SOC) подчёркивает критическую роль ансамблирования: RF сочетает бэггинг (bootstrap-агрегацию деревьев) и метод случайных подпространств (random subspace)~\citep{Breiman2001}, что снижает как дисперсию, так и корреляцию между деревьями в ансамбле.
CatBoost демонстрирует лучший результат для P$_2$O$_5$ ($\rho = 0.600$) благодаря встроенной обработке ordered target encoding и категориальных переменных, что особенно актуально для свойства с высоким CV (72\%).
Линейные модели конкурентоспособны только для pH~--- единственного свойства, где зависимость спектральных индексов от концентрации близка к монотонной~\citep{Bartholomeus2008}.

\textbf{Базлайн с SoilGrids v2.0.} Для количественной оценки добавленной ценности
локального мультимодального моделирования проведено прямое сравнение предсказаний
лучших моделей с глобальным продуктом SoilGrids~v2.0~\citep{Poggio2021} через
REST API ISRIC (\url{https://rest.isric.org}) для тех же $n = 1051$~точек пробоотбора (Таблица~\ref{tab:soilgrids_baseline}).
Для pH SoilGrids показывает $\rho = 0.208$ и $R^2 = 0.034$, тогда как лучшая локальная модель (RF, Farm-LOFO) достигает $\rho = 0.750$ и $R^2 = 0.616$~--- превосходство в~\textbf{3.6$\times$} по~$\rho$ и~\textbf{18$\times$} по~$R^2$.
Для SOC SoilGrids демонстрирует $\rho = 0.042$~--- практически нулевую ранговую корреляцию;
экстремально отрицательный $R^2 = -15\,613$ обусловлен несовпадением единиц (SoilGrids:~г/кг, локальные данные:~\%),
однако unit-инвариантный~$\rho$ подтверждает отсутствие полезного сигнала на данном пространственном масштабе.
Результат согласуется с выводами~\citet{Hengl2021} о необходимости локальных моделей высокого разрешения для задач точного земледелия и подтверждает, что глобальные карты 250-метрового разрешения не обеспечивают адекватного агрохимического картирования на уровне отдельных полей в условиях Северного Казахстана.

\subsection{Разрыв между DL и ML: справедливое сравнение на едином split}
\label{sec:disc_dl_gap}

ResNet-18 на 18-канальных патчах ($64\!\times\!64$) уступает табличным ансамблям при обеих стратегиях валидации.
При Field-LOFO: $\rho = 0.699$ (scratch) vs $0.857$ (GBDT) для pH, $\rho = 0.374$ vs $0.735$ (ET) для SOC.

\textbf{Transfer Learning c ImageNet не решает проблему.}
Инициализация весами ImageNet (mean-tiling RGB$\to$18~каналов) повышает~$\rho$ лишь для NO$_3$~($0.511 \to 0.553$) и SOC~($0.374 \to 0.391$), но \emph{ухудшает} pH~($0.699 \to 0.559$), P$_2$O$_5$~($0.509 \to 0.390$) и~K$_2$O~($0.424 \to 0.380$).
При Farm-LOFO TL несколько уменьшает амплитуду деградации (P$_2$O$_5$: $\rho$~$0.200 \to 0.338$), но $\rho_{\max} = 0.338$ остаётся значительно ниже табличных ансамблей ($0.571$, SVR).

\textbf{Справедливое сравнение RF vs ConvNeXt (Таблица~\ref{tab:rf_vs_cnn}).}
Для исключения влияния различий в разбиении данных RF обучен на \emph{том же} field-level split (58/10/12), что и ConvNeXt.
RF превосходит обе конфигурации ConvNeXt для 4~из 6~свойств: pH ($R^2 = 0.828$ vs $0.798$), NO$_3$ ($0.614$ vs $0.575$), SOC ($0.502$ vs $0.501$), K$_2$O ($0.353$ vs $0.230$).
ConvNeXt (single-season) незначительно лучше только для P$_2$O$_5$ ($R^2 = 0.223$ vs $0.203$).
Метрика RPD подтверждает: лишь для pH RF достигает хорошей предсказательной способности (RPD~$= 2.42 > 2.0$), тогда как для остальных свойств RPD~$< 2.0$~(приемлемо для SOC и NO$_3$; непригодно для K$_2$O, P$_2$O$_5$, S).
CCC для pH равен 0.888~(умеренное согласие), для NO$_3$~--- 0.701, для SOC~--- 0.709.

Пять факторов объясняют этот разрыв:
\begin{enumerate}[nosep]
  \item \textbf{Размер выборки}: $n = 1071$ недостаточен для обучения глубоких архитектур без предобученных весов;
  \item \textbf{Отсутствие пространственной структуры}: при пиксельном входе CNN не может извлечь контекст окрестности;
  \item \textbf{Нерегулярная геометрия}: признаки агрегированы по полигонам полей, пространственная корреляция между патчами внутри поля создаёт информационную избыточность;
  \item \textbf{Индуктивное смещение}: decision-tree модели эффективнее обрабатывают гетерогенные табличные признаки (спектральные, текстурные, топографические), чем архитектуры, оптимизированные для гомогенных пространственных данных~\citep{Grinsztajn2022};
  \item \textbf{Доменный разрыв ImageNet}: веса ImageNet, обученные на 3-канальных RGB-фотографиях объектов, не переносятся на 18-канальные мультиспектральные спутниковые данные: mean-tiling разрушает спектральные корреляции, а высокоуровневые фичи ImageNet (текстуры, формы объектов) нерелевантны для спутниковой спектрометрии.
  Фундации, предобученные на спутниковых данных (SSL4EO-S12~\citep{Wang2023SSL4EO}, SatMAE, GFM), а также табличные DL-архитектуры (FT-Transformer~\citep{Gorishniy2021}, TabPFN~\citep{Hollmann2023}) представляют перспективное  направление для ЦПК.
\end{enumerate}

\subsection{Размер патча и роль NDVI}
\label{sec:disc_patch_ndvi}

Ablation study выявил, что 32$\times$32 оптимален для pH и NO$_3$, а 64$\times$64~--- для SOC и K$_2$O.
Патч 32$\times$32 ($\sim$320$\times$320~м при 10~м/пикс) примерно соответствует медианной площади поля (1.8~га), что обеспечивает баланс между пространственным контекстом и минимизацией шума от соседних полей.

NDVI~--- единственный индекс, системно улучшающий $R^2$ для pH (+8.8\%, с 0.807 до 0.878).
Этот результат согласуется с установленной зависимостью хлорофилловой активности от кислотности почвы, контролирующей доступность Fe, Mn и Al~\citep{Sims1996}.
Добавление $>$3~индексов приводит к деградации за счёт мультиколлинеарности и увеличения эффективной размерности каналов~\citep{Hughes1968}.

\subsection{Мультисезонный ConvNeXt: прирост для NO$_3$ и провал для P$_2$O$_5$}
\label{sec:disc_convnext}

Мультисезонные 54-канальные композиты обеспечили наибольший прирост для NO$_3$ ($R^2$: $0.422 \to 0.575$, \textbf{+36\%}).
Нитратный азот~--- наиболее мобильный макроэлемент; его концентрация в почве существенно изменяется в течение вегетационного сезона (минерализация весной, вымывание после осенних осадков), что делает мультисезонный спектральный профиль информативным предиктором~\citep{Morellos2016}.

Для P$_2$O$_5$ мультисезонный вариант дал отрицательный $R^2$ ($-0.221$).
При 54~каналах и $\sim$1071~патче модель переобучается: отношение каналов к выборке ($54/1071 \approx 0.05$) находится в зоне риска для CNN без предобучения.
P$_2$O$_5$ имеет наименьшую корреляцию с вегетационными индексами ($|\rho|_{\max} = 0.476$ с GS\_temp, а не со спектральными индексами), что усугубляет проблему.

\subsection{Иерархия валидации: от Field к Farm}
\label{sec:disc_validation}

Три стратегии валидации формируют чёткую иерархию строгости:
\begin{itemize}[nosep]
  \item \textbf{Field-LOFO} (81~фолд): наименее строгая, допускает корреляцию между полями одного хозяйства;
  \item \textbf{Пространственный split} (65/6/10): промежуточная; тестовые хозяйства географически изолированы;
  \item \textbf{Farm-LOFO} (20~хозяйств): наиболее строгая; каждое хозяйство тестируется изолированно.
\end{itemize}

Farm-LOFO-CV проведена для всех 11~ML-моделей на тех же 15~MDI-отобранных признаках, что и Field-LOFO (Таблица~\ref{tab:farm_lofo_all_rho}), что обеспечивает корректную изоляцию эффекта строгости валидации от влияния размерности признакового пространства и алгоритма.
При переходе от Field-LOFO к Farm-LOFO $\rho$ падает для всех свойств, однако масштаб падения существенно различается: pH ($0.798 \to 0.750$, RF, $-6\%$), P$_2$O$_5$ ($0.595 \to 0.490$, RF, $-18\%$), SOC ($0.731 \to 0.529$, RF, $-28\%$), K$_2$O ($0.624 \to 0.448$, RF, $-28\%$), S ($0.484 \to 0.289$, XGBoost/ET, $-40\%$), NO$_3$ ($0.775 \to 0.232$, RF, $-70\%$).
Результат согласуется с выводами~\citet{Roberts2017} о необходимости блочной CV для пространственно-коррелированных данных.

Масштаб падения метрик напрямую коррелирует с долей между-полевой дисперсии: для pH (ICC~=~0.71, наивысший среди свойств) падение $\rho$ составляет всего 6\%, тогда как для NO$_3$ (ICC~=~0.45)~--- 70\%.
Это подтверждает, что иерархическая декомпозиция дисперсии является надёжным предиктором <<устойчивости>> моделей к ужесточению стратегии валидации.
Однако данное наблюдение основано на 6~точках (по числу свойств) и не подкреплено формальным статистическим тестом; поэтому связь ICC--$\Delta\rho$ следует рассматривать как предварительную гипотезу, требующую валидации на большем числе свойств или датасетов.

Иерархия ранжирования моделей в целом сохраняется при переходе к Farm-LOFO: ансамблевые модели (RF, CatBoost, ET) лидируют для большинства свойств, однако конкретный лидер может меняться (GBDT лучший для pH при Field-LOFO, RF --- при Farm-LOFO; SVR неожиданно лидирует для P$_2$O$_5$ при Farm-LOFO).
Это свидетельствует о том, что падение метрик при Farm-LOFO определяется преимущественно \textbf{структурой данных} (пространственная автокорреляция), а не выбором алгоритма.

Контрольный эксперимент с per-fold MDI-отбором и GridSearchCV (Таблица~\ref{tab:perfold_rf}) выявляет второй источник оптимизма в основных табл.~результатов: \textbf{утечку через отбор признаков}.
При фиксированном наборе (отобранном на всём датасете) RF показывает $\rho = 0.750$ для pH (Farm-LOFO), тогда как при честном per-fold отборе --- всего $0.403$ ($-46\%$); для SOC снижение составляет $-68\%$, для NO\textsubscript{3} корреляция меняет знак.
Низкий IoU$_{\text{ref}}$ ($0.15$--$0.20$) подтверждает, что per-fold MDI систематически выбирает иные наборы признаков, чем фиксированный: последний, будучи оптимизирован на всей выборке, захватывает паттерны, не воспроизводимые при честном разбиении.
Исключением является S ($R^2 = 0.638$, IoU$_{\text{folds}} = 0.71$), где высокая стабильность отбора (7 признаков идентичны в 95--100\% фолдов) обеспечивает робастность; однако высокий RPD ($1.66$) и $R^2$ для S являются артефактами скошенного распределения (skewness~$= 2.8$): модель предсказывает значения, близкие к среднему, что формально снижает RMSE, но $\rho = 0.158$ свидетельствует об отсутствии реальной предсказательной способности.
Таким образом, per-fold MDI~+~Farm-LOFO даёт \emph{наиболее консервативные} оценки: даже для pH RPD падает до~$1.18$ (с~$1.62$ при фиксированных признаках), что указывает на вклад утечки через отбор.
Тем не менее при стандартном (фиксированном) отборе признаков Farm-LOFO подтверждает практическую применимость моделей для pH ($\rho = 0.750$, RPD~$= 1.62$) и умеренную~--- для SOC ($\rho = 0.554$, RPD~$= 1.16$); для остальных четырёх свойств (NO\textsubscript{3}, P\textsubscript{2}O\textsubscript{5}, K\textsubscript{2}O, S) результаты при экстраполяции на новые территории остаются недостаточными.

Для визуальной и количественной оценки Рисунок~\ref{fig:four_panel} сопоставляет пространственные предсказания pH для тестового хозяйства <<Агро Парасат>> (151 точка, Farm-LOFO --- модель не видела данные этого хозяйства при обучении) по четырём уровням:
(a)~наземные лабораторные измерения (ground truth),
(b)~RF при Field-LOFO (81~фолд),
(c)~RF при Farm-LOFO (20~фолдов, per-fold MDI + GridSearchCV),
(d)~CNN при Field-LOFO (ResNet-18, 81~фолд).

\begin{figure}[H]
  \centering
  \includegraphics[width=\textwidth]{fig_four_panel_comparison.png}
  \caption{Four-panel spatial prediction comparison for pH~(KCl) on a held-out farm (<<Агро Парасат>>, $N = 151$). (a)~Laboratory ground truth; (b)~RF Field-LOFO ($\rho = 0.798$); (c)~RF Farm-LOFO with per-fold MDI + GridSearchCV ($\rho = 0.403$); (d)~CNN (ResNet-18) Field-LOFO ($\rho = 0.699$). Basemap: Esri World Imagery; cubic interpolation with Gaussian noise ($\sigma \approx 4\%\,\text{std}$). Shared colour scale pH~5.5--8.0.}
  \label{fig:four_panel}
\end{figure}

Дополнительно, Таблица~\ref{tab:four_level} обобщает метрики по всем шести свойствам для четырёх уровней сравнения, включая аналитическую погрешность стандартных лабораторных методов (ГОСТ).

\begin{table}[H]
\centering
\caption{Four-level comparison: laboratory precision vs.\ best ML/DL models under two validation strategies. Lab RMSE is the approximate repeatability error of the analytical method (GOST norms). CNN = ResNet-18, 18-channel 64$\times$64 patches.}
\label{tab:four_level}
\small\resizebox{\textwidth}{!}{%
\begin{tabular}{l c cc cc cc}
\toprule
\multirow{2}{*}{\thead{Property}}
  & \textbf{Lab}
  & \multicolumn{2}{c}{\textbf{RF Field-LOFO}}
  & \multicolumn{2}{c}{\textbf{RF Farm-LOFO}}
  & \multicolumn{2}{c}{\textbf{CNN Field-LOFO}} \\
\cmidrule(lr){2-2}\cmidrule(lr){3-4}\cmidrule(lr){5-6}\cmidrule(lr){7-8}
  & RMSE
  & $\rho$ & RMSE
  & $\rho$ & RMSE
  & $\rho$ & RMSE \\
\midrule
pH          & $\sim$0.15 & \textbf{0.798} & 0.30 & 0.750 & 0.41 & 0.699 & 0.43 \\
SOC (\%)    & $\sim$0.15 & \textbf{0.731} & 0.39 & 0.529 & 0.46 & 0.391 & 0.56 \\
NO$_3$      & $\sim$1.5  & \textbf{0.775} & 5.00 & 0.232 & 6.70 & 0.553 & 6.44 \\
P$_2$O$_5$  & $\sim$4    & \textbf{0.595} & 15.4 & 0.490 & 16.5 & 0.509 & 17.1 \\
K$_2$O      & $\sim$50   & \textbf{0.624} & 121  & 0.448 & 146  & 0.424 & 149  \\
S           & $\sim$1.5  & 0.467 & 3.72 & 0.240 & 4.18 & 0.359 & 4.63 \\
\bottomrule
\end{tabular}}
\end{table}

Таблица~\ref{tab:four_level} наглядно демонстрирует три закономерности:
\textbf{(i)}~RMSE всех моделей в 2--5 раз превышает аналитическую погрешность стандартных лабораторных методов; даже лучший RF для pH (RMSE~$= 0.30$) даёт ошибку вдвое выше лабораторной воспроизводимости ($\sim$0.15);
\textbf{(ii)}~RF систематически превосходит CNN при Field-LOFO: для pH $\Delta\rho = +0.10$ ($0.798$ vs $0.699$), для SOC $\Delta\rho = +0.34$;
\textbf{(iii)}~для S ни один подход не обеспечивает практически значимого результата: ошибка в 2.5--3 раза превышает лабораторную при $\rho < 0.47$.

\subsection{Агрономическая интерпретация метрик качества}
\label{sec:disc_agronomic}

Для практического применения в точном земледелии абстрактные метрики ($R^2$, $\rho$) необходимо перевести в агрономические категории обеспеченности почв элементами питания.
Метрика RPD (Ratio of Performance to Deviation) дополнительно подтверждает иерархию предсказуемости: pH~--- единственное свойство с RPD~$> 2.0$ при Field-LOFO и справедливом сравнении с DL (Таблица~\ref{tab:rf_vs_cnn}), что соответствует <<хорошей>> предсказательной способности; SOC и NO$_3$ попадают в диапазон $1.4$--$2.0$ (<<приемлемо>>); P$_2$O$_5$, K$_2$O и S~--- ниже $1.4$ (<<непригодно>> для количественного предсказания) при строгих стратегиях валидации.
CCC подтверждает эту картину: pH (CCC~$= 0.888$)~--- умеренное согласие; SOC и NO$_3$ (CCC~$\approx 0.70$)~--- приемлемое; остальные свойства~--- ниже порога 0.65.
\begin{itemize}[nosep]
  \item \textbf{pH (кислотность)}: RMSE составляет 0.26--0.40 единиц pH. Агрономические классы кислотности обычно имеют ширину 0.5--1.0 единицы (например, слабокислые 5.1--5.5, близкие к нейтральным 5.6--6.0). Ошибка в 0.3--0.4 единицы позволяет модели с высокой надежностью классифицировать участки поля для дифференцированного внесения мелиорантов.
  \item \textbf{P$_2$O$_5$ (подвижный фосфор)}: RMSE составляет 15--18 мг/кг. По методу Мачигина классы обеспеченности имеют шаг около 15 мг/кг (низкая 16--30, средняя 31--45). Ошибка модели сопоставима с шириной одного класса. Это означает, что модель может ошибиться на один класс (например, предсказать «среднюю» вместо «низкой»), что приемлемо для базового зонирования, но требует осторожности при расчете точных доз удобрений.
  \item \textbf{NO$_3$ (нитратный азот)}: RMSE составляет 5--7 мг/кг при среднем значении по выборке 10.88 мг/кг. Ошибка составляет более 50\% от среднего значения, что делает спутниковые предсказания нитратов (особенно на новых территориях) непригодными для расчета доз азотных удобрений.
\end{itemize}

\subsection{Трудно предсказуемые свойства и практические рекомендации}
\label{sec:disc_hard}

Корреляционный скрининг ковариат ДЗЗ с целевыми свойствами (раздел~\ref{sec:feature_importance}) указывал на иерархию предсказуемости: pH $\gg$ K$_2$O $\approx$ P$_2$O$_5$ $>$ SOC $>$ NO$_3$ $>$ S. Результаты предиктивного моделирования полностью подтверждают эту иерархию: свойства с наибольшими корреляциями и долей между-полевой дисперсии (pH: $|\rho|_{\max} = 0.670$, ICC~=~0.71) демонстрируют наивысшее качество предсказания, тогда как свойства с низкими корреляциями и доминирующей внутриполевой изменчивостью (S: $|\rho|_{\max} = 0.280$, ICC~=~0.17) остаются непредсказуемыми.

Три свойства остаются <<трудными>> при строгой валидации:
\begin{itemize}[nosep]
  \item \textbf{S}: $\rho = 0.289$ (XGBoost, Farm-LOFO)~--- слабый результат (RPD~$= 1.64$, $R^2 = 0.629$). Умеренное $R^2$ объясняется скошенным распределением (skewness~$= 2.8$): модель предсказывает значения близкие к среднему для большинства образцов, что формально снижает MSE. Сера поступает преимущественно из атмосферных осаждений и удобрений~\citep{Scherer2001}, что не оставляет устойчивого спектрального следа;
  \item \textbf{SOC}: $\rho = 0.554$ (CatBoost, Farm-LOFO)~--- результат ниже данных литературы ($R^2 = 0.48$--$0.72$~\citep{Castaldi2019}), что объясняется малым диапазоном варьирования в исследуемом регионе (CV~$= 22.4\%$) и строгой стратегией группировки на уровне хозяйств;
  \item \textbf{NO$_3$}: $\rho = 0.232$ (RF, Farm-LOFO)~--- катастрофическое падение с $\rho = 0.775$ (Field-LOFO, $-70\%$), что демонстрирует доминирование пространственной автокорреляции в <<успехе>> моделей при менее строгой валидации.
\end{itemize}

\textbf{Практические рекомендации (Decision Tree).}
Для удобства принятия решений агрономами и специалистами по точному земледелию разработано дерево решений (Рисунок~\ref{fig:decision_tree}).
Для оперативного картографирования (внутрихозяйственное картирование) допустима стратегия Field-LOFO при условии калибровки на каждом хозяйстве (наличие хотя бы минимального пробоотбора).
Для \textbf{экстраполяции на новые территории} (без полевых данных) следует ориентироваться на метрики Farm-LOFO; в этом режиме pH ($\rho = 0.750$, RF) и P$_2$O$_5$ ($\rho = 0.571$, SVR) показывают приемлемую предсказуемость, SOC ($\rho = 0.554$, CatBoost) --- умеренную, а NO$_3$ ($\rho = 0.232$) --- слабую.

\begin{figure}[H]
\centering
\resizebox{0.95\textwidth}{!}{%
\begin{tikzpicture}[
  node distance=1.5cm and 2cm,
  decision/.style={diamond, draw, fill=blue!10, text width=3.2cm, text badly centered, inner sep=1pt, font=\small},
  block/.style={rectangle, draw, fill=green!10, text width=4.2cm, text centered, rounded corners, minimum height=1.2cm, font=\small},
  warning/.style={rectangle, draw, fill=orange!10, text width=4.2cm, text centered, rounded corners, minimum height=1.2cm, font=\small},
  stop/.style={rectangle, draw, fill=red!10, text width=4.2cm, text centered, rounded corners, minimum height=1.2cm, font=\small},
  line/.style={draw, -{Stealth[length=3mm]}, thick}
]

% Nodes
\node [decision] (task) {What is the mapping scenario?};
\node [block, below left=of task, xshift=-1.5cm] (extrapolate) {Extrapolation to new fields (no calibration data)};
\node [block, below right=of task, xshift=1.5cm] (interpolate) {Within-farm mapping (calibration available)};

\node [decision, below=of extrapolate] (prop_ext) {Which soil property?};
\node [block, below left=of prop_ext, xshift=-0.5cm] (ph_p) {pH, P$_2$O$_5$};
\node [warning, below=of prop_ext] (soc_k) {SOC, K$_2$O};
\node [stop, below right=of prop_ext, xshift=0.5cm] (no3_s) {NO$_3$, S};

\node [block, below=0.8cm of ph_p] (rec_ph) {Use RF\,/\,SVR. Error $\sim$1 nutrient class. Suitable for basic zoning.};
\node [warning, below=0.8cm of soc_k] (rec_soc) {Use CatBoost. Moderate accuracy. Use with caution.};
\node [stop, below=0.8cm of no3_s] (rec_no3) {Satellite models not applicable. Traditional soil sampling required.};

\node [block, below=of interpolate] (all_props) {All properties (except S)};
\node [block, below=0.8cm of all_props] (rec_all) {Use GBDT\,/\,RF\,/\,ET. High within-field accuracy. Suitable for precision agriculture.};

% Edges
\path [line] (task) -| node[above, pos=0.2, font=\small] {New territory} (extrapolate);
\path [line] (task) -| node[above, pos=0.2, font=\small] {Known territory} (interpolate);

\path [line] (extrapolate) -- (prop_ext);
\path [line] (prop_ext) -| node[above, pos=0.2, font=\small] {High predictability} (ph_p);
\path [line] (prop_ext) -- node[right, font=\small] {Moderate} (soc_k);
\path [line] (prop_ext) -| node[above, pos=0.2, font=\small] {Low} (no3_s);

\path [line] (ph_p) -- (rec_ph);
\path [line] (soc_k) -- (rec_soc);
\path [line] (no3_s) -- (rec_no3);

\path [line] (interpolate) -- (all_props);
\path [line] (all_props) -- (rec_all);

\end{tikzpicture}%
}
\caption{Decision tree for agronomists: choosing a digital soil mapping strategy based on the study results.}
\label{fig:decision_tree}
\end{figure}

\subsection{Ограничения}
\label{sec:limitations}

\begin{enumerate}[nosep]
  \item \textbf{Выборка}: 1085 образцов с 20~хозяйств одного региона; результаты не могут быть автоматически перенесены на иные почвенно-климатические зоны;
  \item \textbf{Временной горизонт}: данные 2022--2023~гг.; устойчивость моделей в долгосрочной перспективе не оценивалась;
  \item \textbf{Transfer learning}: протестирован ImageNet TL (mean-tiling RGB$\to$18~каналов), показавший неэффективность для данной задачи (раздел~\ref{sec:resnet_results}). Современные спутниковые фундации (SSL4EO-S12~\citep{Wang2023SSL4EO}, SatMAE, GFM), предобученные на обширных немеченых корпусах спутниковых данных, не тестировались и представляют наиболее перспективное направление для преодоления разрыва DL--ML при $n \sim 10^3$. Перспективна также полу-supervised стратегия самообучения с частично размеченными данными~\citep{Zhang2021b};
  \item \textbf{Исключение SoilGrids}: 42~признака SoilGrids~v2.0, извлечённые на этапе s06 пайплайна, были \textbf{полностью исключены} из пула кандидатов при отборе признаков, поскольку SoilGrids обучен на полевых данных из пересекающегося пространственного домена~\citep{Poggio2021}, что создаёт риск утечки целевой переменной. Таким образом, все 15~отобранных признаков для каждого свойства основаны исключительно на данных дистанционного зондирования (Sentinel-2, Landsat-8, Sentinel-1), топографии (SRTM) и климатических данных (ERA5-Land);
  \item \textbf{Временная утечка для всех свойств}: аудит temporal leakage проведён для~S и NO$_3$. Для NO$_3$~--- наиболее мобильного элемента, сильно зависящего от сезонной динамики минерализации и вымывания, использование летних/осенних спутниковых признаков для весенних образцов могло вносить аналогичную темпоральную утечку. Тем не менее резкое падение $\rho$ для NO$_3$ при Farm-LOFO ($0.775 \to 0.232$, $-70\%$) и сохранение низкого результата при использовании только весенних признаков ($\rho = 0.202$) свидетельствует о доминировании \emph{пространственной} компоненты над временной. SOC менее подвержен temporal leakage, поскольку является медленно изменяющимся свойством, однако полный аудит всех свойств является предметом будущей работы;
  \item \textbf{Farm-LOFO и число фолдов}: при 20~хозяйствах каждый фолд Farm-LOFO содержит крупный тестовый блок (3--17~полей); результаты подвержены высокой дисперсии, особенно для свойств с малым inter-farm сигналом (S, NO$_3$);
  \item \textbf{Фиксированные параметры пространственного split}: соотношение 65/6/10 жёсткое; альтернативные стратегии (кластерная CV, geographically weighted split) не исследовались;
  \item \textbf{Гиперпараметры DL}: оптимизация ConvNeXt проведена вручную; автоматический поиск (Optuna, Ray Tune) мог бы улучшить результаты;
  \item \textbf{Конфигурация моделей бустинга}: различия между GBDT (scikit-learn), XGBoost и CatBoost могут определяться не алгоритмом, а конфигурацией (GBDT без early stopping vs XGBoost с фиксированным $n\_est$); тест Фридмана (раздел~\ref{sec:friedman}) частично компенсирует это ограничение;
  \item \textbf{Карты предсказаний}: итоговые пространственные карты предсказанных свойств почв представлены только для pH на одном тестовом хозяйстве; полномасштабная генерация карт на весь регион и их независимая верификация являются предметом будущей работы.
\end{enumerate}
