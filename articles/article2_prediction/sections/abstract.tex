% ============================================================
\section*{Аннотация}
\addcontentsline{toc}{section}{Аннотация}
% ============================================================

\noindent
Цифровое почвенное картирование (ЦПК) на основе мультимодальных спутниковых данных является ключевым инструментом перехода к точному земледелию, однако для аридных и полуаридных зон Центральной Азии систематические исследования отсутствуют.
В данной работе представлено комплексное сравнение \textbf{11~моделей машинного обучения} (ML, Machine Learning) и \textbf{4~архитектур глубокого обучения} (DL, Deep Learning) для предсказания шести агрохимических свойств почв --- pH~(KCl), органического углерода почвы (SOC, Soil Organic Carbon), NO$_3$, P$_2$O$_5$, K$_2$O и~S --- на сельскохозяйственных угодьях Северного Казахстана ($n = 1085$ образцов, 81~поле, 20~хозяйств, 2022--2023~гг.).
Набор из \textbf{530~мультимодальных признаков} извлечён из Sentinel-2, Landsat-8, Sentinel-1~SAR, SRTM~DEM и ERA5-Land через автоматизированный пайплайн в Google Earth Engine (GEE); признаки SoilGrids~v2.0 извлечены, но исключены из моделирования для предотвращения утечки целевой переменной.
Все модели оценены тремя стратегиями пространственной валидации нарастающей строгости: \textbf{Leave-One-Field-Out CV} (LOFO-CV, 81~фолд), \textbf{Leave-One-Farm-Out CV} (Farm-LOFO, 20~фолдов) и \textbf{оптимизированный пространственный split} (65/6/10~хозяйств).

Наилучшие результаты при Field-LOFO-CV достигнуты ансамблевыми методами: GBDT для pH ($\rho = 0.857$), ET для SOC ($\rho = 0.735$) и RF для NO$_3$ ($\rho = 0.775$).
Ablation study на CNN выявило оптимальный размер патча 32$\times$32 пикселей и значимый прирост от добавления единственного индекса NDVI.
Мультисезонный ConvNeXt с блоками Squeeze-and-Excitation (SE) на 54-канальных композитах улучшил предсказание NO$_3$ на 36\%, подтверждая ценность временно́й динамики для азотзависимых свойств.
При переходе к строжайшей Farm-LOFO-CV (все 11~моделей, 20~фолдов) pH сохраняет высокую предсказательную силу ($\rho = 0.750$, RF), SOC и P$_2$O$_5$ --- умеренную ($\rho = 0.554$ и $0.571$), тогда как NO$_3$ катастрофически теряет качество ($\rho = 0.232$, $-70\%$), что свидетельствует о доминировании пространственной автокорреляции в Field-LOFO метриках.
Тест Фридмана с пост-хок анализом Немени на 81~фолде подтвердил статистическую значимость различий между ансамблями и линейными моделями.
Расширенный аудит утечки данных для серы выявил артефактное завышение $R^2$ при низком $\rho$, обусловленное скошенным распределением.

\medskip
\noindent\textbf{Ключевые слова:} цифровое почвенное картирование, машинное обучение, глубокое обучение, Random Forest, ConvNeXt, Squeeze-and-Excitation, Sentinel-2, пространственная кросс-валидация, LOFO-CV, Северный Казахстан, точное земледелие

\medskip
\noindent\textbf{Аббревиатуры:} DSM~--- Digital Soil Mapping; ML~--- Machine Learning; DL~--- Deep Learning; RF~--- Random Forest; ET~--- Extra Trees; GBDT~--- Gradient Boosted Decision Trees; CNN~--- Convolutional Neural Network; SE~--- Squeeze-and-Excitation; LOFO-CV~--- Leave-One-Field-Out Cross-Validation; GEE~--- Google Earth Engine; SOC~--- Soil Organic Carbon; NDVI~--- Normalized Difference Vegetation Index; SAR~--- Synthetic Aperture Radar; DEM~--- Digital Elevation Model; OOF~--- Out-Of-Fold.
