% ============================================================
\section{Введение}
\label{sec:introduction}
% ============================================================

Почвенные ресурсы являются фундаментальной основой устойчивого сельскохозяйственного производства~\citep{Montanarella2016}. Деградация почв затрагивает около трети мировых земельных ресурсов, а экономические потери от снижения почвенного плодородия оцениваются в триллионы долларов ежегодно.
Точное земледелие~--- парадигма оптимизации агрохимических вмешательств на основе пространственно-детальной информации~\citep{Gebbers2010}~--- требует оперативного мониторинга агрохимических свойств верхнего горизонта.
Традиционные методы оценки, базирующиеся на полевом пробоотборе и лабораторном анализе (стоимость 15--50~USD/образец, срок~2--4~недели), остаются непрактичными для обширных территорий с высоким временным разрешением.

Северный Казахстан~--- один из крупнейших зернопроизводящих регионов мира ($>$20~млн~га пашни), характеризующийся резко континентальным климатом, коротким вегетационным периодом (130--150~дней) и высокой уязвимостью чернозёмно-каштановых почв к деградации~\citep{Swinnen2017, Kraemer2015}.
Несмотря на экономическую значимость, систематических исследований цифрового почвенного картирования (ЦПК) для Центральной Азии практически не проводилось.

\subsection{Машинное обучение в задачах ЦПК}

Современное ЦПК базируется на концептуальной модели SCORPAN~\citep{McBratney2003, Minasny2016}, связывающей свойства почв с ковариатами окружающей среды: $\text{Soil} = f(s, c, o, r, p, a, n)$.
Ансамблевые алгоритмы~--- Random Forest~\citep{Breiman2001}, градиентный бустинг~\citep{Friedman2001}, XGBoost~\citep{Chen2016}, CatBoost~\citep{Dorogush2018}~--- зарекомендовали себя как наиболее эффективные для предсказания pH, SOC и содержания макроэлементов.
Мета-анализ~\citet{Wadoux2020} показал, что Random Forest является наиболее часто используемым алгоритмом в ЦПК, обеспечивая медианный $R^2 = 0.50$--$0.65$ для SOC и $0.55$--$0.75$ для~pH.
Сравнительная оценка SVR, ANN и RF~\citep{Were2015} подтвердила стабильность ансамблей для картирования SOC; выбор оптимального алгоритма для конкретной задачи ЦПК подробно обсуждается в~\citet{Khaledian2020}.
При этом абсолютное большинство исследований ограничиваются 1--3~алгоритмами и не проводят систематического сравнения на едином датасете.

\subsection{Проблема пространственной утечки данных}

Критической методологической проблемой является использование случайного разбиения на обучающую и тестовую выборки (random split), не учитывающего пространственную автокорреляцию между образцами.
\citet{Roberts2017} и~\citet{Wadoux2021} убедительно показали, что такой подход систематически завышает оценки качества на 15--40\%.
Пространственные стратегии Leave-One-Group-Out (LOGO) и Leave-One-Field-Out (LOFO) обеспечивают наиболее реалистичные оценки генерализующей способности моделей~\citep{Meyer2021}.
Более строгие иерархические схемы (Leave-One-Farm-Out) позволяют оценить обобщаемость на уровне целых хозяйств~--- сценарий, наиболее релевантный для практического масштабирования ЦПК.

\subsection{Глубокое обучение для почвенного картирования}

Глубокое обучение представляет перспективное, но существенно менее изученное направление в ЦПК.
\citet{Padarian2019, Padarian2019b} применили полносвязные и глубокие сети к табличным почвенным признакам.
\citet{Behrens2018} использовали CNN на суперпиксельных патчах рельефа.
\citet{TaghizadehMehrjardi2020} показали превосходство мультизадачной CNN над RF для картирования гранулометрического состава в засушливом Иране.
\citet{Tsakiridis2020} применили одномерные CNN для одновременного предсказания нескольких свойств почв из~VNIR-SWIR спектров, а~\citet{Zhong2021}~--- глубокие сверточные сети к библиотеке LUCAS, показав, что CNN, обученные end-to-end на мультиспектральных патчах, способны извлекать пространственные паттерны, недоступные при табличном моделировании, однако преимущество проявляется при $>$5000~образцов.
Архитектуры ConvNeXt~\citep{Liu2022} с depthwise convolution, Layer Normalization и Squeeze-and-Excitation (SE) блоками~\citep{Hu2018} ранее не применялись к мультисезонным мультиспектральным почвенным данным.

\subsection{Исследовательские лакуны и цели работы}

Предварительный корреляционный скрининг 530~ковариат ДЗЗ (раздел~\ref{sec:feature_importance}) показал существенные различия в максимальной ранговой корреляции между свойствами (от $|\rho|_{\max} = 0.670$ для~pH до 0.280 для~S), что определяет выбор целевых свойств и стратегии моделирования.

Анализ существующих работ выявил четыре ключевых разрыва:
\begin{enumerate}[nosep]
  \item \textbf{Географический}: для Центральной Азии~--- региона с чернозёмно-каштановыми почвами и резко континентальным климатом~--- систематических исследований мультимодального ЦПК не проводилось; существующие работы для засушливых зон охватывают преимущественно Иран~\citep{Zeraatpisheh2019} и ограниченные регионы Ближнего Востока;
  \item \textbf{Методологический}: комплексное сравнение $>$10~ML-алгоритмов и нескольких DL-архитектур с иерархической пространственной валидацией остаётся редкостью~\citep{Nussbaum2018, Vaysse2015};
  \item \textbf{Архитектурный}: потенциал ConvNeXt с SE-блоками на мультисезонных мультиканальных патчах не исследован; ablation study размера патча для почвенных задач отсутствуют;
  \item \textbf{Целевой}: предсказание макро- и мезоэлементов (NO$_3$, P$_2$O$_5$, K$_2$O, S) изучено значительно хуже, чем SOC и~pH; большинство работ по дистанционному картированию SOC~\citep{Xu2020, Chen2024, Wang2021} сосредоточены на одном регионе или спектральном диапазоне, не захватывая полный спектр почвенных питательных веществ.
\end{enumerate}

\noindent\textbf{Цели настоящей работы:}
Настоящая работа концентрируется на построении предиктивных моделей и их строгой пространственной валидации.
\begin{enumerate}[nosep]
  \item Систематическое сравнение \textbf{11~ML-моделей} и \textbf{4~DL-архитектур} для предсказания шести агрохимических свойств;
  \item Применение \textbf{трёхуровневой пространственной валидации} (Field-LOFO, Farm-LOFO, оптимизированный spatial split) для количественной оценки пространственной утечки;
  \item Первое \textbf{ablation study} влияния размера патча (16$\times$16, 32$\times$32, 64$\times$64) и числа спектральных индексов на качество CNN для ЦПК;
  \item Разработка и оценка \textbf{мультисезонного ConvNeXt} с SE-блоками на 54-канальных композитах;
  \item Расширенный аудит утечки данных с differentiальным анализом временно́й и пространственной компонент.
\end{enumerate}
