% ============================================================
\section{Заключение}
\label{sec:conclusion}
% ============================================================

В настоящей работе проведено систематическое сравнение 11~ML- и 4~DL-моделей для предсказания шести агрохимических свойств почвы (pH, SOC, NO$_3$, P$_2$O$_5$, K$_2$O, S) по мультиспектральным спутниковым данным (Sentinel-2, Landsat~8, Sentinel-1) с трёхуровневой иерархической валидацией.
Основные результаты:

\begin{enumerate}[nosep]
  \item \textbf{Ансамблевые модели (RF, GBDT, ET) превосходят все альтернативы} при табличном моделировании.
        GBDT достигает $\rho = 0.857$ для pH, RF~--- $\rho = 0.775$ для NO$_3$ (Field-LOFO-CV).
        Разрыв с одиночным деревом (CART) составляет до $\Delta\rho = 0.50$ (SOC).

  \item \textbf{CNN конкурентоспособны только для отдельных свойств.}
        Ablation study показал, что CNN с патчами 32$\times$32 + NDVI достигает $R^2 = 0.878$ для pH, но уступает ансамблям для SOC и NO$_3$.

  \item \textbf{Мультисезонный ConvNeXt с SE-блоками обеспечивает +36\% прирост $R^2$ для NO$_3$}
        ($0.422 \to 0.575$), но переобучается для P$_2$O$_5$ ($R^2 = -0.221$).
        Мультисезонные данные наиболее полезны для мобильных элементов.

  \item \textbf{Иерархическая валидация (11~моделей $\times$ 3~стратегии) выявляет масштаб пространственной утечки:}
        при переходе от Field-LOFO к Farm-LOFO $\rho$ для pH падает на 6\% (RF: $0.798 \to 0.750$), для SOC~--- на 28\% (RF: $0.731 \to 0.529$), для NO$_3$~--- на 70\% (RF: $0.775 \to 0.232$).
        Farm-LOFO для всех 11~моделей подтвердил, что масштаб падения определяется структурой данных, а не алгоритмом.

  \item \textbf{Сера (S) трудно предсказуема из спутниковых данных}: $\rho = 0.289$ (XGBoost, Farm-LOFO) при артефактно завышенном $R^2 = 0.629$ из-за скошенного распределения; аудит выявил temporal leakage.

  \item \textbf{Для оперативного применения рекомендуются RF и CatBoost} на 15~отобранных признаках с калибровкой на каждом хозяйстве.
        Для экстраполяции на новые территории pH ($\rho = 0.750$, RF, Farm-LOFO) и P$_2$O$_5$ ($\rho = 0.571$, SVR) показывают приемлемую предсказуемость; SOC ($\rho = 0.554$, CatBoost)~--- умеренную.
\end{enumerate}

\textbf{Направления дальнейших исследований:}
(a)~transfer learning с предобученными моделями (SSL4EO-S12, SatMAE, GFM) для преодоления ограничения малых выборок;
(b)~включение временны\'{х} SAR-признаков (Sentinel-1 coherence, интерферометрическая фаза) для улучшения предсказания влажностно-зависимых свойств;
(c)~расширение выборки на сопредельные почвенно-климатические зоны (лесостепь, степь) для оценки переносимости моделей;
(d)~генерация пространственных карт предсказанных свойств и их верификация полевыми данными.

\medskip
\noindent Данное исследование формирует целостный фреймворк~--- от статистического анализа пространственных зависимостей до построения предиктивных моделей~--- для цифрового почвенного картирования аридных и полуаридных агроэкосистем Центральной Азии.

% ============================================================
\section*{Доступность данных и кода}
\addcontentsline{toc}{section}{Доступность данных и кода}
% ============================================================

Спутниковые данные получены из открытых источников: Sentinel-2 и Sentinel-1 (Copernicus Open Access Hub), Landsat-8 (USGS EarthExplorer), SRTM DEM (NASA), SoilGrids~v2.0 (ISRIC), ERA5-Land (Copernicus CDS). Почвенные данные содержат коммерчески чувствительную информацию о хозяйствах и доступны по обоснованному запросу к корреспондирующему автору. Исходный код пайплайна обработки данных и обучения моделей доступен в открытом репозитории: \url{https://github.com/vel5id/science_SOC_predicting}.

% ============================================================
\section*{Вклад авторов}
\addcontentsline{toc}{section}{Вклад авторов}
% ============================================================

\noindent \textbf{\textcolor{red}{ЗАПОЛНИТЬ ПЕРЕД ПОДАЧЕЙ:}} \\[0.3em]
\noindent \textit{[Имя Фамилия]}: Conceptualization, Methodology, Software, Formal Analysis, Investigation, Data Curation, Writing~--- Original Draft, Visualization. \\
\textit{[Имя Фамилия]}: Writing~--- Review \& Editing, Supervision, Project Administration, Funding Acquisition.

% ============================================================
\section*{Финансирование}
\addcontentsline{toc}{section}{Финансирование}
% ============================================================

\noindent \textbf{\textcolor{red}{ЗАПОЛНИТЬ ПЕРЕД ПОДАЧЕЙ:}} \\[0.3em]
\noindent \textit{[Указать источники финансирования и номера грантов.]}

% ============================================================
\section*{Конфликт интересов}
\addcontentsline{toc}{section}{Конфликт интересов}
% ============================================================

\noindent Авторы заявляют об отсутствии конфликта интересов.
