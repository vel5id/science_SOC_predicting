% ============================================================
\section{Район исследования и данные}
\label{sec:study_area}
% ============================================================

\subsection{Район исследования}

Исследование проведено на сельскохозяйственных угодьях Северного Казахстана в пределах Костанайской, Акмолинской и Северо-Казахстанской областей (50°--54°~с.ш., 64°--72°~в.д.)~--- части Казахстанской целинной зоны с~$\sim$80\% пашни страны.

\textbf{Климат}: резко континентальный (Dfb/BSk по Кёппену); среднегодовая температура $+0.5$\ldots$+3.0$~°C; осадки 250--400~мм/год; вегетационный период 130--150~дней.
\textbf{Почвы}: чернозёмы обыкновенные и южные (Haplic Chernozems), каштановые (Kastanozems); содержание гумуса 2--6\%; pH от~5.3 до~8.0; гранулометрический состав~--- средний и тяжёлый суглинок.
\textbf{Рельеф}: слабоволнистая равнина, 180--420~м над~ур.~моря, уклоны 0--3°.
\textbf{Сельское хозяйство}: яровая пшеница, ячмень, масличные; 15--45\% площади~--- no-till; норма NP~--- до 90~кг/га~N, 40~кг/га~P.

\begin{figure}[H]
  \centering
  \includegraphics[width=0.90\textwidth]{fig1_study_area.png}
  \caption{Study area map. Northern Kazakhstan (Kostanay, Akmola, and North Kazakhstan regions). Points indicate the location of 81~sampled fields; symbol size is proportional to the number of samples per field.}
  \label{fig:study_area}
\end{figure}

\subsection{Почвенные данные}
\label{sec:soil_data}

Набор данных содержит \textbf{1085~образцов} пахотного горизонта (0--25~см), собранных на \textbf{20~хозяйствах}, объединяющих \textbf{81~уникальное поле}.
В 2022~г. обследовано 174~образца на 18~полях 5~хозяйств; в 2023~г.~--- 911~образцов на 64~полях 15~хозяйств (одно поле обследовано в оба года, что даёт $18 + 64 - 1 = 81$ уникальное поле).
На каждом поле пробоотбор проводился по регулярной сетке с шагом~$\sim$200~м и отступом от границ не менее 20~м.
Число точек сетки варьировало от 3 до 20 в зависимости от площади поля (20--50~га), что определяет количество образцов на поле.
Для каждой точки отбиралась смешанная проба из 15--20~уколов в радиусе~$\sim$10~м.
На каждом поле определены все шесть целевых свойств: pH~(KCl), гумус (SOC), NO$_3$, P$_2$O$_5$, K$_2$O и~S.
Пробоотбор проведён в два этапа: весенний (март--май; 817~образцов) и послеуборочный (сентябрь--октябрь; 268~образцов).
Последствия неравномерного временного распределения пробоотбора для экстракции спутниковых признаков обсуждаются в разделе~\ref{sec:leakage_audit}.

\textbf{Иерархическая структура выборки.}
Данные имеют иерархическую структуру: 20~хозяйств $\supset$ 81~поле $\supset$ 1085~образцов.
Число полей на хозяйство варьирует от 1 до 17 (медиана~--- 3.5), число образцов на поле~--- от 3 до 20 (медиана~--- 11).
Эта иерархия определяет стратегию пространственной кросс-валидации (раздел~\ref{sec:cv_strategy}): группировка по полям (Field-LOFO) изолирует внутриполевую автокорреляцию, а группировка по хозяйствам (Farm-LOFO)~--- межполевую корреляцию внутри одного хозяйства.

\textbf{Временная неоднородность.}
Выборка включает 174~образца из 2022~г. и 911 из 2023~г.
Климатические условия двух сезонов различались: в 2022~г. наблюдалось умеренное избыточное увлажнение в первой половине вегетации, тогда как 2023~г. характеризовался более засушливым началом лета.
Межгодовая изменчивость климата может влиять на спутниковые индексы (NDVI, MSI) и, как следствие, на качество предсказания~--- данный эффект частично компенсируется включением климатических признаков ERA5-Land (GS\_temp, GS\_precip) в набор ковариат.

Лабораторные определения: pH~(KCl)~--- потенциометрия (ГОСТ~26483-85); SOC~--- метод Тюрина (ГОСТ~26213-91); NO$_3$~--- ионометрия (ГОСТ~26951-86); P$_2$O$_5$ и K$_2$O~--- метод Мачигина (ГОСТ~26205-91); S~--- турбидиметрия.
Описательная статистика агрохимических свойств представлена в Таблице~\ref{tab:descriptive}.

\begin{table}[H]
\centering
\caption{Descriptive statistics of agrochemical properties ($n = 1085$)}
\label{tab:descriptive}
\small
\begin{tabular}{lccccccccc}
\toprule
\thead{Property} & \thead{Unit} & \thead{Mean} & \thead{Median} & \thead{SD} & \thead{Min} & \thead{Max} & \thead{CV, \%} & \thead{Skew.} & \thead{Kurt.} \\
\midrule
pH (KCl)   & ---    & 6.99  & 7.30  & 0.66  & 5.30  & 8.00   & 9.4  & $-0.92$ & $-0.62$ \\
SOC        & \%     & 2.45  & 2.44  & 0.55  & 0.64  & 4.41   & 22.4 & $-0.26$ & 1.98  \\
NO$_3$     & mg/kg  & 10.88 & 9.29  & 7.90  & 0.16  & 64.00  & 72.6 & 2.43    & 9.33  \\
P$_2$O$_5$ & mg/kg  & 25.77 & 18.80 & 20.31 & 3.22  & 150.00 & 78.8 & 2.54    & 8.15  \\
K$_2$O     & mg/kg  & 650.6 & 664.0 & 166.7 & 106.0 & 1204.0 & 25.6 & $-0.49$ & 0.80  \\
S          & mg/kg  & 8.81  & 6.89  & 7.62  & 0.92  & 62.80  & 86.4 & 3.54    & 14.29 \\
\bottomrule
\end{tabular}
\end{table}

Тест Шапиро--Уилка подтвердил ненормальность всех шести свойств ($p < 0.001$), что обосновывает использование непараметрического коэффициента Спирмена ($\rho$) как основной метрики ранжирования моделей.
