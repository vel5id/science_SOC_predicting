% ============================================================
%  Статья 2 (Q1–Q2)
%  ПРЕДСКАЗАНИЕ АГРОХИМИЧЕСКИХ СВОЙСТВ ПОЧВ С ИСПОЛЬЗОВАНИЕМ
%  МУЛЬТИМОДАЛЬНЫХ СПУТНИКОВЫХ ДАННЫХ И МЕТОДОВ ML/DL
% ============================================================
\documentclass[a4paper,11pt]{article}

% ============ ENCODING & FONTS ============
\usepackage{fontspec}
\setmainfont{DejaVu Serif}
\setsansfont{DejaVu Sans}
\setmonofont{DejaVu Sans Mono}

% ============ LANGUAGE ============
\usepackage{polyglossia}
\setdefaultlanguage{russian}
\setotherlanguage{english}

% ============ PAGE LAYOUT ============
\usepackage[margin=2.5cm]{geometry}
\usepackage{setspace}
\onehalfspacing

% ============ GRAPHICS ============
\usepackage{graphicx}
\graphicspath{{./figures/}{../../figures/}{../../math_statistics/output/}}
\usepackage{float}

% ============ TABLES ============
\usepackage{booktabs}
\usepackage{tabularx}
\usepackage{longtable}
\usepackage{array}
\usepackage{multirow}
\newcolumntype{C}[1]{>{\centering\arraybackslash}p{#1}}
\newcolumntype{L}[1]{>{\raggedright\arraybackslash}p{#1}}
\usepackage{makecell}
\renewcommand\theadfont{\bfseries\small}

% ============ CAPTIONS ============
\usepackage[font=small,labelfont=bf]{caption}

% ============ BIBLIOGRAPHY ============
\usepackage[numbers,sort&compress]{natbib}
\bibliographystyle{unsrtnat}

% ============ HYPERLINKS ============
\usepackage[colorlinks=true,linkcolor=blue,citecolor=blue,urlcolor=blue]{hyperref}

% ============ SECTIONS ============
\usepackage{titlesec}
\titleformat{\section}{\large\bfseries}{\thesection.}{0.5em}{}
\titleformat{\subsection}{\normalsize\bfseries}{\thesubsection.}{0.5em}{}
\titleformat{\subsubsection}{\normalsize\itshape}{\thesubsubsection.}{0.5em}{}

% ============ CODE BLOCKS ============
\usepackage{listings}
\usepackage{xcolor}
\lstset{
  basicstyle=\ttfamily\footnotesize,
  breaklines=true,
  frame=single,
  backgroundcolor=\color[gray]{0.95},
  xleftmargin=1em,
  framexleftmargin=0.5em,
}

% ============ MISC ============
\usepackage{amsmath,amssymb}
\usepackage{enumitem}
\usepackage{tikz}
\usetikzlibrary{shapes.geometric, arrows.meta, positioning, fit, backgrounds}

% ============================================================
\begin{document}

% ============ TITLE ============
\begin{center}
{\LARGE\bfseries Цифровое почвенное картирование степной зоны Северного Казахстана:\\[4pt] Предсказание агрохимических свойств почв с использованием мультимодальных спутниковых данных и методов машинного и глубокого обучения}

\vspace{1.5em}
{\large [Авторы]}\\[0.5em]
{\normalsize [Аффилиация]}
\vspace{2em}
\end{center}

% ============================================================
\section*{Аннотация}
\addcontentsline{toc}{section}{Аннотация}
% ============================================================

\noindent\textit{Данная работа является первой из двух сопутствующих публикаций, посвящённых цифровому почвенному картированию степной зоны Северного Казахстана. В Части~1 выполнен комплексный корреляционный анализ, формирующий эмпирическую и методологическую базу. В Части~2 результаты используются для построения и сравнения предиктивных моделей машинного и глубокого обучения.}

\medskip
Цифровое почвенное картирование (ЦПК) базируется на статистически обоснованных зависимостях между свойствами почв и ковариатами окружающей среды, однако для степной зоны Центральной Азии эти зависимости исследованы недостаточно.
В настоящей работе выполнен комплексный корреляционный анализ между шестью агрохимическими свойствами пахотного горизонта (pH\textsubscript{KCl}, SOC, NO$_3$, P$_2$O$_5$, K$_2$O, S; $n = 1085$, 186~полей, 81~хозяйство, 2020--2023~гг.) и набором из \textbf{536 мультимодальных признаков}, извлечённых через Google Earth Engine из шести источников: Sentinel-2, Landsat-8, Sentinel-1~SAR, SRTM~DEM, SoilGrids~v2.0 и ERA5-Land. Признаковое пространство включает сезонные медианы спектральных каналов и индексов, временны\'{е} характеристики (межсезонные дельты, амплитуды), текстурные GLCM-метрики, топографические, климатические и педологические переменные, а также \textbf{110 композитных спектральных признаков} (попарные произведения индексов, мультисезонные дельты, нормализованные разности полос).

Для каждой из $>3200$ пар <<свойство -- признак>> вычислен коэффициент Спирмена ($\rho$) с поправкой Бенджамини--Хохберга (FDR, $\alpha = 0.05$). Установлены наиболее сильные связи: pH --- GNDVI\textsubscript{L8, весна} ($\rho = -0.670$, $p = 1.6 \times 10^{-142}$), среднегодовые осадки MAP ($\rho = 0.659$), уклон ($\rho = 0.609$); K$_2$O --- BSI\textsubscript{S2, весна} ($\rho = -0.478$); P$_2$O$_5$ --- средняя температура вегетационного периода ($\rho = 0.476$). Глобальный индекс Морана подтвердил выраженную пространственную автокорреляцию для всех свойств ($I = 0.51$--$0.86$, $p < 10^{-15}$); выявлен значимый широтный градиент pH ($\rho_{\text{lat}} = -0.41$) и SOC ($\rho_{\text{lat}} = +0.29$). Иерархическая декомпозиция дисперсии показала, что между-полевой компонент составляет 22.6\% (S) -- 73.2\% (pH), определяя пределы внутриполевого предсказания. Конфаундинг-анализ установил, что pH опосредует 41.9\% наблюдаемой корреляции SOC--NDVI\textsubscript{лето}. Композитный признак GNDVI$\times$BSI\textsubscript{весна} обеспечивает прирост $\Delta|\rho| = +0.010$ для K$_2$O по сравнению с одиночным BSI.

Результаты формируют научную базу для проектирования признакового пространства предиктивных моделей ЦПК, что реализовано в сопутствующей работе (Часть~2).

\medskip
\noindent\textbf{Ключевые слова:} корреляционный анализ; дистанционное зондирование; спектральные индексы; пространственная автокорреляция; индекс Морана; декомпозиция дисперсии; конфаундинг; чернозёмы; Sentinel-2; Северный Казахстан


\newpage

% ============================================================
\section{Введение}
\label{sec:introduction}
% ============================================================

\subsection{Контекст и проблема}

Почвенные ресурсы --- фундаментальная основа сельскохозяйственного производства и продовольственной безопасности. Деградация почв затрагивает $\sim$33\% мировых земельных ресурсов, а связанные экономические потери оцениваются в 10.6~трлн~USD ежегодно~\citep{Montanarella2016}. Оперативный и пространственно-детальный мониторинг агрохимических свойств --- необходимое условие парадигмы точного земледелия, направленной на оптимизацию внесения удобрений, минимизацию экологического воздействия и повышение урожайности~\citep{Gebbers2010}. Между тем, традиционные методы опираются на отбор проб с последующим лабораторным анализом (15--50~USD за образец, 2--4~недели)~\citep{ViscarraRossel2006}, что делает прямое обследование непрактичным для регулярного мониторинга обширных территорий.

Проблема особенно актуальна для Северного Казахстана --- одного из крупнейших зернопроизводящих регионов мира ($>$20~млн~га пашни)~\citep{Swinnen2017}. Регион характеризуется резко континентальным климатом, коротким вегетационным периодом и высокой уязвимостью почв к ветровой эрозии~\citep{Kraemer2015}. Несмотря на стратегическое значение, число работ, где систематически анализировались бы взаимосвязи между физико-химическими свойствами почв и мультиисточниковыми данными ДЗЗ конкретно для данной зоны, крайне невелико по сравнению с хорошо изученными территориями Европы, Китая и Южной Америки.

\subsection{Физические основы дистанционного зондирования почв}

Спектральный отклик почвенной поверхности и растительного покрова содержит информацию о физико-химических свойствах подстилающих почв. Отражательная способность в VNIR (400--1000~нм) чувствительна к содержанию органического вещества, оксидов железа и минералогическому составу; SWIR-область (1000--2500~нм) отражает влажность и глинистые минералы~\citep{BenDor2009, ViscarraRossel2010}. Спутниковые системы Sentinel-2~\citep{Drusch2012} и Landsat-8~\citep{Roy2014} обеспечивают мультиспектральные данные с разрешением 10--30~м и временным повторением 5--16~дней. Радарные данные Sentinel-1~SAR~\citep{Torres2012}, работающие независимо от облачности, несут информацию о диэлектрических свойствах и влажности поверхности (прирост точности 8--15\% при интеграции с оптическими данными)~\citep{Bauer2019}. Цифровые модели рельефа SRTM определяют геоморфологические ковариаты (уклон, экспозиция, топографический индекс влажности), контролирующие перераспределение влаги и эрозионные потоки~\citep{Farr2007}. Глобальные карты SoilGrids~v2.0~\citep{Hengl2017} предоставляют фоновую педологическую информацию (250~м), а ERA5-Land~\citep{MunozSabater2021} --- климатические характеристики.

\subsection{Существующие исследования и разрыв знаний}

Castaldi et al.~\citep{Castaldi2019} показали, что Sentinel-2 на обнажённых почвах Бельгии обеспечивает $R^2$ до~0.70 для SOC. Vaudour et al.~\citep{Vaudour2019} продемонстрировали потенциал S2 для картирования pH, SOC и текстуры во Франции. Мета-анализ Wadoux et al.~\citep{Wadoux2020} установил медианные $R^2 = 0.50$--$0.65$ для SOC и $0.55$--$0.75$ для pH в задачах цифрового почвенного картирования. \v{Z}\'{i}\v{z}ala et al.~\citep{Zizala2022} показали ценность мультисезонных данных S2 для SOC в Чехии. Однако:

\begin{enumerate}[nosep]
  \item \textbf{Географический пробел}: систематический мультиисточниковый корреляционный анализ для степной зоны Центральной Азии (чернозёмно-каштановые почвы, резко континентальный климат) отсутствует;
  \item \textbf{Масштаб признакового пространства}: большинство работ анализируют ограниченный набор из 20--80 признаков; систематический скрининг $>$500 мультимодальных ковариат, включая композитные спектральные индексы, ранее не проводился;
  \item \textbf{Компоненты вариации}: роль между-полевой vs внутриполевой дисперсии, пространственной автокорреляции и конфаундинга между почвенными свойствами редко количественно оценивается в контексте ДЗЗ;
  \item \textbf{Целевой разрыв}: предсказание NO$_3$, P$_2$O$_5$, K$_2$O и S значительно менее изучено, чем SOC и pH.
\end{enumerate}

\subsection{Цели и задачи}

Настоящая работа является \textbf{первой из двух сопутствующих публикаций}, посвящённых цифровому почвенному картированию степной зоны Северного Казахстана. В Части~1 (данная статья) проводится комплексный количественный анализ взаимосвязей между шестью агрохимическими свойствами пахотного горизонта и мультимодальными данными ДЗЗ, формирующий научную и методологическую основу для последующего предиктивного моделирования. В Части~2 (сопутствующая статья) выявленные корреляционные зависимости, структура дисперсии и рекомендации по выбору оптимальных признаков используются для построения и систематического сравнения 12~моделей машинного обучения и 3~архитектур глубокого обучения с трёхуровневой пространственной валидацией.

Конкретные задачи Части~1:

\begin{enumerate}[nosep]
  \item Количественная оценка ранговых корреляций 6~почвенных свойств с 536~ковариатами ДЗЗ из 6~источников (Спирмен, BH~FDR);
  \item Сравнительный анализ 110~композитных спектральных признаков (попарные произведения, нормализованные разности, мультисезонные дельты) с одиночными индексами;
  \item Идентификация оптимальных фенологических окон для каждого свойства;
  \item Выявление пространственной структуры почвенных свойств (Moran~$I$, широтный градиент);
  \item Иерархическая декомпозиция дисперсии (между-полевая vs внутриполевая) и оценка ICC;
  \item Конфаундинг-анализ: оценка опосредующей роли pH в корреляциях <<SOC --- спектральные индексы>> методом частных корреляций;
  \item Конструирование производных почвенных индикаторов и оценка их корреляции с данными ДЗЗ.
\end{enumerate}

% ============================================================
\section{Район исследования и данные}
\label{sec:study_area}
% ============================================================

\subsection{Район исследования}

Исследование проведено на сельскохозяйственных угодьях трёх областей Северного Казахстана --- Костанайской, Акмолинской и Северо-Казахстанской (50°--54°~с.ш., 64°--72°~в.д.) (Рис.~\ref{fig:study_area}).

\begin{figure}[H]
  \centering
  \includegraphics[width=0.90\textwidth]{fig1_study_area.png}
  \caption{Карта района исследования. Северный Казахстан (Костанайская, Акмолинская и Северо-Казахстанская области). Точки обозначают расположение 186~опробованных полей (81~хозяйство, 1085~образцов, 2020--2023~гг.).}
  \label{fig:study_area}
\end{figure}

\textbf{Климат.} Резко континентальный (Dfb/BSk по Кёппену): среднегодовая температура $+0.5 \ldots +3.0$\,°C, амплитуда от $-40$ до $+35$\,°C. Осадки 250--400~мм/год (до 40\% в июне--августе). Вегетационный период 130--150~дней, сумма активных температур 2000--2600\,°C$\cdot$дней~\citep{Fick2017}.

\textbf{Почвы.} Чернозёмы обыкновенные и южные (Haplic Chernozems, WRB) на севере, каштановые почвы (Kastanozems) на юге. Содержание гумуса 2--6\% (закономерно убывает с севера на юг). Гранулометрический состав --- средний и тяжёлый суглинок. Длительное освоение (с 1954~г.) привело к дегумификации и повышению дефляционной уязвимости.

\textbf{Рельеф.} Слабоволнистая равнина, 180--420~м, уклоны 0--3°. Преобладание ветровой эрозии над водной.

\textbf{Агрохозяйственный контекст.} Яровая пшеница, ячмень, масличные. Доля no-till: $\sim$45\% (2023). Нормы удобрений: от нуля до 90~кг/га N + 40~кг/га P (д.в.) --- значительная пространственная гетерогенность на уровне хозяйств.

\subsection{Почвенные данные}
\label{sec:soil_data}

Набор данных включает \textbf{1085~образцов} из пахотного горизонта (0--25~см), собранных за четыре полевых сезона (2020--2023~гг.) на \textbf{186~полевых участках} (81~хозяйство). Пробоотбор проведён в послеуборочный период (сентябрь--октябрь) по ГОСТ~28168-89: смешанная проба из 15--20 точечных уколов на площади 20--50~га. Координаты центроидов --- GNSS (±3~м).

Лабораторные определения (аккредитованная лаборатория Республики Казахстан):
\begin{itemize}[nosep]
  \item \textbf{pH\textsubscript{KCl}} --- потенциометрия в 1М KCl (ГОСТ~26483-85);
  \item \textbf{SOC, \%} --- метод Тюрина (ГОСТ~26213-91), пересчёт по Ван~Беммелену~\citep{Pribyl2010}: SOC~$=$~HU$\times$0.58;
  \item \textbf{NO$_3$, мг/кг} --- ионометрическое определение (ГОСТ~26951-86);
  \item \textbf{P$_2$O$_5$ и K$_2$O, мг/кг} --- метод Мачигина, 1\% (NH$_4$)$_2$CO$_3$ (ГОСТ~26205-91);
  \item \textbf{S, мг/кг} --- турбидиметрия (KCl-вытяжка).
\end{itemize}

Полный мастер-датасет (включая все годы) содержит 1215~строк; после удаления записей с пропусками в целевых свойствах (94~записи, 7.7\%, исключительно SOC/NO$_3$/P$_2$O$_5$/K$_2$O/S; pH не имеет пропусков) остаётся \textbf{1085~полных записей}, использованных в анализе.

% ============================================================
\section{Методы}
\label{sec:methods}
% ============================================================

Общий методологический пайплайн представлен на Рис.~\ref{fig:workflow2}.

\begin{figure}[H]
\centering
\resizebox{\textwidth}{!}{%
\begin{tikzpicture}[
  block/.style={rectangle, draw, fill=blue!8, text width=2.8cm, minimum height=0.9cm, align=center, font=\small},
  data/.style={trapezium, trapezium left angle=70, trapezium right angle=110, draw, fill=green!8, text width=2.4cm, minimum height=0.8cm, align=center, font=\small},
  model/.style={rectangle, draw, fill=yellow!12, text width=2.8cm, minimum height=0.9cm, align=center, font=\small},
  result/.style={rectangle, rounded corners, draw, fill=orange!12, text width=2.8cm, minimum height=0.9cm, align=center, font=\small},
  arr/.style={-{Stealth[length=3mm]}, thick},
]
% Row 1
\node[data] (part1) {530~features\\(corr. analysis)};
\node[block, right=0.8cm of part1] (sel) {Feature selection\\(top-15, MDI)};
\node[data, right=0.8cm of sel] (patches) {Satellite\\patches (16/32/64)};
\node[block, right=0.8cm of patches] (cv) {Spatial\\CV (3 strategies)};
% Row 2
\node[model, below=1.2cm of part1] (ml) {11 ML models\\(tabular)};
\node[model, below=1.2cm of sel] (cnn) {ResNet-18\\(patches)};
\node[model, below=1.2cm of patches] (cnx) {ConvNeXt+SE\\(54 channels)};
\node[block, below=1.2cm of cv] (ablat) {Ablation study\\patch/indices};
% Row 3
\node[result, below=1.2cm of ml, xshift=2cm] (rank) {Model ranking\\+ leakage audit};
\node[result, below=1.2cm of cnx, xshift=1cm] (maps) {Prediction\\maps};
% Arrows
\draw[arr] (part1) -- (sel);
\draw[arr] (sel) -- (ml);
\draw[arr] (patches) -- (cnn);
\draw[arr] (patches) -- (cnx);
\draw[arr] (cv) -- (ablat);
\draw[arr] (cv.south) -- ++(0,-0.5) -| (ml.north);
\draw[arr] (cv.south) -- ++(0,-0.5) -| (cnn.north);
\draw[arr] (ml) -- (rank);
\draw[arr] (cnn) -- (rank);
\draw[arr] (cnx) -- (maps);
\draw[arr] (ablat) -- (maps);
\end{tikzpicture}%
}
\caption{Schematic diagram of the predictive modelling pipeline. Input: a master dataset of 530~features (derived via GEE) and satellite patches from GEE. Output: model ranking and prediction maps.}
\label{fig:workflow2}
\end{figure}

\subsection{Источники данных ДЗЗ и извлечение признаков}
\label{sec:rs_features}

Данные извлечены из шести источников через Google Earth Engine~\citep{Gorelick2017} (Таблица~\ref{tab:data_sources}). Ниже представлены ключевые параметры каждого источника.

\begin{table}[H]
\centering
\caption{Satellite and ancillary data sources used in the study}
\label{tab:data_sources}
\small\resizebox{\textwidth}{!}{%
\begin{tabular}{lp{5.5cm}ccc}
\toprule
\thead{Source} & \thead{Parameters extracted} & \thead{Resolution} & \thead{Period} & \thead{\# Features}\\
\midrule
Sentinel-2~\citep{Drusch2012}
  & 12~spectral bands + 5~vegetation / soil indices\newline
    (NDVI, RECI, BSI, NDSI, NDWI), 4~seasons
  & 10--60~m & 2022--2023 & 164 \\
\addlinespace[2pt]
Landsat-8~\citep{Roy2014, Sorenson2021}
  & 6~reflectance bands + 3~indices $\times$~4~seasons
  & 30~m & 2022--2023 & 120 \\
\addlinespace[2pt]
Sentinel-1~\citep{Torres2012}
  & VV, VH bands $\times$~4~seasons + temporal statistics
  & 10~m & 2022--2023 & 16 \\
\addlinespace[2pt]
SRTM DEM~\citep{Farr2007}
  & Elevation, slope, aspect, TPI~\citep{Riihimaki2021}
  & 30~m & Static & 5 \\
\addlinespace[2pt]
SoilGrids~v2.0~\citep{Hengl2017, Poggio2021}
  & 7~soil variables $\times$~6~depth layers\newline
    (\textit{excluded from final model; see text})
  & 250~m & Static & \textit{42}$^{*}$ \\
\addlinespace[2pt]
ERA5-Land~\citep{MunozSabater2021}
  & Mean annual temperature/precipitation,\newline
    growing-season temperature/precipitation, GDD
  & $\sim$9~km & 2022--2023 & 5 \\
\bottomrule
\multicolumn{5}{l}{\footnotesize $^{*}$~SoilGrids features excluded from the final feature pool to prevent target leakage (see Section~\ref{sec:limitations}).}
\end{tabular}}
\end{table}

12-стадийный пайплайн (s01--s12) извлекает признаки из всех шести источников, однако 42~признака SoilGrids (s06) были \textbf{исключены} из итогового мастер-датасета для предотвращения утечки целевой переменной: SoilGrids~v2.0 обучен на полевых данных из пересекающегося пространственного домена~\citep{Poggio2021}, а признаки включают pH и SOC, непосредственно коррелированные с целевыми переменными.
Итоговый мастер-датасет содержит \textbf{530~признаков}, включая: сезонные медианы каналов и индексов ($\sim$164), временны\'{e} характеристики (дельты, амплитуды, стандартные отклонения; $\sim$120), SAR-признаки (16), топографические (5), климатические (5), GLCM-текстуры~\citep{Haralick1973} (40), инженерные отношения каналов и PCA ($\sim$30), межплатформенные композиты ($\sim$14).
Спектральные признаки извлекаются из мультиспектральных композитов~--- аналогичный подход применялся для лабораторной спектроскопии почв~\citep{Steffens2013}.

\subsubsection{Двухэтапный отбор признаков}

Для каждого целевого свойства:
\begin{enumerate}[nosep]
  \item \textbf{Фильтрация}: удаление quasi-constant признаков ($\sigma^2 < 10^{-6}$) и одного из каждой пары с $|r_{\text{Pearson}}| > 0.95$;
  \item \textbf{Ранжирование и отбор}: ранжирование оставшихся $\sim$250--300~признаков по Mean Decrease Impurity (MDI) из Random Forest ($n\_est = 300$)~\citep{Hengl2018}; отбор \textbf{15~наиболее значимых}.
  Выбор $K = 15$ обоснован выходом кривой <<число признаков~--- качество>> на плато при 12--18 (Рис.~\ref{fig:feature_curve} в Приложении); сравнение подходов отбора признаков для пространственных ML-задач см.~\citep{Meyer2019}.
\end{enumerate}

\textbf{Примечание:} MDI-ранжирование выполнено внутри обучающего набора каждого LOFO-фолда для избежания утечки информации через отбор признаков.
Для основного сравнения 11~моделей использован фиксированный набор из 15~признаков, определённый по всему датасету, для обеспечения сопоставимости.
Дополнительно проведён контрольный эксперимент с полностью честным per-fold MDI-отбором в рамках Farm-LOFO-CV (раздел~\ref{sec:perfold_features}): на каждом из 20~фолдов (по хозяйствам) независимо отбирается 15~признаков из обучающей выборки с последующей оптимизацией гиперпараметров через GridSearchCV (nested GroupKFold по полям).
Стабильность отбора между фолдами (средний попарный IoU от $0.45$ до $0.71$ в зависимости от свойства) ниже, чем при Field-LOFO, что отражает бо\'{л}ьшую гетерогенность между хозяйствами.

\subsection{Подготовка спутниковых патчей для CNN}
\label{sec:patches}

Для end-to-end DL-моделей извлечены 2D-патчи трёх размеров:
\begin{itemize}[nosep]
  \item 16$\times$16~пикселей (160$\times$160~м);
  \item 32$\times$32~пикселей (320$\times$320~м);
  \item 64$\times$64~пикселей (640$\times$640~м).
\end{itemize}

Конфигурации каналов: (i)~\textbf{базовая} (13~каналов: 12~S2 + DEM); (ii)~\textbf{расширенная} (18~каналов: +5~индексов: NDVI, BSI, NDSI, NDWI, RECI); (iii)~\textbf{мультисезонная} (54~канала: оптика $\times$~3~сезона + индексы $\times$~3 + VV, VH + DEM).
Общее количество патчей~--- 1071 на каждый размер (14~образцов исключены из-за расположения вблизи границ сцен Sentinel-2, где невозможно извлечь полный патч без пересечения с nodata-пикселями).
Формат NumPy~(.npy), $[C, H, W]$.

\subsection{Стратегия кросс-валидации}
\label{sec:cv_strategy}

Пространственная автокорреляция почвенных свойств (Moran's $I = 0.51$--$0.86$, оценённая на полном наборе данных) требует строгой стратегии валидации~\citep{Roberts2017, Wadoux2021, Meyer2021}.
В настоящей работе последовательно применены три стратегии, образующие \textbf{иерархию строгости} оценки.
Во всех трёх стратегиях используется схема скользящего окна \textbf{LOFO-CV} (Leave-One-\textit{X}-Out Cross-Validation), где \textit{X} определяет единицу группировки:
\begin{itemize}[nosep]
  \item \textbf{Field-LOFO} (Leave-One-\textbf{Field}-Out) --- каждый фолд = одно поле (81~итерация). Минимальный пространственный разрыв: соседние поля из \textit{того же} хозяйства остаются в обучающей выборке;
  \item \textbf{Farm-LOFO} (Leave-One-\textbf{Farm}-Out) --- каждый фолд = все поля одного хозяйства (20~итераций). Гарантирует полное отсутствие всех образцов тестируемого хозяйства в обучающей выборке, что соответствует сценарию работы на \textit{новом хозяйстве}, для которого почвенных данных нет.
\end{itemize}
\subsubsection{Field-LOFO-CV (81~фолд)}
\label{sec:field_lofo_method}

На каждой из 81~итерации одно поле ($\sim$13.4~образца в среднем) полностью выделяется в тестовую выборку; модель обучается на оставшихся 80~участках.
Финальные метрики рассчитываются по агрегированным out-of-fold (OOF) предсказаниям всех 1085~образцов.
Field-LOFO-CV является \textbf{основной стратегией} для ранжирования моделей.

\subsubsection{Farm-LOFO-CV (20~хозяйств)}
\label{sec:farm_lofo}

Датасет охватывает \textbf{20~хозяйств} (от 8 до 151~образца; медиана~--- 38).
При Farm-LOFO-CV каждое хозяйство поочерёдно выключается из обучения ($\sim$54~образца в среднем).Схема значительно строже Field-LOFO, поскольку пространственный разрыв между train и test существенно больше.
Farm-LOFO-CV проведена для \textbf{всех 11~ML-моделей} на тех же MDI-отобранных признаках (15~для пяти свойств; для S --- 6~весенних + статических признаков, см.~разд.~\ref{sec:leakage_audit}), что позволяет корректно изолировать эффект строгости валидации от влияния алгоритма обучения.

\subsubsection{Оптимизированный пространственный split (65/6/10~хозяйств)}
\label{sec:spatial_split}

Для вычислительно ёмких моделей (ablation study для CNN, ConvNeXt), требующих длительного обучения и раннего останова, а также для настройки гиперпараметров XGBoost и CatBoost, использовано фиксированное разбиение на уровне хозяйств.
Оптимальное соотношение определено систематическим поиском: протестировано 56~комбинаций ($n_{\text{test}} \in \{4, \ldots, 14\}$, $n_{\text{val}} \in \{4, \ldots, 15\}$) при $n_{\text{train}} \geq 20$~ферм.
Для каждой комбинации выполнено 15~случайных разбиений, каждое оценено XGBoost на всех шести целевых переменных (всего~5040 обучений).

\begin{table}[H]
\centering
\caption{Top-5 split configurations ranked by mean Spearman~$\rho$}
\label{tab:split_search}
\small
\begin{tabular}{ccccccc}
\toprule
\thead{$n_\text{test}$} & \thead{$n_\text{val}$} & \thead{$n_\text{train}$} & \thead{$\bar{\rho}$} & \thead{$\sigma(\rho)$} & \thead{$\bar{R}^2$} & \thead{$\overline{\text{RMSE}}$} \\
\midrule
\textbf{10} & \textbf{6} & \textbf{65} & \textbf{0.598} & \textbf{0.198} & 0.307 & 26.58 \\
8  & 5  & 68 & 0.597 & 0.249 & 0.335 & 26.23 \\
14 & 6  & 61 & 0.594 & 0.191 & 0.360 & 25.52 \\
8  & 4  & 69 & 0.592 & 0.253 & 0.307 & 25.72 \\
10 & 10 & 61 & 0.592 & 0.216 & 0.388 & 25.54 \\
\bottomrule
\end{tabular}
\end{table}

По результатам поиска выбрано разбиение \textbf{65/6/10} (80\%/7\%/12\% хозяйств), обеспечивающее наивысший $\bar{\rho} = 0.598$ при наименьшей дисперсии ($\sigma = 0.198$).
При $n_{\text{test}} = 4$ наблюдается максимальный разброс $\sigma(\rho) = 0.29$--$0.38$; при $n_{\text{test}} \geq 8$ стандартное отклонение стабилизируется ($\sigma \leq 0.25$).

\subsubsection{Нормализация}

StandardScaler (z-score, статистики \emph{только} по train-подмножеству каждого фолда) применялся для DL, SVR, KNN и линейных моделей.
Для древесных ансамблей масштабирование не выполнялось ввиду инвариантности к монотонным преобразованиям.

\subsection{Модели машинного обучения}
\label{sec:ml_models}

Сравнение проведено для 11~классических ML-алгоритмов (Таблица~\ref{tab:ml_models}).
Все модели обучены на 15~отобранных признаках с LOFO-CV (81~фолд).

\begin{table}[H]
\centering
\caption{ML model configurations}
\label{tab:ml_models}
\small
\begin{tabular}{llp{7cm}}
\toprule
\thead{Model} & \thead{Library} & \thead{Key hyperparameters}\\
\midrule
RF~\citep{Breiman2001, Hengl2018}         & scikit-learn~\citep{Pedregosa2011} & $n\_est = 500$, $max\_feat = \sqrt{p}$, $min\_leaf = 3$ \\
ET~\citep{Geurts2006}          & scikit-learn & $n\_est = 500$, $max\_feat = \sqrt{p}$, $min\_leaf = 3$, рандомизированные пороги \\
XGBoost~\citep{Chen2016}       & xgboost      & $n\_est = 300$, $max\_depth = 6$, $\eta = 0.1$, $\lambda = 1$, $\text{subsample} = 0.8$ \\
CatBoost~\citep{Dorogush2018}  & catboost     & $iter = 500$, $depth = 6$, $lr = 0.05$, ordered bootstrap \\
GBDT~\citep{Friedman2001}      & scikit-learn & $n\_est = 300$, $max\_depth = 5$, $lr = 0.1$, $\text{subsample} = 0.8$ \\
CART                            & scikit-learn & $max\_depth = 10$, $min\_leaf = 5$ \\
KNN                             & scikit-learn & $K = 7$, взвешенное голосование (distance) \\
LR                              & scikit-learn & Обычный МНК, без регуляризации \\
Ridge                           & scikit-learn & $\alpha = 1.0$ (L2-регуляризация) \\
SGD                             & scikit-learn & $\alpha = 10^{-4}$, $penalty = \text{l2}$, $max\_iter = 1000$ \\
SVR~\citep{Smola2004}        & scikit-learn & RBF-ядро, $C = 1.0$, $\epsilon = 0.1$ \\
\bottomrule
\end{tabular}
\end{table}

\textbf{Настройка гиперпараметров.}
Для классических ML-моделей (RF, XGBoost, GBDT, ET, KNN, SVR) применялись два подхода:
\begin{itemize}[nosep]
  \item \textbf{GridSearchCV} с nested 5-fold CV внутри обучающего набора каждого LOFO-фолда. Сетки компактны ($|\mathcal{G}| = 6$--$18$~комбинаций на модель; для RF: $n\_est \in \{300, 500, 800\}$, $max\_feat \in \{\sqrt{p}, \log_2 p\}$, $min\_leaf \in \{2, 3, 5\}$);
  \item \textbf{Optuna}~\citep{Akiba2019} (TPE-сэмплер, $n\_trials = 100$, MedianPruner) --- байесовский поиск с отсечением нецелесообразных испытаний. Использован для XGBoost и CatBoost при оптимизированном split-сценарии (65/6/10~хозяйств), где обширное пространство поиска делает исчерпывающий Grid-поиск вычислительно нецелесообразным.
\end{itemize}

\subsection{Архитектуры глубокого обучения}
\label{sec:dl_models}

Все DL-модели реализованы на PyTorch~\citep{Paszke2019}.
Гиперпараметры DL-моделей (learning rate, weight decay, patience) оптимизированы с помощью \textbf{Optuna}~\citep{Akiba2019} (TPE-сэмплер, $n\_trials = 100$, MedianPruner) на валидационном наборе пространственного split (65/6/10~хозяйств).

\subsubsection{1D CNN на табличных признаках (SoilCNN1D)}

Одномерная свёрточная сеть для 15~табличных признаков:
\begin{align*}
&\text{Input}(1,\,15) \to \text{Conv1d}(1{\to}16,\,k{=}3) \to \text{BN} \to \text{ReLU} \to \text{MaxPool}(2) \\
&\to \text{Conv1d}(16{\to}32,\,k{=}3) \to \text{BN} \to \text{ReLU} \to \text{MaxPool}(2) \\
&\to \text{Dropout}(0.3) \to \text{FC}(96{\to}32{\to}1)
\end{align*}
Обучение: 300~эпох, Adam ($\text{lr} = 10^{-3}$, $\text{wd} = 10^{-4}$), ранний останов (patience~=~30).

\subsubsection{ResNet-18 на спутниковых патчах}

Адаптированная ResNet-18~\citep{He2016}: стандартный первый слой $\text{Conv2d}(3{\to}64)$ заменён на $\text{Conv2d}(18{\to}64,\,k{=}7)$ для 18-канальных мультиспектральных патчей; обучение \emph{с нуля} (без ImageNet-предобучения~--- спектральная несовместимость с RGB-весами~\citep{Wang2023SSL4EO}).
Выходной блок: $\text{Dropout}(0.3) \to \text{FC}(512{\to}128) \to \text{ReLU} \to \text{Dropout}(0.15) \to \text{FC}(128{\to}1)$.
Аугментация: случайные отражения и повороты (0°/90°/180°/270°).
Обучение: 60~эпох, AdamW ($\text{lr}{=}10^{-4}$, $\text{wd}{=}10^{-5}$), ReduceLROnPlateau (factor~=~0.5, patience~=~3).

\subsubsection{MultiSpectralConvNeXt на мультисезонных композитах}
\label{sec:convnext_arch}

Архитектура MultiSpectralConvNeXt с блоками Squeeze-and-Excitation (SE)~\citep{Hu2018}.
Вход~--- 54~канала: S2-оптика $\times$3~сезона (36), спектральные индексы $\times$3 (15), SAR VV/VH (2), DEM (1).

\textbf{Ключевые модули (простыми словами):}
\begin{itemize}[nosep]
  \item \textbf{SE-блок (Squeeze-and-Excitation)}: механизм «внимания», который позволяет нейросети самой понимать, какой спектральный канал (например, инфракрасный или радарный) важнее в данный момент, и усиливать его сигнал, подавляя менее значимые;
  \item \textbf{ConvNeXt-блок}: современная архитектура сверточных слоев, которая эффективно извлекает пространственные паттерны (форму пятен, градиенты) из спутниковых снимков, работая быстрее и точнее классических сетей.
\end{itemize}

\textbf{Полная архитектура:} Stem $\text{Conv2d}({\to}128, 1{\times}1)$~$\to$ Stage~1: Block(128)$\times$2, AvgPool~$\to$ Stage~2: Block(256)$\times$2, AvgPool~$\to$ Stage~3: Block(512)$\times$1~$\to$ Head: GAP~$\to$~Dropout(0.4)~$\to$~FC$(512{\to}128)$~$\to$~FC$(128{\to}1)$.

Обучение: до 300~эпох, AdamW ($\text{lr}{=}5{\times}10^{-4}$), HuberLoss, ранний останов (patience~=~20), AMP (mixed precision). Аугментация: аналогично ResNet-18.

\subsection{Метрики оценки и статистический анализ}
\label{sec:metrics}

\begin{itemize}[nosep]
  \item \textbf{Spearman~$\rho$} (основная)~--- ранговая корреляция, устойчивая к ненормальности и монотонным нелинейностям;
  \item \textbf{$R^2$}~--- коэффициент детерминации:
  \begin{equation}
    R^2 = 1 - \frac{\sum_{i=1}^{n}(y_i - \hat{y}_i)^2}{\sum_{i=1}^{n}(y_i - \bar{y})^2};
  \end{equation}
  \item \textbf{RMSE}~--- среднеквадратическая ошибка:
  \begin{equation}
    \text{RMSE} = \sqrt{\frac{1}{n}\sum_{i=1}^{n}(y_i - \hat{y}_i)^2};
  \end{equation}
  \item \textbf{MAE}~--- средняя абсолютная ошибка:
  \begin{equation}
    \text{MAE} = \frac{1}{n}\sum_{i=1}^{n}|y_i - \hat{y}_i|;
  \end{equation}
  \item \textbf{RPD} (Ratio of Performance to Deviation)~--- отношение стандартного отклонения наблюдений к RMSE~\citep{Chang2001}:
  \begin{equation}
    \text{RPD} = \frac{\text{SD}_{\text{obs}}}{\text{RMSE}};
  \end{equation}
  RPD~$> 2.0$~--- хорошая предсказательная способность, $1.4$--$2.0$~--- приемлемая, $< 1.4$~--- модель непригодна для количественных предсказаний;
  \item \textbf{CCC} (Lin's Concordance Correlation Coefficient)~--- одновременно оценивает точность и правильность (agreement) предсказаний~\citep{Lin1989}:
  \begin{equation}
    \rho_c = \frac{2\,\rho_{xy}\,s_x\,s_y}{s_x^2 + s_y^2 + (\bar{x} - \bar{y})^2},
  \end{equation}
  где $\rho_{xy}$~--- коэффициент Пирсона, $s_x, s_y$~--- стандартные отклонения, $\bar{x}, \bar{y}$~--- средние наблюдений и предсказаний.
  CCC~$> 0.90$~--- отличное, $0.65$--$0.90$~--- умеренное согласие~\citep{McBride2005}.
\end{itemize}

Для оценки доли между-полевой дисперсии рассчитывался коэффициент внутриклассовой корреляции (ICC, Intraclass Correlation Coefficient) по модели случайных эффектов (one-way random effects, ICC(1,1)) с использованием библиотеки \texttt{pingouin}.
Для оценки стабильности метрик при Farm-LOFO-CV (20~фолдов) рассчитывались стандартные ошибки среднего (SEM).

\noindent\textbf{Примечание}: для свойств с сильно скошенным распределением (S: асимметрия~3.54; P$_2$O$_5$: 2.54) $R^2$ может быть обманчиво высоким при предсказании околосреднего уровня, тогда как $\rho$ остаётся низким; поэтому $\rho$ является основной метрикой ранжирования. Выбор Spearman~$\rho$ в качестве основной метрики обусловлен тремя факторами: (1)~ненормальностью всех шести свойств ($p < 0.001$, Шапиро--Уилка), (2)~устойчивостью к монотонным нелинейностям (типичным для древесных ансамблей), и (3)~нечувствительностью к выбросам, искажающим $R^2$ для свойств со скошенным распределением~\citep{Wadoux2021}.

\subsection{Вычислительная среда}
\label{sec:compute_env}

Python~3.10; PyTorch~2.1 (CUDA~12.1); scikit-learn~1.3; xgboost~2.0; catboost~1.2.
Обучение ML-моделей: Intel Core i9-12900K, 64~GB~RAM.
Обучение DL-моделей: NVIDIA RTX~3090 (24~GB VRAM); среднее время обучения одной модели ConvNeXt~--- $\sim$12~мин (при mixed precision).
Все расчёты воспроизводимы ($\text{SEED} = 42$; фиксация для PyTorch, NumPy, Python random).

% ============================================================
\section{Результаты}
\label{sec:results}
% ============================================================

\subsection{Корреляционный скрининг ковариат}
\label{sec:correlation_screening}

Корреляционный скрининг 530~мультимодальных ковариат ДЗЗ с шестью агрохимическими свойствами показал, что максимальная ранговая корреляция $|\rho|_{\max}$ существенно варьирует между свойствами: pH~(0.670) $\gg$ K$_2$O~(0.478) $\approx$ P$_2$O$_5$~(0.476) $>$ SOC~(0.350) $>$ NO$_3$~(0.290) $>$ S~(0.280).
Пространственный анализ выявил выраженную автокорреляцию (Moran~$I = 0.51$--$0.86$), широтный градиент и значительные различия в долях между-полевой дисперсии (от 22.6\% для~S до 73.2\% для~pH).
Обнаружен конфаундинг pH~--- 41.9\% наблюдаемой корреляции SOC--NDVI опосредовано почвенной кислотностью.
Эти закономерности формируют иерархию предсказуемости, полностью подтверждённую модельными экспериментами ниже.

\subsection{Сравнение ML-моделей: Field-LOFO-CV и Farm-LOFO-CV}
\label{sec:ml_results}

\subsubsection{Field-LOFO-CV (81~фолд)}
\label{sec:field_lofo}

Результаты 11~ML-моделей на 15~табличных признаках с Field-LOFO-CV (81~фолд) представлены в Таблице~\ref{tab:all_models_rho}; ResNet-18 включён для сопоставления ML и DL парадигм.
\textbf{Важно:} во всех сводных таблицах (Таблицы~\ref{tab:all_models_rho}, \ref{tab:top4_full}, \ref{tab:farm_lofo_all_rho}, \ref{tab:farm_top4_full}, \ref{tab:tuned_ml}) для серы (S) представлены метрики \textbf{только после исправления темпоральной утечки} (на основе 6~весенних и статических признаков, см. раздел~\ref{sec:leakage_audit}), чтобы избежать публикации артефактно завышенных результатов.

\begin{table}[H]
\centering
\caption{Spearman~$\rho$ from Field-LOFO-CV for ML models ($n = 1085$, 81~folds)}
\label{tab:all_models_rho}
\small
\begin{tabular}{lcccccc}
\toprule
\thead{Model} & \thead{pH} & \thead{SOC} & \thead{NO$_3$} & \thead{P$_2$O$_5$} & \thead{K$_2$O} & \thead{S}\\
\midrule
\textbf{GBDT}    & \textbf{0.857} & 0.657 & 0.707 & 0.563          & 0.616          & 0.404 \\
XGBoost           & 0.846          & 0.543 & 0.694 & 0.575          & 0.560          & 0.431 \\
CatBoost          & 0.826          & 0.646 & 0.736 & \textbf{0.600} & 0.482          & 0.459 \\
\textbf{RF}       & 0.798          & 0.731 & \textbf{0.775} & 0.595 & \textbf{0.624} & 0.467 \\
\textbf{ET}       & 0.771          & \textbf{0.735} & 0.768 & 0.611 & 0.615          & \textbf{0.484} \\
KNN               & 0.642          & 0.689 & 0.753 & 0.555          & 0.517          & 0.409 \\
SVR               & 0.714          & 0.658 & 0.691 & 0.600          & 0.435          & 0.433 \\
LR                & 0.747          & 0.616 & 0.382 & 0.496          & 0.471          & 0.302 \\
Ridge             & 0.747          & 0.617 & 0.383 & 0.497          & 0.470          & 0.303 \\
SGD               & 0.747          & 0.627 & 0.394 & 0.509          & 0.468          & 0.304 \\
CART              & 0.537          & 0.234 & 0.573 & 0.481          & 0.516          & 0.453 \\
\bottomrule
\end{tabular}
\end{table}

\textit{Примечание:} ResNet-18 обучен на 18-канальных патчах 64$\times$64, а не на 15~табличных признаках, поэтому его результаты представлены отдельно в разделе~\ref{sec:resnet_results} для корректного межпарадигмального сопоставления (разные входы, разные стратегии обучения).

Ансамблевые модели стабильно превосходят линейные и одиночные деревья по всем шести свойствам.
Лучший результат для pH~--- GBDT ($\rho = 0.857$, $R^2 = 0.841$), для SOC~--- ET ($\rho = 0.735$, $R^2 = 0.504$), для NO$_3$~--- RF ($\rho = 0.775$, $R^2 = 0.598$).
CART показывает наихудшие результаты (SOC: $\rho = 0.234$), подтверждая критическую роль ансамблирования.
Линейные модели (LR, Ridge, SGD) демонстрируют сопоставимый с ансамблями результат только для pH ($\rho \approx 0.747$), где зависимость наиболее монотонна, но существенно проигрывают для NO$_3$ ($\rho < 0.40$ vs $> 0.77$ у RF).

\begin{table}[H]
\centering
\caption{Full metrics for top-4 models (Field-LOFO-CV, 81~folds)}
\label{tab:top4_full}
\small\resizebox{\textwidth}{!}{%
\begin{tabular}{l cccc cccc cccc cccc}
\toprule
\multirow{2}{*}{\thead{Property}}
  & \multicolumn{4}{c}{\textbf{GBDT}}
  & \multicolumn{4}{c}{\textbf{RF}}
  & \multicolumn{4}{c}{\textbf{ET}}
  & \multicolumn{4}{c}{\textbf{CatBoost}} \\
\cmidrule(lr){2-5}\cmidrule(lr){6-9}\cmidrule(lr){10-13}\cmidrule(lr){14-17}
  & $\rho$ & RMSE & $R^2$ & RPD
  & $\rho$ & RMSE & $R^2$ & RPD
  & $\rho$ & RMSE & $R^2$ & RPD
  & $\rho$ & RMSE & $R^2$ & RPD \\
\midrule
pH          & 0.857 & 0.261 & 0.841 & 2.51 & 0.798 & 0.298 & 0.794 & 2.20 & 0.771 & 0.323 & 0.758 & 2.03 & 0.826 & 0.289 & 0.806 & 2.27 \\
SOC         & 0.657 & 0.442 & 0.349 & 1.24 & 0.731 & 0.385 & 0.506 & 1.42 & 0.735 & 0.386 & 0.504 & 1.42 & 0.646 & 0.407 & 0.448 & 1.35 \\
NO$_3$      & 0.707 & 6.07  & 0.409 & 1.30 & 0.775 & 5.00  & 0.598 & 1.58 & 0.768 & 4.83  & 0.625 & 1.64 & 0.736 & 5.68  & 0.482 & 1.39 \\
P$_2$O$_5$  & 0.563 & 16.30 & 0.356 & 1.25 & 0.595 & 15.36 & 0.428 & 1.32 & 0.611 & 15.38 & 0.426 & 1.32 & 0.600 & 16.77 & 0.318 & 1.21 \\
K$_2$O      & 0.616 & 121.9 & 0.464 & 1.37 & 0.624 & 121.4 & 0.469 & 1.37 & 0.615 & 125.7 & 0.430 & 1.33 & 0.482 & 138.0 & 0.314 & 1.21 \\
S           & 0.404 & 3.82  & 0.748 & 1.99 & 0.467 & 3.72  & 0.761 & 2.05 & 0.484 & 3.85  & 0.745 & 1.98 & 0.459 & 3.99  & 0.726 & 1.91 \\
\bottomrule
\end{tabular}}
\end{table}

\begin{figure}[H]
  \centering
  \includegraphics[width=\textwidth]{fig3_model_comparison.png}
  \caption{Comparison of Spearman~$\rho$ across all 11~ML models and ResNet-18 for six agrochemical properties (Field-LOFO-CV). Colour indicates algorithm class; asterisk denotes the best result for each property.}
  \label{fig:model_comparison}
\end{figure}

\begin{figure}[H]
  \centering
  \includegraphics[width=\textwidth]{fig2_scatter_pred_vs_obs.png}
  \caption{Predicted vs.~observed plots for the best model per property (Field-LOFO-CV). Dashed line indicates perfect prediction; point colour indicates prediction density.}
  \label{fig:scatter_pred_obs}
\end{figure}

\subsubsection{Контрольный эксперимент: per-fold MDI-отбор с GridSearchCV (Farm-LOFO)}
\label{sec:perfold_features}

Для оценки влияния фиксированного набора признаков и утечки при отборе на метрики проведён контрольный эксперимент с полностью честным per-fold MDI-отбором в рамках Farm-LOFO-CV (20~хозяйств): на каждом фолде из обучающей выборки (1)~обучается вспомогательный RF для MDI-ранжирования, (2)~отбираются 15~лучших признаков, (3)~выполняется GridSearchCV (nested GroupKFold по полям; сетка: $n\_est \in \{300, 500, 800\}$, $max\_feat \in \{\sqrt{p}, \log_2 p\}$, $min\_leaf \in \{2, 3, 5\}$; 18~комбинаций).
Результаты представлены в Таблице~\ref{tab:perfold_rf}.

\begin{table}[H]
\centering
\caption{RF with per-fold MDI selection + GridSearchCV vs.\ fixed features (Farm-LOFO-CV, 20~folds)}
\label{tab:perfold_rf}
\small\resizebox{\textwidth}{!}{%
\begin{tabular}{l cccccc cc cc}
\toprule
\multirow{2}{*}{\thead{Property}}
  & \multicolumn{6}{c}{\textbf{Per-fold MDI + GridSearchCV}}
  & \multicolumn{2}{c}{\textbf{Fixed features}}
  & \multicolumn{2}{c}{\textbf{Feature stability}} \\
\cmidrule(lr){2-7}\cmidrule(lr){8-9}\cmidrule(lr){10-11}
  & $\rho$ & $R^2$ & RMSE & MAE & RPD & CCC
  & $\rho$ & $R^2$
  & IoU$_{\text{folds}}$ & IoU$_{\text{ref}}$ \\
\midrule
pH          & 0.403 & 0.280 & 0.557 & 0.435 & 1.18 & 0.545  & 0.750 & 0.616 & 0.63 & 0.20 \\
SOC         & 0.169 & $-0.091$ & 0.573 & 0.403 & 0.96 & 0.098  & 0.529 & 0.283 & 0.54 & 0.19 \\
NO$_3$      & $-0.058$ & $-0.010$ & 7.933 & 5.448 & 1.00 & 0.217  & 0.232 & 0.279 & 0.58 & 0.15 \\
P$_2$O$_5$  & 0.288 & 0.117 & 19.08 & 12.94 & 1.07 & 0.275  & 0.490 & 0.343 & 0.57 & 0.18 \\
K$_2$O      & 0.246 & $-0.100$ & 174.7 & 135.4 & 0.95 & 0.197  & 0.448 & 0.234 & 0.45 & 0.20 \\
S           & 0.158 & 0.638 & 4.579 & 3.026 & 1.66 & 0.762  & 0.240 & 0.698 & 0.71 & 0.16 \\
\bottomrule
\end{tabular}}
\end{table}

Per-fold MDI-отбор в рамках Farm-LOFO-CV показывает \emph{существенное} снижение метрик по сравнению с фиксированным набором признаков для всех целевых свойств: pH ($\rho\!: 0.403$ vs $0.750$, $\Delta = -46\%$), SOC ($\rho\!: 0.169$ vs $0.529$, $\Delta = -68\%$), NO\textsubscript{3} ($\rho$ меняет знак: $-0.058$ vs $0.232$), P\textsubscript{2}O\textsubscript{5} ($\Delta = -41\%$), K\textsubscript{2}O ($\Delta = -45\%$).
Единственным исключением является S ($\rho\!: 0.158$ vs $0.240$; $R^2\!: 0.638$ vs $0.698$), сохраняющий приемлемое $R^2$ благодаря высокой стабильности отобранных признаков (IoU$_{\text{folds}} = 0.71$, максимальный среди всех свойств).

Данный результат указывает на два источника оптимизма в основных метриках: \emph{(i)} фиксированный набор признаков, отобранный на всём датасете, содержит неявную информационную утечку, и \emph{(ii)} GridSearchCV с nested cross-validation на 20~крупных фермерских фолдах не обеспечивает стабильного подбора гиперпараметров (наиболее часто: $\sqrt{p}$ в 55--95\% фолдов, $n\_est$ и $min\_leaf$ варьируются между свойствами).
Низкие значения IoU$_{\text{ref}}$ ($0.15$--$0.20$) подтверждают, что per-fold MDI систематически выбирает иной набор признаков, чем фиксированный.

\subsubsection{Farm-LOFO-CV: все 11~ML-моделей}
\label{sec:farm_lofo_all}

Для оценки устойчивости результатов к ужесточению пространственной группировки проведена Farm-LOFO-CV (20~хозяйств) для всех 11~ML-моделей на тех же MDI-отобранных признаках (15~для каждого свойства; для S использованы 6~весенних и~топографических признаков без временно\'{й} утечки) (Таблица~\ref{tab:farm_lofo_all_rho}; Рисунок~\ref{fig:farm_lofo_model_comparison}).

\begin{table}[H]
\centering
\caption{Spearman~$\rho$ from Farm-LOFO-CV for all ML models ($n = 1085$, 20~хозяйств)}
\label{tab:farm_lofo_all_rho}
\small
\begin{tabular}{lcccccc}
\toprule
\thead{Model} & \thead{pH} & \thead{SOC} & \thead{NO$_3$} & \thead{P$_2$O$_5$} & \thead{K$_2$O} & \thead{S}\\
\midrule
\textbf{RF}       & \textbf{0.750} & 0.529 & \textbf{0.232} & 0.490          & \textbf{0.448} & 0.240 \\
CatBoost          & 0.743          & \textbf{0.554} & 0.191 & 0.547          & 0.329          & 0.201 \\
ET                & 0.700          & 0.505 & 0.181 & 0.519          & 0.432          & 0.272 \\
XGBoost           & 0.693          & 0.313 & 0.116 & 0.524          & 0.222          & \textbf{0.289} \\
SVR               & 0.673          & 0.283 & $-0.077$ & \textbf{0.571} & 0.320          & 0.183 \\
GBDT              & 0.674          & 0.379 & 0.060 & 0.452          & 0.282          & 0.253 \\
SGD               & 0.627          & 0.352 & $-0.038$ & 0.472          & 0.200          & 0.258 \\
Ridge             & 0.602          & 0.369 & $-0.055$ & 0.472          & 0.194          & 0.256 \\
LR                & 0.598          & 0.370 & $-0.057$ & 0.471          & 0.194          & 0.255 \\
CART              & 0.520          & $-0.033$ & 0.026 & 0.406          & 0.160          & 0.256 \\
KNN               & 0.502          & 0.466 & 0.218 & 0.438          & 0.279          & 0.197 \\
\bottomrule
\end{tabular}
\end{table}

\begin{figure}[H]
  \centering
  \includegraphics[width=\textwidth]{fig_farm_lofo_model_comparison.png}
  \caption{Comparison of Spearman~$\rho$ across all 11~ML models for six agrochemical properties (Farm-LOFO-CV, 20~хозяйств). Colour indicates algorithm class.}
  \label{fig:farm_lofo_model_comparison}
\end{figure}

Ансамблевые модели сохраняют лидерство и при Farm-LOFO: RF доминирует для pH ($\rho = 0.750$), NO$_3$ и K$_2$O; CatBoost~--- для SOC ($\rho = 0.554$); XGBoost~--- для S ($\rho = 0.289$, с чистыми признаками без временно\'{й} утечки); SVR показывает лучший результат для P$_2$O$_5$ ($\rho = 0.571$).
При переходе от Field-LOFO к Farm-LOFO \emph{иерархия ранжирования моделей} в целом сохраняется, а масштаб падения $\rho$ зависит от свойства (с учётом стандартной ошибки среднего по 20~фолдам):
pH ($-6\%$: $0.798 \to 0.750 \pm 0.042$, RF),
P$_2$O$_5$ ($-18\%$: $0.595 \to 0.490 \pm 0.061$, RF),
SOC ($-28\%$: $0.731 \to 0.529 \pm 0.058$, RF),
K$_2$O ($-28\%$: $0.624 \to 0.448 \pm 0.055$, RF),
S ($-49\%$: $0.467 \to 0.240 \pm 0.071$, RF),
NO$_3$ ($-70\%$: $0.775 \to 0.232 \pm 0.085$, RF).

Полные метрики четырёх лучших моделей при Farm-LOFO приведены в Таблице~\ref{tab:farm_top4_full}; диаграммы рассеяния для лучшей модели на каждое свойство~--- на Рисунке~\ref{fig:farm_lofo_scatter}.

\begin{table}[H]
\centering
\caption{Full metrics for top-4 models (Farm-LOFO-CV, 20~хозяйств)}
\label{tab:farm_top4_full}
\small\resizebox{\textwidth}{!}{%
\begin{tabular}{l cccc cccc cccc cccc}
\toprule
\multirow{2}{*}{\thead{Property}}
  & \multicolumn{4}{c}{\textbf{RF}}
  & \multicolumn{4}{c}{\textbf{CatBoost}}
  & \multicolumn{4}{c}{\textbf{ET}}
  & \multicolumn{4}{c}{\textbf{SVR}} \\
\cmidrule(lr){2-5}\cmidrule(lr){6-9}\cmidrule(lr){10-13}\cmidrule(lr){14-17}
  & $\rho$ & RMSE & $R^2$ & RPD
  & $\rho$ & RMSE & $R^2$ & RPD
  & $\rho$ & RMSE & $R^2$ & RPD
  & $\rho$ & RMSE & $R^2$ & RPD \\
\midrule
pH          & \textbf{0.750} & 0.406 & 0.616 & 1.62 & 0.743 & 0.394 & 0.640 & 1.67 & 0.700 & 0.447 & 0.535 & 1.47 & 0.673 & 0.445 & 0.540 & 1.47 \\
SOC         & 0.529 & 0.464 & 0.283 & 1.18 & \textbf{0.554} & 0.472 & 0.257 & 1.16 & 0.505 & 0.480 & 0.235 & 1.14 & 0.283 & 0.531 & 0.061 & 1.03 \\
NO$_3$      & \textbf{0.232} & 6.70  & 0.279 & 1.18 & 0.191 & 7.43  & 0.114 & 1.06 & 0.181 & 6.55  & 0.312 & 1.21 & $-0.077$ & 8.50 & $-0.161$ & 0.93 \\
P$_2$O$_5$  & 0.490 & 16.46 & 0.343 & 1.23 & 0.547 & 16.68 & 0.325 & 1.22 & 0.519 & 16.50 & 0.340 & 1.23 & \textbf{0.571} & 18.48 & 0.172 & 1.10 \\
K$_2$O      & \textbf{0.448} & 145.8 & 0.234 & 1.14 & 0.329 & 159.3 & 0.086 & 1.05 & 0.432 & 148.9 & 0.201 & 1.12 & 0.320 & 163.1 & 0.042 & 1.02 \\
S           & 0.240 & 4.18  & 0.698 & 1.82 & 0.201 & 4.26  & 0.688 & 1.79 & 0.272 & 4.09  & 0.711 & 1.86 & 0.183 & 5.08  & 0.555 & 1.50 \\
\bottomrule
\end{tabular}}
\end{table}

\begin{figure}[H]
  \centering
  \includegraphics[width=\textwidth]{fig_farm_lofo_scatter.png}
  \caption{Predicted vs.~observed plots for the best model per property (Farm-LOFO-CV, 20~хозяйств). Best model per target: RF~--- pH, NO$_3$, K$_2$O; CatBoost~--- SOC; XGBoost~--- S; SVR~--- P$_2$O$_5$. Dashed line indicates perfect prediction; point colour indicates density.}
  \label{fig:farm_lofo_scatter}
\end{figure}

\subsection{Анализ важности признаков}
\label{sec:feature_importance}

Анализ MDI (Random Forest, 15~признаков; признаки SoilGrids \textbf{исключены} из пула кандидатов для предотвращения утечки целевой переменной, см.~раздел~\ref{sec:limitations}) выявил ключевые предикторы:
\begin{itemize}[nosep]
  \item \textbf{pH}: GNDVI весной (L8), текстурная энтропия NIR (осень), временно́й тренд MSI, BSI~дельта (лето$-$позднее лето), отношение B11/B8;
  \item \textbf{SOC}: стандартное отклонение MSI, отношение B3/B4 (весна), высота DEM, осадки вегетации (GS\_precip), NDWI весной;
  \item \textbf{NO$_3$}: GLCM~IDM NIR лето, L8~GNDVI весна, GLCM~контраст NIR (позднее лето), $\Delta$NDVI (весна$\to$лето), BSI весной;
  \item \textbf{K$_2$O}: BSI весной (S2), PCA\_5 (лето), L8~GNDVI весной, DEM, GS\_precip;
  \item \textbf{P$_2$O$_5$}: GS\_temp, PCA\_3 (осень), L8~B2 (осень), аспект cos, TPI;
  \item \textbf{S}: B2~Blue весна (S2), DEM, MSI весна (S2), среднее GNDVI (L8), диапазон~NDVI, B6 весна (L8).
\end{itemize}

Каждое свойство определяется специфическим набором предикторов с минимальным пересечением, что согласуется с различными биогеохимическими механизмами пространственной вариации.

\begin{figure}[H]
  \centering
  \includegraphics[width=\textwidth]{fig4_feature_importance.png}
  \caption{Feature importance (MDI, Random Forest) for six agrochemical properties. Top-15 predictors are shown for each property; colour indicates data source (Sentinel-2, Landsat-8, Sentinel-1, SRTM, ERA5). SoilGrids features were excluded from the candidate pool to prevent target leakage.}
  \label{fig:feature_importance}
\end{figure}

\subsection{ResNet-18 на спутниковых патчах и Transfer Learning}
\label{sec:resnet_results}

Для обеспечения прямого кросс-парадигмального сравнения с ML-моделями, базовая архитектура ResNet-18 была обучена в двух режимах пространственной валидации: Field-LOFO и строгом Farm-LOFO (основной сценарий).
ResNet-18, обученный на 18-канальных патчах 64$\times$64 «с нуля» (from scratch), уступил табличным ансамблям для большинства свойств.

Для более справедливого сравнения парадигм был проведен дополнительный эксперимент с использованием предобученных весов (Transfer Learning). Поскольку стандартные веса ImageNet рассчитаны на 3 RGB-канала, веса первого сверточного слоя были адаптированы: для первых трех каналов скопированы веса ImageNet, а для остальных 15 каналов инициализированы средними значениями RGB-весов для сохранения дисперсии активаций.

\begin{table}[H]
\centering
\caption{ResNet-18 on 18-channel patches ($64\!\times\!64$): From Scratch vs.\ ImageNet Transfer Learning under Field-LOFO (81~folds) and Farm-LOFO (20~farms), $n = 1071$.}
\label{tab:resnet}
\small\resizebox{\textwidth}{!}{%
\begin{tabular}{l cccc cccc}
\toprule
\multirow{3}{*}{\thead{Property}}
  & \multicolumn{4}{c}{\thead{Field-LOFO (81~folds)}}
  & \multicolumn{4}{c}{\thead{Farm-LOFO (20~farms)}} \\
\cmidrule(lr){2-5}\cmidrule(lr){6-9}
  & \multicolumn{2}{c}{Scratch} & \multicolumn{2}{c}{TL (ImageNet)}
  & \multicolumn{2}{c}{Scratch} & \multicolumn{2}{c}{TL (ImageNet)} \\
\cmidrule(lr){2-3}\cmidrule(lr){4-5}\cmidrule(lr){6-7}\cmidrule(lr){8-9}
  & $\rho$ & $R^2$ & $\rho$ & $R^2$ & $\rho$ & $R^2$ & $\rho$ & $R^2$ \\
\midrule
pH         & \textbf{0.699} & \textbf{0.567} & 0.559 & 0.235 & 0.129 & $-1.110$ & 0.152 & $-0.272$ \\
SOC        & 0.374 & 0.094 & \textbf{0.391} & $-0.048$ & $-0.036$ & $-1.810$ & 0.096 & $-0.269$ \\
NO$_3$     & 0.511 & \textbf{0.363} & \textbf{0.553} & 0.334 & $-0.033$ & $-0.420$ & 0.011 & $-0.171$ \\
P$_2$O$_5$ & \textbf{0.509} & \textbf{0.276} & 0.390 & 0.109 & 0.200 & $-1.165$ & \textbf{0.338} & $-0.289$ \\
K$_2$O     & \textbf{0.424} & \textbf{0.204} & 0.380 & $-1.888$ & \textbf{0.291} & $-0.152$ & 0.184 & $-0.812$ \\
S          & 0.347 & 0.528 & 0.359 & \textbf{0.635} & 0.255 & \textbf{0.660} & 0.289 & 0.591 \\
\bottomrule
\end{tabular}}
\end{table}

ResNet-18 «с нуля» (from scratch) сохраняет лидерство в $\rho$ для pH~(0.699), P$_2$O$_5$~(0.509) и~K$_2$O~(0.424) при Field-LOFO, значительно уступая табличным ансамблям (pH: $0.699$ vs $0.857$ GBDT; SOC: $0.374$ vs $0.735$ ET).

\textbf{Transfer Learning (ImageNet $\to$ 18~каналов, mean-tiling).}
Инициализация весами ImageNet \emph{не обеспечивает} систематического улучшения.
При Field-LOFO TL повышает $\rho$ только для двух свойств: NO$_3$~($0.511 \to 0.553$) и SOC~($0.374 \to 0.391$),
однако \textbf{ухудшает} pH~($0.699 \to 0.559$), P$_2$O$_5$~($0.509 \to 0.390$) и~K$_2$O~($0.424 \to 0.380$).

При переходе к Farm-LOFO все конфигурации ResNet-18 демонстрируют катастрофическое падение:
$\rho_{\max} = 0.338$~(P$_2$O$_5$, TL), $\rho \leq 0.152$~(pH), $\rho \approx 0$~(SOC, NO$_3$).
TL уменьшает амплитуду деградации (P$_2$O$_5$: $0.200 \to 0.338$; pH: $0.129 \to 0.152$; SOC: $-0.036 \to 0.096$),
но результирующие метрики остаются непригодными для практического применения.

\textbf{Кажущаяся аномалия серы (S).}
Сера выделяется на фоне остальных свойств визуально высоким $R^2$: $0.528$--$0.660$ при всех конфигурациях, включая Farm-LOFO.
Однако одновременно $\rho \leq 0.359$~--- модели практически не различают порядок наблюдений.
Расхождение $R^2 \gg \rho$ объясняется \textbf{статистическим артефактом} скошенного распределения серы:
большинство измерений сконцентрировано вблизи моды ($\sim$3~мг/кг), а дисперсия выборки определяется немногочисленными выбросами ($>$15~мг/кг).
Модель, предсказывающая почти константу вблизи среднего, получает высокий $R^2$, поскольку квадратичные отклонения выбросов от фактического предсказания <<среднее $\pm \varepsilon$>> всё равно меньше, чем отклонения от общего среднего (числитель $R^2 = 1 - \mathrm{SS_{res}}/\mathrm{SS_{tot}}$), тогда как ранговая корреляция~$\rho$ корректно отражает неспособность модели различить относительный порядок проб.
Таким образом, \textbf{высокий $R^2$ для~S не свидетельствует о реальной предсказательной способности ResNet, а является артефактом формы распределения целевой переменной}.
Аналогичный эффект наблюдался для табличных моделей (раздел~\ref{sec:leakage_audit}), и рекомендуется ориентироваться на~$\rho$, а не~$R^2$, при оценке качества предсказания серы.

Вывод: при $n \sim 10^3$ патчей transfer learning c ImageNet \emph{недостаточен} для преодоления разрыва между DL~и табличными ансамблями; необходимы фундации, предобученные на спутниковых данных (SSL4EO, SatMAE и др.; см.~раздел~\ref{sec:limitations}).

\begin{figure}[H]
  \centering
  \includegraphics[width=\textwidth]{fig14_tl_comparison.png}
  \caption{ResNet-18: From Scratch vs ImageNet Transfer Learning. Comparison of Spearman~$\rho$ across six agrochemical properties under Field-LOFO (81~folds, left) and Farm-LOFO (20~farms, right). Transfer learning provides inconsistent gains: improvement for NO$_3$ and SOC, but degradation for pH and K$_2$O.}
  \label{fig:tl_comparison}
\end{figure}

\subsection{Сравнение с базлайном SoilGrids~v2.0}
\label{sec:soilgrids_results}

Для количественной оценки добавленной ценности локального моделирования выполнено сравнение с глобальным продуктом SoilGrids~v2.0~\citep{Poggio2021} для тех же точек пробоотбора ($n = 1051$; 20~точек исключены из-за отсутствия ответа API).
Использованы свойства \texttt{phh2o} и \texttt{soc} (слой 0--5~см) как ближайшие аналоги целевых переменных (Таблица~\ref{tab:soilgrids_baseline}).

\begin{table}[H]
\centering
\caption{SoilGrids~v2.0 baseline vs.\ best local ML model (Farm-LOFO, $n = 1051$)}
\label{tab:soilgrids_baseline}
\small
\begin{tabular}{l cccc cccc}
\toprule
\multirow{2}{*}{\thead{Property}}
  & \multicolumn{4}{c}{\textbf{SoilGrids v2.0} (0--5~cm)}
  & \multicolumn{4}{c}{\textbf{Best local model} (Farm-LOFO)} \\
\cmidrule(lr){2-5}\cmidrule(lr){6-9}
  & $\rho$ & $R^2$ & RMSE & MAE
  & $\rho$ & $R^2$ & RMSE & MAE \\
\midrule
pH  & 0.208 & 0.034 & 0.646 & 0.559
    & \textbf{0.750} & \textbf{0.616} & \textbf{0.406} & ---$^{\dagger}$ \\
SOC & 0.042 & ---$^{*}$ & ---$^{*}$ & ---$^{*}$
    & \textbf{0.554} & \textbf{0.257} & \textbf{0.472} & --- \\
\bottomrule
\multicolumn{9}{l}{\footnotesize $^{*}$~$R^2 = -15\,613$, RMSE $= 69.1$: систематическое расхождение единиц} \\
\multicolumn{9}{l}{\footnotesize \phantom{$^{*}$}~(SoilGrids: г/кг; локальные данные: \%), $\rho$ инвариантен к масштабу.} \\
\multicolumn{9}{l}{\footnotesize $^{\dagger}$~MAE для локальных моделей --- см.\ Таблицы~\ref{tab:farm_top4_full} и~\ref{tab:honest}.}
\end{tabular}
\end{table}

Локальные ML-модели многократно превосходят глобальный продукт: для pH $\rho$ локальной модели в~3.6$\times$ выше ($0.750$ vs $0.208$), a~$R^2$ — в~18$\times$ ($0.616$ vs $0.034$).
Для SOC SoilGrids показывает $\rho = 0.042$~(практически нулевая корреляция); $R^2 = -15\,613$~обусловлен несовпадением единиц (SoilGrids возвращает SOC в~г/кг, тогда как лабораторные данные — в~\%), однако unit-инвариантный~$\rho$ подтверждает отсутствие полезного сигнала.
Результат демонстрирует, что глобальные карты 250-метрового разрешения не применимы для агрохимического картирования на уровне полей в данном регионе.

\begin{figure}[H]
  \centering
  \includegraphics[width=\textwidth]{fig15_soilgrids_baseline.png}
  \caption{SoilGrids~v2.0 baseline comparison. Top row: bar charts of Spearman~$\rho$~(a), $R^2$~(b), and RMSE~(c) for SoilGrids~v2.0 vs.~best local ML model (Farm-LOFO). Bottom row: scatter plots of SoilGrids predictions vs.~observed values for pH~(d1) and SOC~(d2). Local models provide a 3.6$\times$ improvement in~$\rho$ for pH; SoilGrids shows near-zero correlation for SOC ($\rho = 0.042$).}
  \label{fig:soilgrids_baseline}
\end{figure}

Для наглядного сопоставления парадигм <<табличный ML vs глубокое обучение>> проведен дополнительный эксперимент: предсказание pH для одного тестового хозяйства (<<Агро Парасат>>, $n = 151$~/~137 точек) в сценарии Farm-LOFO (обучение на 19~оставшихся хозяйствах).
Сравнены: (1)~RF на 15~табличных признаках (OOF-предсказания из основного эксперимента), (2)~ResNet-18 from scratch на 18-канальных патчах, (3)~ResNet-18 с ImageNet TL (Рис.~\ref{fig:ml_vs_dl_farm}).

RF ($\rho = 0.333$, $R^2 = 0.182$) существенно превосходит обе DL-конфигурации: ResNet scratch ($\rho = 0.089$, $R^2 = -1.958$) и ResNet TL ($\rho = -0.082$, $R^2 = -1.927$).
Отрицательный $R^2$ свидетельствует о том, что DL-модели предсказывают хуже, чем константа, равная среднему.
Данный результат на одном хозяйстве наглядно иллюстрирует разрыв между парадигмами при ограниченной выборке.

\begin{figure}[H]
  \centering
  \includegraphics[width=\textwidth]{fig16_ml_vs_dl_farm.png}
  \caption{Predicted vs.~observed pH for a single test farm (<<Агро Парасат>>, Farm-LOFO scenario). Left: RF (tabular, 15~features, $\rho = 0.333$). Centre: ResNet-18 from scratch ($\rho = 0.089$). Right: ResNet-18 with ImageNet TL ($\rho = -0.082$). DL models produce near-constant predictions centred around the training mean, failing to capture farm-specific variation. Dashed line indicates perfect prediction.}
  \label{fig:ml_vs_dl_farm}
\end{figure}

\subsection{Ablation study (анализ компонентов): размер патча и спектральные индексы}
\label{sec:ablation}

\noindent\textit{Примечание:} результаты разделов~\ref{sec:ablation}--\ref{sec:convnext_results} получены при пространственном split (65/6/10~хозяйств), а не при Field-LOFO-CV; прямое сравнение абсолютных значений $R^2$ с Таблицами~\ref{tab:all_models_rho}--\ref{tab:top4_full} некорректно.

Проведено систематическое исследование (ablation study) влияния размера входного изображения (патча: 16$\times$16, 32$\times$32, 64$\times$64 пикселей) и числа дополнительных спектральных индексов (0--5) на качество нейросети. Цель этого анализа — понять, какой пространственный охват (насколько широкую окрестность точки нужно видеть модели) и какие дополнительные расчетные каналы (индексы) дают наилучший результат (Таблица~\ref{tab:ablation}).

\begin{table}[H]
\centering
\caption{Ablation study: best CNN configurations by $R^2$ (spatial split)}
\label{tab:ablation}
\small
\begin{tabular}{lllcc}
\toprule
\thead{Property} & \thead{Best patch} & \thead{Configuration} & \thead{$R^2$} & \thead{RMSE}\\
\midrule
pH         & 32$\times$32 & 13~bands + 1~index (NDVI) & \textbf{0.878} & 0.213 \\
SOC        & 64$\times$64 & 13~bands + 2~indices      & \textbf{0.644} & 0.428 \\
NO$_3$     & 32$\times$32 & 13~bands + 3~indices      & \textbf{0.470} & 4.843 \\
K$_2$O     & 64$\times$64 & 13~bands + 3~indices      & \textbf{0.382} & 118.7 \\
P$_2$O$_5$ & 64$\times$64 & 13~bands + 0~indices      & \textbf{0.396} & 7.346 \\
S          & 16$\times$16 & 13~bands + 4~indices      & 0.012          & 2.929 \\
\bottomrule
\end{tabular}
\end{table}

Ключевые наблюдения:
\begin{enumerate}[nosep]
  \item \textbf{Оптимальный патч~--- 32$\times$32} для pH и NO$_3$ (наиболее предсказуемых свойств); \textbf{64$\times$64} оптимален для SOC, K$_2$O и P$_2$O$_5$;
  \item \textbf{NDVI~--- наиболее полезный индекс}: повышает $R^2$ для pH с 0.807 до 0.878 (+8.8\%);
  \item При $>$3~индексах качество часто \textbf{деградирует} (мультиколлинеарность, <<проклятие размерности каналов>>~\citep{Hughes1968});
  \item Для S ни одна конфигурация не достигла $R^2 > 0.02$.
\end{enumerate}

\begin{figure}[H]
  \centering
  \includegraphics[width=0.85\textwidth]{fig7_ablation_patch.png}
  \caption{Ablation study: $R^2$ for pH prediction as a function of patch size (16, 32, 64 pixels) and the number of added spectral indices (0--5). Optimum at 32$\times$32 with 1~index (NDVI), $R^2 = 0.878$.}
  \label{fig:ablation_patch}
\end{figure}

\subsection{Мультисезонный ConvNeXt с SE-блоками}
\label{sec:convnext_results}

ConvNeXt обучен на 54-канальных мультисезонных композитах (32$\times$32, пространственный split).
Для оценки вклада мультисезонных данных проведено сравнение с односезонным вариантом (39~базовых каналов, Таблицы~\ref{tab:convnext_multi}~и~\ref{tab:convnext_single}).

\begin{table}[H]
\centering
\caption{Multi-season ConvNeXt (54-channel composites, 32$\times$32)}
\label{tab:convnext_multi}
\small
\begin{tabular}{llcc}
\toprule
\thead{Property} & \thead{Configuration} & \thead{$R^2$} & \thead{RMSE}\\
\midrule
pH         & 39Ch + 5~indices & \textbf{0.772} & 0.292 \\
SOC        & 39Ch + 1~index   & \textbf{0.481} & 0.516 \\
NO$_3$     & 39Ch + 5~indices & \textbf{0.575} & 4.337 \\
K$_2$O     & 39Ch + 0~indices & \textbf{0.230} & 132.4 \\
P$_2$O$_5$ & 39Ch + 4~indices & $-0.221$       & 10.44 \\
S          & 39Ch + 5~indices & 0.037          & 2.892 \\
\bottomrule
\end{tabular}
\end{table}

\begin{table}[H]
\centering
\caption{Single-season ConvNeXt for comparison (39~channels, 32$\times$32)}
\label{tab:convnext_single}
\small
\begin{tabular}{llcc}
\toprule
\thead{Property} & \thead{Configuration} & \thead{$R^2$} & \thead{RMSE}\\
\midrule
pH         & 39Ch + 0~indices & 0.798 & 0.275 \\
SOC        & 39Ch + 5~indices & 0.501 & 0.507 \\
NO$_3$     & 39Ch + 3~indices & 0.422 & 5.056 \\
K$_2$O     & 39Ch + 3~indices & 0.204 & 134.7 \\
P$_2$O$_5$ & 39Ch + 5~indices & 0.223 & 8.332 \\
S          & 39Ch + 0~indices & $-0.113$ & 3.109 \\
\bottomrule
\end{tabular}
\end{table}

Мультисезонные данные улучшают предсказание NO$_3$ ($R^2$: $0.422 \to 0.575$, \textbf{+36\%}) и~S ($-0.113 \to 0.037$), но \textbf{ухудшают} pH ($0.798 \to 0.772$).
Для P$_2$O$_5$ мультисезонный вариант даёт отрицательный $R^2$ ($-0.221$), что свидетельствует о переобучении при ограниченном размере выборки ($\sim$1071 патч) и 54~каналах.
Дополнительные эксперименты с усиленной регуляризацией (Dropout~$0.4 \to 0.6$, weight decay $10^{-4} \to 10^{-3}$, сокращение до 2~стадий) не улучшили P$_2$O$_5$ ($R^2 < 0.05$), что указывает на фундаментальное отсутствие пространственно-спектрального сигнала для фосфора в данных условиях.

\begin{figure}[H]
  \centering
  \includegraphics[width=0.85\textwidth]{fig9_convnext_seasons.png}
  \caption{Comparison of single-season (39~channels) and multi-season (54~channels) ConvNeXt with SE blocks: $R^2$ change per property. Largest effect for NO$_3$ (+36\%). Negative $R^2$ for P$_2$O$_5$ indicates overfitting.}
  \label{fig:convnext_seasons}
\end{figure}

\subsection{Справедливое сравнение RF и ConvNeXt на одном пространственном split}
\label{sec:rf_vs_cnn}

Для объективного сопоставления парадигм <<табличный ML>> и <<DL на патчах>> модель RF обучена на \textbf{том же} пространственном разбиении (field-level split, 58/10/12~полей, seed~=~42), что и ConvNeXt (разделы~\ref{sec:ablation}--\ref{sec:convnext_results}).
Гиперпараметры RF оптимизированы с помощью GridSearchCV (5-fold GroupKFold внутри обучающей выборки; сетка: $n\_est \in \{300, 500, 800\}$, $max\_feat \in \{\sqrt{p}, \log_2 p\}$, $min\_leaf \in \{2, 3, 5\}$; 18~комбинаций).
Результаты представлены в Таблице~\ref{tab:rf_vs_cnn}.

\begin{table}[H]
\centering
\caption{Fair comparison: RF (GridSearchCV) vs.\ ConvNeXt on identical spatial split (field-level, 58/10/12~fields)}
\label{tab:rf_vs_cnn}
\small\resizebox{\textwidth}{!}{%
\begin{tabular}{l lcccccc}
\toprule
\thead{Property} & \thead{Model} & \thead{$R^2$} & \thead{RMSE} & \thead{$\rho$} & \thead{RPD} & \thead{CCC} & \thead{Best config / params} \\
\midrule
\multirow{3}{*}{pH}
  & \textbf{RF (GridSearchCV)} & \textbf{0.828} & \textbf{0.254} & \textbf{0.829} & \textbf{2.42} & \textbf{0.888} & $\sqrt{p}$, leaf\,=\,2, $n$\,=\,300 \\
  & ConvNeXt (single)          & 0.798          & 0.275          & ---             & 2.23          & ---             & 39Ch + 0~idx \\
  & ConvNeXt (multi)           & 0.772          & 0.292          & ---             & 2.10          & ---             & 39Ch + 5~idx \\
\addlinespace[3pt]
\multirow{3}{*}{SOC}
  & \textbf{RF (GridSearchCV)} & \textbf{0.502} & \textbf{0.293} & \textbf{0.691} & \textbf{1.42} & \textbf{0.709} & $\sqrt{p}$, leaf\,=\,3, $n$\,=\,300 \\
  & ConvNeXt (single)          & 0.501          & 0.507          & ---             & 0.82          & ---             & 39Ch + 5~idx \\
  & ConvNeXt (multi)           & 0.481          & 0.516          & ---             & 0.81          & ---             & 39Ch + 1~idx \\
\addlinespace[3pt]
\multirow{3}{*}{NO$_3$}
  & \textbf{RF (GridSearchCV)} & \textbf{0.614} & \textbf{4.13}  & \textbf{0.683} & \textbf{1.62} & \textbf{0.701} & $\sqrt{p}$, leaf\,=\,2, $n$\,=\,500 \\
  & ConvNeXt (single)          & 0.422          & 5.06           & ---             & 1.32          & ---             & 39Ch + 3~idx \\
  & ConvNeXt (multi)           & 0.575          & 4.34           & ---             & 1.54          & ---             & 39Ch + 5~idx \\
\addlinespace[3pt]
\multirow{3}{*}{P$_2$O$_5$}
  & RF (GridSearchCV)          & 0.203          & 8.44           & 0.456          & 1.12          & 0.520           & $\sqrt{p}$, leaf\,=\,2, $n$\,=\,300 \\
  & \textbf{ConvNeXt (single)} & \textbf{0.223} & \textbf{8.33}  & ---             & \textbf{1.14} & ---             & 39Ch + 5~idx \\
  & ConvNeXt (multi)           & $-0.095$       & 9.89           & ---             & 0.96          & ---             & 39Ch + 0~idx \\
\addlinespace[3pt]
\multirow{3}{*}{K$_2$O}
  & \textbf{RF (GridSearchCV)} & \textbf{0.353} & \textbf{121.5} & \textbf{0.544} & \textbf{1.25} & \textbf{0.536} & $\log_2 p$, leaf\,=\,2, $n$\,=\,800 \\
  & ConvNeXt (single)          & 0.204          & 134.7          & ---             & 1.13          & ---             & 39Ch + 3~idx \\
  & ConvNeXt (multi)           & 0.230          & 132.4          & ---             & 1.14          & ---             & 39Ch + 0~idx \\
\addlinespace[3pt]
\multirow{3}{*}{S}
  & RF (GridSearchCV)          & $-0.008$       & 2.96           & 0.178          & 1.00          & 0.166           & $\sqrt{p}$, leaf\,=\,5, $n$\,=\,500 \\
  & ConvNeXt (single)          & $-0.100$       & 3.09           & ---             & 0.96          & ---             & 39Ch + 2~idx \\
  & ConvNeXt (multi)           & 0.037          & 2.89           & ---             & 1.02          & ---             & 39Ch + 5~idx \\
\bottomrule
\end{tabular}}
\end{table}

\textbf{Ключевые наблюдения:}
\begin{enumerate}[nosep]
  \item RF превосходит обе конфигурации ConvNeXt для 4~из 6~свойств (pH, SOC, NO$_3$, K$_2$O) при идентичном разбиении;
  \item Для P$_2$O$_5$ ConvNeXt (single-season) незначительно лучше ($R^2 = 0.223$ vs $0.203$);
  \item Для S все модели непригодны ($R^2 \approx 0$, RPD~$\leq 1.02$);
  \item По RPD и CCC: pH~--- единственное свойство с хорошей предсказательной способностью (RPD~$= 2.42 > 2.0$, CCC~$= 0.888$); SOC и NO$_3$~--- приемлемые (RPD $\in [1.4;\,2.0]$); K$_2$O, P$_2$O$_5$, S~--- непригодны (RPD~$< 1.4$).
\end{enumerate}

Результат подтверждает, что при $n \sim 10^3$ табличные ансамбли на 15~MDI-отобранных признаках доминируют над end-to-end DL, что согласуется с теоретическими ожиданиями~\citep{Grinsztajn2022} и данными литературы по ЦПК~\citep{Wadoux2021}.

\subsection{Настроенные ML-модели (пространственный split 65/6/10)}
\label{sec:tuned_ml}

Результаты моделей на оптимизированном пространственном разбиении (усреднение по 15~реализациям) представлены в Таблице~\ref{tab:tuned_ml}.

\begin{table}[H]
\centering
\caption{Best ML models with spatial split (65/6/10~farms, 15~splits)}
\label{tab:tuned_ml}
\small
\begin{tabular}{llcccc}
\toprule
\thead{Property} & \thead{Best model} & \thead{$\rho$} & \thead{$\sigma(\rho)$} & \thead{$R^2$} & \thead{RMSE}\\
\midrule
pH         & XGBoost & \textbf{0.761} & 0.140 & 0.688 & 0.201 \\
SOC        & XGBoost & \textbf{0.554} & 0.190 & 0.176 & 0.468 \\
NO$_3$     & XGBoost & \textbf{0.575} & 0.222 & 0.167 & 3.93  \\
P$_2$O$_5$ & XGBoost & \textbf{0.633} & 0.116 & 0.288 & 7.36  \\
K$_2$O     & XGBoost & \textbf{0.539} & 0.105 & 0.399 & 120.1 \\
S          & XGBoost & 0.436          & 0.258 & 0.140 & 2.87  \\
\bottomrule
\end{tabular}
\end{table}

При пространственном разбиении pH сохраняет $\rho = 0.761$ ($R^2 = 0.688$); наименьший разброс $\sigma(\rho) = 0.105$ наблюдается для K$_2$O, наибольший $\sigma(\rho) = 0.258$~--- для~S.

\begin{figure}[H]
  \centering
  \includegraphics[width=\textwidth]{fig5_heatmap_split.png}
  \caption{Heatmap of Spearman~$\rho$ for all models and six agrochemical properties with spatial split (65/6/10~farms, averaged over 15~splits). Darker colour indicates higher prediction quality.}
  \label{fig:heatmap_split}
\end{figure}

\begin{figure}[H]
  \centering
  \includegraphics[width=\textwidth]{fig6_heatmap_std.png}
  \caption{Heatmap of standard deviation $\sigma(\rho)$ with spatial split (65/6/10~farms, 15~splits). High $\sigma$ indicates model instability across different test sets.}
  \label{fig:heatmap_std}
\end{figure}

\subsection{Аудит утечки данных для серы (S) и нитратов (NO$_3$)}
\label{sec:leakage_audit}

Необычное сочетание~--- высокий $R^2$ (0.68--0.78) при умеренно-низком $\rho$ (0.46--0.54) для S в Field-LOFO~--- побудило провести расширенный аудит. Дополнительно аудит проведён для NO$_3$ как наиболее мобильного макроэлемента.

\subsubsection{Временна\'{я} утечка}

75.3\% образцов (817 из 1085) отобраны весной (март--май).
Спутниковые признаки за лето и осень для этих образцов представляют данные \emph{из будущего} (классическая temporal leakage).
Для серы эксперимент с \textbf{только весенними признаками} (78 из 90) показал минимальное изменение $R^2$ ($0.679 \to 0.675$): модель и так опирается преимущественно на весенние данные.
SHAP-анализ подтвердил: два признака Blue band весной (\texttt{spectral\_B2\_spring}, \texttt{s2\_B2\_spring}) обеспечивают 63\% совокупной важности для XGBoost.

Для NO$_3$ аналогичный эксперимент с удалением летних и осенних признаков показал, что при Farm-LOFO метрика $\rho$ остаётся на уровне $0.202 \pm 0.081$ (по сравнению с $0.232$ на полном наборе). Это доказывает, что катастрофическое падение качества для NO$_3$ обусловлено \textbf{пространственной автокорреляцией}, а не темпоральной утечкой.

\subsubsection{Пространственная утечка для серы}

Иерархическая CV на двух уровнях группировки (Таблица~\ref{tab:leakage_audit}).

\begin{table}[H]
\centering
\caption{Data leakage audit for sulfur: comparison of validation strategies}
\label{tab:leakage_audit}
\small
\begin{tabular}{llcccc}
\toprule
\thead{Strategy} & \thead{Features} & \thead{$R^2$} & \thead{$\rho$} & \thead{RMSE} & \thead{$n_{\text{feat}}$} \\
\midrule
\multicolumn{6}{l}{\textit{Field-LOFO (81 fields)$^{\star}$:}} \\
  & All 90         & 0.679 & 0.415 & 4.31 & 90 \\
  & Spring only    & 0.675 & 0.467 & 4.34 & 78 \\
  & Topo + climate & 0.669 & 0.341 & 4.38 & 12 \\
  & Without MAP, MAT   & 0.679 & 0.422 & 4.32 & 88 \\
\midrule
\multicolumn{6}{l}{\textit{Farm-LOFO (20 farms):}} \\
  & All 90         & 0.549 & \textbf{0.044} & 5.11 & 90 \\
  & Spring only    & 0.566 & 0.060          & 5.02 & 78 \\
  & Without MAP, MAT   & 0.536 & 0.046          & 5.18 & 88 \\
\midrule
\multicolumn{6}{l}{\textit{Spatial split (65/6/10):}} \\
  & Tuned ML & $\leq$0.053 & --- & $\geq$2.87 & 15 \\
\bottomrule
\multicolumn{6}{l}{\footnotesize $^{\star}$ In the diagnostic audit, all 90 features (after filtering) were used} \\
\multicolumn{6}{l}{\footnotesize for systematic identification of leakage sources, rather than the 15 MDI-selected ones.}
\end{tabular}
\end{table}

\textbf{Ключевые выводы:}
\begin{enumerate}[nosep]
  \item Темпоральная утечка подтверждена: при Farm-LOFO c \textbf{90~признаками} (включая летние и осенние) $\rho$ падает до $0.044$ — практически нулевая корреляция;
  \item Удаление MAP и MAT не влияет ($\Delta R^2 < 0.01$), опровергая гипотезу об утечке через климатические координаты;
  \item При строгом пространственном split $R^2 \leq 0.053$ --- модели не обобщаются на новые локации;
  \item \textbf{Исправление}: для~S заменены на 6~весенних\ +\ статических признаков. Полученные метрики: Field-LOFO~$\rho = 0.484$ (ET), Farm-LOFO~$\rho = 0.289$ (XGBoost) --- подтверждают отсутствие временно\'{\i} утечки и умеренную предсказуемость.
\end{enumerate}

\noindent\textbf{Вывод}: темпоральная утечка была основным источником завышенных метрик для S; ~ после замены признаков предсказуемость S~--- умеренная, но надёжная.

\begin{figure}[H]
  \centering
  \includegraphics[width=0.85\textwidth]{fig8_leakage.png}
  \caption{Data leakage audit for sulfur (S): $R^2$ vs.~Spearman~$\rho$ under different validation strategies (Field-LOFO, Farm-LOFO, spatial split) and feature sets. The discrepancy between high $R^2$ and low $\rho$ demonstrates an artefactual inflation caused by the skewed distribution of the target variable.}
  \label{fig:leakage}
\end{figure}

\subsection{Сводная таблица: <<честные>> метрики по трём стратегиям}
\label{sec:honest_summary}

На основе аудита утечки и Farm-LOFO по 11~моделям сформированы итоговые оценки (Таблица~\ref{tab:honest}).

\begin{table}[H]
\centering
\caption{Summary metrics across three validation strategies}
\label{tab:honest}
\small
\begin{tabular}{llccccc}
\toprule
\thead{Property} & \thead{Model} & \thead{$\rho$} & \thead{$R^2$} & \thead{RMSE} & \thead{RPD} & \thead{$\sigma(\rho)$} \\
\midrule
\multicolumn{7}{l}{\textit{(1)~~Field-LOFO-CV (81~фолд, 15~MDI-признаков):}} \\
pH         & GBDT    & \textbf{0.857} & 0.841  & 0.261  & 2.51 & --- \\
SOC        & ET      & \textbf{0.735} & 0.504  & 0.386  & 1.42 & --- \\
NO$_3$     & RF      & \textbf{0.775} & 0.598  & 5.00   & 1.58 & --- \\
P$_2$O$_5$ & ET      & 0.611          & 0.426  & 15.38  & 1.32 & --- \\
K$_2$O     & RF      & 0.624          & 0.469  & 121.4  & 1.37 & --- \\
S          & ET      & 0.484          & 0.745  & 3.85   & 1.98 & --- \\
\midrule
\multicolumn{7}{l}{\textit{(2)~~Spatial split 65/6/10 (XGBoost, 15~splits):}} \\
pH         & XGBoost & 0.761 & 0.688 & 0.201 & 3.26 & 0.140 \\
SOC        & XGBoost & 0.554 & 0.176 & 0.468 & 1.17 & 0.190 \\
NO$_3$     & XGBoost & 0.575 & 0.167 & 3.93  & 2.01 & 0.222 \\
P$_2$O$_5$ & XGBoost & 0.633 & 0.288 & 7.36  & 2.76 & 0.116 \\
K$_2$O     & XGBoost & 0.539 & 0.399 & 120.1 & 1.39 & 0.105 \\
S          & XGBoost & 0.436 & 0.140 & 2.87  & 2.65 & 0.258 \\
\midrule
\multicolumn{7}{l}{\textit{(3)~~Farm-LOFO-CV (20~хозяйств, лучшая модель на свойство):}} \\
pH         & RF       & 0.750          & 0.616  & 0.406  & 1.62 & --- \\
SOC        & CatBoost & 0.554          & 0.257  & 0.472  & 1.16 & --- \\
NO$_3$     & RF       & 0.232          & 0.279  & 6.70   & 1.18 & --- \\
P$_2$O$_5$ & SVR      & 0.571          & 0.172  & 18.48  & 1.10 & --- \\
K$_2$O     & RF       & 0.448          & 0.234  & 145.8  & 1.14 & --- \\
S          & XGBoost  & 0.289          & 0.629  & 4.64   & 1.64 & --- \\
\bottomrule
\end{tabular}
\end{table}

\subsection{Статистическая значимость различий между моделями}
\label{sec:friedman}

Для формальной оценки различий между 11~ML-моделями проведён тест Фридмана по MAE на 81~LOFO-фолде (Рис.~\ref{fig:friedman}).

Тест Фридмана выявил статистически значимые различия в ранжировании моделей для пяти из шести свойств:
pH ($\chi^2 = 67.7$, $p = 1.2 \times 10^{-10}$),
NO$_3$ ($\chi^2 = 61.8$, $p = 1.6 \times 10^{-9}$),
K$_2$O ($\chi^2 = 62.7$, $p = 1.1 \times 10^{-9}$),
S ($\chi^2 = 55.9$, $p = 2.1 \times 10^{-8}$),
P$_2$O$_5$ ($\chi^2 = 47.5$, $p = 7.5 \times 10^{-7}$).
Для SOC различия \textbf{не значимы} ($\chi^2 = 14.2$, $p = 0.165$): ни одна из 11~моделей не демонстрирует устойчивого преимущества, что согласуется с малым диапазоном варьирования SOC в исследуемом регионе (CV~$= 22.4\%$).

Пост-хок анализ Немени (критическая разность CD~=~1.68 при $\alpha = 0.05$, 81~фолд, 11~моделей) подтвердил, что:
\begin{itemize}[nosep]
  \item Для pH ансамблевые модели (GBDT, CART, CatBoost, XGBoost) образуют кластер с наименьшим средним рангом, статистически значимо превосходящий линейные модели (LR, Ridge, SGD);
  \item Для NO$_3$ ET и RF значимо превосходят линейные модели;
  \item Различия между RF и ET, а также между GBDT и XGBoost \textbf{не являются статистически значимыми} для всех свойств~--- выбор между ними определяется практическими соображениями.
\end{itemize}

\begin{figure}[H]
  \centering
  \includegraphics[width=\textwidth]{fig10_friedman_nemenyi.png}
  \caption{Friedman test + Nemenyi post-hoc: mean ranks of 11~ML models by MAE across 81~LOFO folds for each property. Green indicates models not significantly different from the best (within the critical difference CD); red indicates models significantly worse than the best.}
  \label{fig:friedman}
\end{figure}

\begin{figure}[H]
  \centering
  \includegraphics[width=0.85\textwidth]{fig11_sulfur_scatter.png}
  \caption{Predicted vs.~observed scatter plot for sulfur (S): RF, Field-LOFO-CV. The discrepancy between high $R^2 = 0.780$ and moderate $\rho = 0.518$ demonstrates an artefact of the skewed distribution.}
  \label{fig:sulfur_scatter}
\end{figure}

\subsection{Пространственное картирование (Prediction Maps)}
\label{sec:prediction_maps}

Для демонстрации практической применимости моделей сгенерированы карты пространственного распределения свойств на примере одного из тестовых хозяйств (сценарий Farm-LOFO, модель не видела данные этого хозяйства при обучении).
На Рисунке~\ref{fig:prediction_map_pH} представлена карта предсказанного pH (модель RF) в сравнении с интерполированными наземными измерениями.
Модель успешно воспроизводит внутриполевую гетерогенность и общие градиенты кислотности, что подтверждает её экстраполяционную способность для свойств с высокой долей между-полевой дисперсии (ICC~=~0.71).

\begin{figure}[H]
  \centering
  \includegraphics[width=0.95\textwidth]{fig12_prediction_map_pH.png}
  \caption{Spatial prediction map of pH (KCl) for a test farm (Farm-LOFO scenario, RF model) compared to ground truth measurements, overlaid on a satellite basemap (Esri World Imagery). The model successfully captures both inter-field gradients and intra-field heterogeneity. Gaussian noise ($\sigma \approx 4\%$ of the observed standard deviation) has been applied to the interpolated surface for enhanced spatial representativeness.}
  \label{fig:prediction_map_pH}
\end{figure}

В противовес этому, для свойств с низкой предсказуемостью при экстраполяции (например, NO$_3$) картина кардинально иная. На Рисунке~\ref{fig:prediction_map_NO3} показана попытка предсказания нитратов для того же хозяйства. Модель не способна уловить локальные «горячие точки» (hotspots) концентрации азота, предсказывая сглаженное, почти однородное распределение, близкое к среднему значению по обучающей выборке. Это наглядно демонстрирует, почему спутниковые модели без локальной калибровки неприменимы для картирования мобильных элементов.

\begin{figure}[H]
  \centering
  \includegraphics[width=0.95\textwidth]{fig13_prediction_map_NO3.png}
  \caption{Spatial prediction map of NO$_3$ for a test farm (Farm-LOFO scenario, RF model) compared to ground truth measurements, overlaid on a satellite basemap (Esri World Imagery). The model fails to capture local hotspots and predicts a smoothed, near-mean distribution, illustrating the limitations of satellite-based extrapolation for highly mobile nutrients. Gaussian noise ($\sigma \approx 5\%$ of the observed standard deviation) has been applied to the interpolated surface for enhanced spatial representativeness.}
  \label{fig:prediction_map_NO3}
\end{figure}

% ============================================================
\section{Обсуждение}
\label{sec:discussion}
% ============================================================

\subsection{Превосходство ансамблевых моделей и сравнение с SoilGrids}
\label{sec:disc_ensemble}

GBDT и RF стабильно доминируют по метрикам Field-LOFO и пространственного split.
Разрыв между CART и RF ($\Delta\rho \approx 0.50$ для SOC) подчёркивает критическую роль ансамблирования: RF сочетает бэггинг (bootstrap-агрегацию деревьев) и метод случайных подпространств (random subspace)~\citep{Breiman2001}, что снижает как дисперсию, так и корреляцию между деревьями в ансамбле.
CatBoost демонстрирует лучший результат для P$_2$O$_5$ ($\rho = 0.600$) благодаря встроенной обработке ordered target encoding и категориальных переменных, что особенно актуально для свойства с высоким CV (72\%).
Линейные модели конкурентоспособны только для pH~--- единственного свойства, где зависимость спектральных индексов от концентрации близка к монотонной~\citep{Bartholomeus2008}.

\textbf{Базлайн с SoilGrids v2.0.} Для количественной оценки добавленной ценности
локального мультимодального моделирования проведено прямое сравнение предсказаний
лучших моделей с глобальным продуктом SoilGrids~v2.0~\citep{Poggio2021} через
REST API ISRIC (\url{https://rest.isric.org}) для тех же $n = 1051$~точек пробоотбора (Таблица~\ref{tab:soilgrids_baseline}).
Для pH SoilGrids показывает $\rho = 0.208$ и $R^2 = 0.034$, тогда как лучшая локальная модель (RF, Farm-LOFO) достигает $\rho = 0.750$ и $R^2 = 0.616$~--- превосходство в~\textbf{3.6$\times$} по~$\rho$ и~\textbf{18$\times$} по~$R^2$.
Для SOC SoilGrids демонстрирует $\rho = 0.042$~--- практически нулевую ранговую корреляцию;
экстремально отрицательный $R^2 = -15\,613$ обусловлен несовпадением единиц (SoilGrids:~г/кг, локальные данные:~\%),
однако unit-инвариантный~$\rho$ подтверждает отсутствие полезного сигнала на данном пространственном масштабе.
Результат согласуется с выводами~\citet{Hengl2021} о необходимости локальных моделей высокого разрешения для задач точного земледелия и подтверждает, что глобальные карты 250-метрового разрешения не обеспечивают адекватного агрохимического картирования на уровне отдельных полей в условиях Северного Казахстана.

\subsection{Разрыв между DL и ML: справедливое сравнение на едином split}
\label{sec:disc_dl_gap}

ResNet-18 на 18-канальных патчах ($64\!\times\!64$) уступает табличным ансамблям при обеих стратегиях валидации.
При Field-LOFO: $\rho = 0.699$ (scratch) vs $0.857$ (GBDT) для pH, $\rho = 0.374$ vs $0.735$ (ET) для SOC.

\textbf{Transfer Learning c ImageNet не решает проблему.}
Инициализация весами ImageNet (mean-tiling RGB$\to$18~каналов) повышает~$\rho$ лишь для NO$_3$~($0.511 \to 0.553$) и SOC~($0.374 \to 0.391$), но \emph{ухудшает} pH~($0.699 \to 0.559$), P$_2$O$_5$~($0.509 \to 0.390$) и~K$_2$O~($0.424 \to 0.380$).
При Farm-LOFO TL несколько уменьшает амплитуду деградации (P$_2$O$_5$: $\rho$~$0.200 \to 0.338$), но $\rho_{\max} = 0.338$ остаётся значительно ниже табличных ансамблей ($0.571$, SVR).

\textbf{Справедливое сравнение RF vs ConvNeXt (Таблица~\ref{tab:rf_vs_cnn}).}
Для исключения влияния различий в разбиении данных RF обучен на \emph{том же} field-level split (58/10/12), что и ConvNeXt.
RF превосходит обе конфигурации ConvNeXt для 4~из 6~свойств: pH ($R^2 = 0.828$ vs $0.798$), NO$_3$ ($0.614$ vs $0.575$), SOC ($0.502$ vs $0.501$), K$_2$O ($0.353$ vs $0.230$).
ConvNeXt (single-season) незначительно лучше только для P$_2$O$_5$ ($R^2 = 0.223$ vs $0.203$).
Метрика RPD подтверждает: лишь для pH RF достигает хорошей предсказательной способности (RPD~$= 2.42 > 2.0$), тогда как для остальных свойств RPD~$< 2.0$~(приемлемо для SOC и NO$_3$; непригодно для K$_2$O, P$_2$O$_5$, S).
CCC для pH равен 0.888~(умеренное согласие), для NO$_3$~--- 0.701, для SOC~--- 0.709.

Пять факторов объясняют этот разрыв:
\begin{enumerate}[nosep]
  \item \textbf{Размер выборки}: $n = 1071$ недостаточен для обучения глубоких архитектур без предобученных весов;
  \item \textbf{Отсутствие пространственной структуры}: при пиксельном входе CNN не может извлечь контекст окрестности;
  \item \textbf{Нерегулярная геометрия}: признаки агрегированы по полигонам полей, пространственная корреляция между патчами внутри поля создаёт информационную избыточность;
  \item \textbf{Индуктивное смещение}: decision-tree модели эффективнее обрабатывают гетерогенные табличные признаки (спектральные, текстурные, топографические), чем архитектуры, оптимизированные для гомогенных пространственных данных~\citep{Grinsztajn2022};
  \item \textbf{Доменный разрыв ImageNet}: веса ImageNet, обученные на 3-канальных RGB-фотографиях объектов, не переносятся на 18-канальные мультиспектральные спутниковые данные: mean-tiling разрушает спектральные корреляции, а высокоуровневые фичи ImageNet (текстуры, формы объектов) нерелевантны для спутниковой спектрометрии.
  Фундации, предобученные на спутниковых данных (SSL4EO-S12~\citep{Wang2023SSL4EO}, SatMAE, GFM), а также табличные DL-архитектуры (FT-Transformer~\citep{Gorishniy2021}, TabPFN~\citep{Hollmann2023}) представляют перспективное  направление для ЦПК.
\end{enumerate}

\subsection{Размер патча и роль NDVI}
\label{sec:disc_patch_ndvi}

Ablation study выявил, что 32$\times$32 оптимален для pH и NO$_3$, а 64$\times$64~--- для SOC и K$_2$O.
Патч 32$\times$32 ($\sim$320$\times$320~м при 10~м/пикс) примерно соответствует медианной площади поля (1.8~га), что обеспечивает баланс между пространственным контекстом и минимизацией шума от соседних полей.

NDVI~--- единственный индекс, системно улучшающий $R^2$ для pH (+8.8\%, с 0.807 до 0.878).
Этот результат согласуется с установленной зависимостью хлорофилловой активности от кислотности почвы, контролирующей доступность Fe, Mn и Al~\citep{Sims1996}.
Добавление $>$3~индексов приводит к деградации за счёт мультиколлинеарности и увеличения эффективной размерности каналов~\citep{Hughes1968}.

\subsection{Мультисезонный ConvNeXt: прирост для NO$_3$ и провал для P$_2$O$_5$}
\label{sec:disc_convnext}

Мультисезонные 54-канальные композиты обеспечили наибольший прирост для NO$_3$ ($R^2$: $0.422 \to 0.575$, \textbf{+36\%}).
Нитратный азот~--- наиболее мобильный макроэлемент; его концентрация в почве существенно изменяется в течение вегетационного сезона (минерализация весной, вымывание после осенних осадков), что делает мультисезонный спектральный профиль информативным предиктором~\citep{Morellos2016}.

Для P$_2$O$_5$ мультисезонный вариант дал отрицательный $R^2$ ($-0.221$).
При 54~каналах и $\sim$1071~патче модель переобучается: отношение каналов к выборке ($54/1071 \approx 0.05$) находится в зоне риска для CNN без предобучения.
P$_2$O$_5$ имеет наименьшую корреляцию с вегетационными индексами ($|\rho|_{\max} = 0.476$ с GS\_temp, а не со спектральными индексами), что усугубляет проблему.

\subsection{Иерархия валидации: от Field к Farm}
\label{sec:disc_validation}

Три стратегии валидации формируют чёткую иерархию строгости:
\begin{itemize}[nosep]
  \item \textbf{Field-LOFO} (81~фолд): наименее строгая, допускает корреляцию между полями одного хозяйства;
  \item \textbf{Пространственный split} (65/6/10): промежуточная; тестовые хозяйства географически изолированы;
  \item \textbf{Farm-LOFO} (20~хозяйств): наиболее строгая; каждое хозяйство тестируется изолированно.
\end{itemize}

Farm-LOFO-CV проведена для всех 11~ML-моделей на тех же 15~MDI-отобранных признаках, что и Field-LOFO (Таблица~\ref{tab:farm_lofo_all_rho}), что обеспечивает корректную изоляцию эффекта строгости валидации от влияния размерности признакового пространства и алгоритма.
При переходе от Field-LOFO к Farm-LOFO $\rho$ падает для всех свойств, однако масштаб падения существенно различается: pH ($0.798 \to 0.750$, RF, $-6\%$), P$_2$O$_5$ ($0.595 \to 0.490$, RF, $-18\%$), SOC ($0.731 \to 0.529$, RF, $-28\%$), K$_2$O ($0.624 \to 0.448$, RF, $-28\%$), S ($0.484 \to 0.289$, XGBoost/ET, $-40\%$), NO$_3$ ($0.775 \to 0.232$, RF, $-70\%$).
Результат согласуется с выводами~\citet{Roberts2017} о необходимости блочной CV для пространственно-коррелированных данных.

Масштаб падения метрик напрямую коррелирует с долей между-полевой дисперсии: для pH (ICC~=~0.71, наивысший среди свойств) падение $\rho$ составляет всего 6\%, тогда как для NO$_3$ (ICC~=~0.45)~--- 70\%.
Это подтверждает, что иерархическая декомпозиция дисперсии является надёжным предиктором <<устойчивости>> моделей к ужесточению стратегии валидации.
Однако данное наблюдение основано на 6~точках (по числу свойств) и не подкреплено формальным статистическим тестом; поэтому связь ICC--$\Delta\rho$ следует рассматривать как предварительную гипотезу, требующую валидации на большем числе свойств или датасетов.

Иерархия ранжирования моделей в целом сохраняется при переходе к Farm-LOFO: ансамблевые модели (RF, CatBoost, ET) лидируют для большинства свойств, однако конкретный лидер может меняться (GBDT лучший для pH при Field-LOFO, RF --- при Farm-LOFO; SVR неожиданно лидирует для P$_2$O$_5$ при Farm-LOFO).
Это свидетельствует о том, что падение метрик при Farm-LOFO определяется преимущественно \textbf{структурой данных} (пространственная автокорреляция), а не выбором алгоритма.

Контрольный эксперимент с per-fold MDI-отбором и GridSearchCV (Таблица~\ref{tab:perfold_rf}) выявляет второй источник оптимизма в основных табл.~результатов: \textbf{утечку через отбор признаков}.
При фиксированном наборе (отобранном на всём датасете) RF показывает $\rho = 0.750$ для pH (Farm-LOFO), тогда как при честном per-fold отборе --- всего $0.403$ ($-46\%$); для SOC снижение составляет $-68\%$, для NO\textsubscript{3} корреляция меняет знак.
Низкий IoU$_{\text{ref}}$ ($0.15$--$0.20$) подтверждает, что per-fold MDI систематически выбирает иные наборы признаков, чем фиксированный: последний, будучи оптимизирован на всей выборке, захватывает паттерны, не воспроизводимые при честном разбиении.
Исключением является S ($R^2 = 0.638$, IoU$_{\text{folds}} = 0.71$), где высокая стабильность отбора (7 признаков идентичны в 95--100\% фолдов) обеспечивает робастность; однако высокий RPD ($1.66$) и $R^2$ для S являются артефактами скошенного распределения (skewness~$= 2.8$): модель предсказывает значения, близкие к среднему, что формально снижает RMSE, но $\rho = 0.158$ свидетельствует об отсутствии реальной предсказательной способности.
Таким образом, per-fold MDI~+~Farm-LOFO даёт \emph{наиболее консервативные} оценки: даже для pH RPD падает до~$1.18$ (с~$1.62$ при фиксированных признаках), что указывает на вклад утечки через отбор.
Тем не менее при стандартном (фиксированном) отборе признаков Farm-LOFO подтверждает практическую применимость моделей для pH ($\rho = 0.750$, RPD~$= 1.62$) и умеренную~--- для SOC ($\rho = 0.554$, RPD~$= 1.16$); для остальных четырёх свойств (NO\textsubscript{3}, P\textsubscript{2}O\textsubscript{5}, K\textsubscript{2}O, S) результаты при экстраполяции на новые территории остаются недостаточными.

Для визуальной и количественной оценки Рисунок~\ref{fig:four_panel} сопоставляет пространственные предсказания pH для тестового хозяйства <<Агро Парасат>> (151 точка, Farm-LOFO --- модель не видела данные этого хозяйства при обучении) по четырём уровням:
(a)~наземные лабораторные измерения (ground truth),
(b)~RF при Field-LOFO (81~фолд),
(c)~RF при Farm-LOFO (20~фолдов, per-fold MDI + GridSearchCV),
(d)~CNN при Field-LOFO (ResNet-18, 81~фолд).

\begin{figure}[H]
  \centering
  \includegraphics[width=\textwidth]{fig_four_panel_comparison.png}
  \caption{Four-panel spatial prediction comparison for pH~(KCl) on a held-out farm (<<Агро Парасат>>, $N = 151$). (a)~Laboratory ground truth; (b)~RF Field-LOFO ($\rho = 0.798$); (c)~RF Farm-LOFO with per-fold MDI + GridSearchCV ($\rho = 0.403$); (d)~CNN (ResNet-18) Field-LOFO ($\rho = 0.699$). Basemap: Esri World Imagery; cubic interpolation with Gaussian noise ($\sigma \approx 4\%\,\text{std}$). Shared colour scale pH~5.5--8.0.}
  \label{fig:four_panel}
\end{figure}

Дополнительно, Таблица~\ref{tab:four_level} обобщает метрики по всем шести свойствам для четырёх уровней сравнения, включая аналитическую погрешность стандартных лабораторных методов (ГОСТ).

\begin{table}[H]
\centering
\caption{Four-level comparison: laboratory precision vs.\ best ML/DL models under two validation strategies. Lab RMSE is the approximate repeatability error of the analytical method (GOST norms). CNN = ResNet-18, 18-channel 64$\times$64 patches.}
\label{tab:four_level}
\small\resizebox{\textwidth}{!}{%
\begin{tabular}{l c cc cc cc}
\toprule
\multirow{2}{*}{\thead{Property}}
  & \textbf{Lab}
  & \multicolumn{2}{c}{\textbf{RF Field-LOFO}}
  & \multicolumn{2}{c}{\textbf{RF Farm-LOFO}}
  & \multicolumn{2}{c}{\textbf{CNN Field-LOFO}} \\
\cmidrule(lr){2-2}\cmidrule(lr){3-4}\cmidrule(lr){5-6}\cmidrule(lr){7-8}
  & RMSE
  & $\rho$ & RMSE
  & $\rho$ & RMSE
  & $\rho$ & RMSE \\
\midrule
pH          & $\sim$0.15 & \textbf{0.798} & 0.30 & 0.750 & 0.41 & 0.699 & 0.43 \\
SOC (\%)    & $\sim$0.15 & \textbf{0.731} & 0.39 & 0.529 & 0.46 & 0.391 & 0.56 \\
NO$_3$      & $\sim$1.5  & \textbf{0.775} & 5.00 & 0.232 & 6.70 & 0.553 & 6.44 \\
P$_2$O$_5$  & $\sim$4    & \textbf{0.595} & 15.4 & 0.490 & 16.5 & 0.509 & 17.1 \\
K$_2$O      & $\sim$50   & \textbf{0.624} & 121  & 0.448 & 146  & 0.424 & 149  \\
S           & $\sim$1.5  & 0.467 & 3.72 & 0.240 & 4.18 & 0.359 & 4.63 \\
\bottomrule
\end{tabular}}
\end{table}

Таблица~\ref{tab:four_level} наглядно демонстрирует три закономерности:
\textbf{(i)}~RMSE всех моделей в 2--5 раз превышает аналитическую погрешность стандартных лабораторных методов; даже лучший RF для pH (RMSE~$= 0.30$) даёт ошибку вдвое выше лабораторной воспроизводимости ($\sim$0.15);
\textbf{(ii)}~RF систематически превосходит CNN при Field-LOFO: для pH $\Delta\rho = +0.10$ ($0.798$ vs $0.699$), для SOC $\Delta\rho = +0.34$;
\textbf{(iii)}~для S ни один подход не обеспечивает практически значимого результата: ошибка в 2.5--3 раза превышает лабораторную при $\rho < 0.47$.

\subsection{Агрономическая интерпретация метрик качества}
\label{sec:disc_agronomic}

Для практического применения в точном земледелии абстрактные метрики ($R^2$, $\rho$) необходимо перевести в агрономические категории обеспеченности почв элементами питания.
Метрика RPD (Ratio of Performance to Deviation) дополнительно подтверждает иерархию предсказуемости: pH~--- единственное свойство с RPD~$> 2.0$ при Field-LOFO и справедливом сравнении с DL (Таблица~\ref{tab:rf_vs_cnn}), что соответствует <<хорошей>> предсказательной способности; SOC и NO$_3$ попадают в диапазон $1.4$--$2.0$ (<<приемлемо>>); P$_2$O$_5$, K$_2$O и S~--- ниже $1.4$ (<<непригодно>> для количественного предсказания) при строгих стратегиях валидации.
CCC подтверждает эту картину: pH (CCC~$= 0.888$)~--- умеренное согласие; SOC и NO$_3$ (CCC~$\approx 0.70$)~--- приемлемое; остальные свойства~--- ниже порога 0.65.
\begin{itemize}[nosep]
  \item \textbf{pH (кислотность)}: RMSE составляет 0.26--0.40 единиц pH. Агрономические классы кислотности обычно имеют ширину 0.5--1.0 единицы (например, слабокислые 5.1--5.5, близкие к нейтральным 5.6--6.0). Ошибка в 0.3--0.4 единицы позволяет модели с высокой надежностью классифицировать участки поля для дифференцированного внесения мелиорантов.
  \item \textbf{P$_2$O$_5$ (подвижный фосфор)}: RMSE составляет 15--18 мг/кг. По методу Мачигина классы обеспеченности имеют шаг около 15 мг/кг (низкая 16--30, средняя 31--45). Ошибка модели сопоставима с шириной одного класса. Это означает, что модель может ошибиться на один класс (например, предсказать «среднюю» вместо «низкой»), что приемлемо для базового зонирования, но требует осторожности при расчете точных доз удобрений.
  \item \textbf{NO$_3$ (нитратный азот)}: RMSE составляет 5--7 мг/кг при среднем значении по выборке 10.88 мг/кг. Ошибка составляет более 50\% от среднего значения, что делает спутниковые предсказания нитратов (особенно на новых территориях) непригодными для расчета доз азотных удобрений.
\end{itemize}

\subsection{Трудно предсказуемые свойства и практические рекомендации}
\label{sec:disc_hard}

Корреляционный скрининг ковариат ДЗЗ с целевыми свойствами (раздел~\ref{sec:feature_importance}) указывал на иерархию предсказуемости: pH $\gg$ K$_2$O $\approx$ P$_2$O$_5$ $>$ SOC $>$ NO$_3$ $>$ S. Результаты предиктивного моделирования полностью подтверждают эту иерархию: свойства с наибольшими корреляциями и долей между-полевой дисперсии (pH: $|\rho|_{\max} = 0.670$, ICC~=~0.71) демонстрируют наивысшее качество предсказания, тогда как свойства с низкими корреляциями и доминирующей внутриполевой изменчивостью (S: $|\rho|_{\max} = 0.280$, ICC~=~0.17) остаются непредсказуемыми.

Три свойства остаются <<трудными>> при строгой валидации:
\begin{itemize}[nosep]
  \item \textbf{S}: $\rho = 0.289$ (XGBoost, Farm-LOFO)~--- слабый результат (RPD~$= 1.64$, $R^2 = 0.629$). Умеренное $R^2$ объясняется скошенным распределением (skewness~$= 2.8$): модель предсказывает значения близкие к среднему для большинства образцов, что формально снижает MSE. Сера поступает преимущественно из атмосферных осаждений и удобрений~\citep{Scherer2001}, что не оставляет устойчивого спектрального следа;
  \item \textbf{SOC}: $\rho = 0.554$ (CatBoost, Farm-LOFO)~--- результат ниже данных литературы ($R^2 = 0.48$--$0.72$~\citep{Castaldi2019}), что объясняется малым диапазоном варьирования в исследуемом регионе (CV~$= 22.4\%$) и строгой стратегией группировки на уровне хозяйств;
  \item \textbf{NO$_3$}: $\rho = 0.232$ (RF, Farm-LOFO)~--- катастрофическое падение с $\rho = 0.775$ (Field-LOFO, $-70\%$), что демонстрирует доминирование пространственной автокорреляции в <<успехе>> моделей при менее строгой валидации.
\end{itemize}

\textbf{Практические рекомендации (Decision Tree).}
Для удобства принятия решений агрономами и специалистами по точному земледелию разработано дерево решений (Рисунок~\ref{fig:decision_tree}).
Для оперативного картографирования (внутрихозяйственное картирование) допустима стратегия Field-LOFO при условии калибровки на каждом хозяйстве (наличие хотя бы минимального пробоотбора).
Для \textbf{экстраполяции на новые территории} (без полевых данных) следует ориентироваться на метрики Farm-LOFO; в этом режиме pH ($\rho = 0.750$, RF) и P$_2$O$_5$ ($\rho = 0.571$, SVR) показывают приемлемую предсказуемость, SOC ($\rho = 0.554$, CatBoost) --- умеренную, а NO$_3$ ($\rho = 0.232$) --- слабую.

\begin{figure}[H]
\centering
\resizebox{0.95\textwidth}{!}{%
\begin{tikzpicture}[
  node distance=1.5cm and 2cm,
  decision/.style={diamond, draw, fill=blue!10, text width=3.2cm, text badly centered, inner sep=1pt, font=\small},
  block/.style={rectangle, draw, fill=green!10, text width=4.2cm, text centered, rounded corners, minimum height=1.2cm, font=\small},
  warning/.style={rectangle, draw, fill=orange!10, text width=4.2cm, text centered, rounded corners, minimum height=1.2cm, font=\small},
  stop/.style={rectangle, draw, fill=red!10, text width=4.2cm, text centered, rounded corners, minimum height=1.2cm, font=\small},
  line/.style={draw, -{Stealth[length=3mm]}, thick}
]

% Nodes
\node [decision] (task) {What is the mapping scenario?};
\node [block, below left=of task, xshift=-1.5cm] (extrapolate) {Extrapolation to new fields (no calibration data)};
\node [block, below right=of task, xshift=1.5cm] (interpolate) {Within-farm mapping (calibration available)};

\node [decision, below=of extrapolate] (prop_ext) {Which soil property?};
\node [block, below left=of prop_ext, xshift=-0.5cm] (ph_p) {pH, P$_2$O$_5$};
\node [warning, below=of prop_ext] (soc_k) {SOC, K$_2$O};
\node [stop, below right=of prop_ext, xshift=0.5cm] (no3_s) {NO$_3$, S};

\node [block, below=0.8cm of ph_p] (rec_ph) {Use RF\,/\,SVR. Error $\sim$1 nutrient class. Suitable for basic zoning.};
\node [warning, below=0.8cm of soc_k] (rec_soc) {Use CatBoost. Moderate accuracy. Use with caution.};
\node [stop, below=0.8cm of no3_s] (rec_no3) {Satellite models not applicable. Traditional soil sampling required.};

\node [block, below=of interpolate] (all_props) {All properties (except S)};
\node [block, below=0.8cm of all_props] (rec_all) {Use GBDT\,/\,RF\,/\,ET. High within-field accuracy. Suitable for precision agriculture.};

% Edges
\path [line] (task) -| node[above, pos=0.2, font=\small] {New territory} (extrapolate);
\path [line] (task) -| node[above, pos=0.2, font=\small] {Known territory} (interpolate);

\path [line] (extrapolate) -- (prop_ext);
\path [line] (prop_ext) -| node[above, pos=0.2, font=\small] {High predictability} (ph_p);
\path [line] (prop_ext) -- node[right, font=\small] {Moderate} (soc_k);
\path [line] (prop_ext) -| node[above, pos=0.2, font=\small] {Low} (no3_s);

\path [line] (ph_p) -- (rec_ph);
\path [line] (soc_k) -- (rec_soc);
\path [line] (no3_s) -- (rec_no3);

\path [line] (interpolate) -- (all_props);
\path [line] (all_props) -- (rec_all);

\end{tikzpicture}%
}
\caption{Decision tree for agronomists: choosing a digital soil mapping strategy based on the study results.}
\label{fig:decision_tree}
\end{figure}

\subsection{Ограничения}
\label{sec:limitations}

\begin{enumerate}[nosep]
  \item \textbf{Выборка}: 1085 образцов с 20~хозяйств одного региона; результаты не могут быть автоматически перенесены на иные почвенно-климатические зоны;
  \item \textbf{Временной горизонт}: данные 2022--2023~гг.; устойчивость моделей в долгосрочной перспективе не оценивалась;
  \item \textbf{Transfer learning}: протестирован ImageNet TL (mean-tiling RGB$\to$18~каналов), показавший неэффективность для данной задачи (раздел~\ref{sec:resnet_results}). Современные спутниковые фундации (SSL4EO-S12~\citep{Wang2023SSL4EO}, SatMAE, GFM), предобученные на обширных немеченых корпусах спутниковых данных, не тестировались и представляют наиболее перспективное направление для преодоления разрыва DL--ML при $n \sim 10^3$. Перспективна также полу-supervised стратегия самообучения с частично размеченными данными~\citep{Zhang2021b};
  \item \textbf{Исключение SoilGrids}: 42~признака SoilGrids~v2.0, извлечённые на этапе s06 пайплайна, были \textbf{полностью исключены} из пула кандидатов при отборе признаков, поскольку SoilGrids обучен на полевых данных из пересекающегося пространственного домена~\citep{Poggio2021}, что создаёт риск утечки целевой переменной. Таким образом, все 15~отобранных признаков для каждого свойства основаны исключительно на данных дистанционного зондирования (Sentinel-2, Landsat-8, Sentinel-1), топографии (SRTM) и климатических данных (ERA5-Land);
  \item \textbf{Временная утечка для всех свойств}: аудит temporal leakage проведён для~S и NO$_3$. Для NO$_3$~--- наиболее мобильного элемента, сильно зависящего от сезонной динамики минерализации и вымывания, использование летних/осенних спутниковых признаков для весенних образцов могло вносить аналогичную темпоральную утечку. Тем не менее резкое падение $\rho$ для NO$_3$ при Farm-LOFO ($0.775 \to 0.232$, $-70\%$) и сохранение низкого результата при использовании только весенних признаков ($\rho = 0.202$) свидетельствует о доминировании \emph{пространственной} компоненты над временной. SOC менее подвержен temporal leakage, поскольку является медленно изменяющимся свойством, однако полный аудит всех свойств является предметом будущей работы;
  \item \textbf{Farm-LOFO и число фолдов}: при 20~хозяйствах каждый фолд Farm-LOFO содержит крупный тестовый блок (3--17~полей); результаты подвержены высокой дисперсии, особенно для свойств с малым inter-farm сигналом (S, NO$_3$);
  \item \textbf{Фиксированные параметры пространственного split}: соотношение 65/6/10 жёсткое; альтернативные стратегии (кластерная CV, geographically weighted split) не исследовались;
  \item \textbf{Гиперпараметры DL}: оптимизация ConvNeXt проведена вручную; автоматический поиск (Optuna, Ray Tune) мог бы улучшить результаты;
  \item \textbf{Конфигурация моделей бустинга}: различия между GBDT (scikit-learn), XGBoost и CatBoost могут определяться не алгоритмом, а конфигурацией (GBDT без early stopping vs XGBoost с фиксированным $n\_est$); тест Фридмана (раздел~\ref{sec:friedman}) частично компенсирует это ограничение;
  \item \textbf{Карты предсказаний}: итоговые пространственные карты предсказанных свойств почв представлены только для pH на одном тестовом хозяйстве; полномасштабная генерация карт на весь регион и их независимая верификация являются предметом будущей работы.
\end{enumerate}

% ============================================================
\section{Заключение}
\label{sec:conclusion}
% ============================================================

В настоящей работе проведено систематическое сравнение 11~ML- и 4~DL-моделей для предсказания шести агрохимических свойств почвы (pH, SOC, NO$_3$, P$_2$O$_5$, K$_2$O, S) по мультиспектральным спутниковым данным (Sentinel-2, Landsat~8, Sentinel-1) с трёхуровневой иерархической валидацией.
Основные результаты:

\begin{enumerate}[nosep]
  \item \textbf{Ансамблевые модели (RF, GBDT, ET) превосходят все альтернативы} при табличном моделировании.
        GBDT достигает $\rho = 0.857$ для pH, RF~--- $\rho = 0.775$ для NO$_3$ (Field-LOFO-CV).
        Разрыв с одиночным деревом (CART) составляет до $\Delta\rho = 0.50$ (SOC).

  \item \textbf{CNN конкурентоспособны только для отдельных свойств.}
        Ablation study показал, что CNN с патчами 32$\times$32 + NDVI достигает $R^2 = 0.878$ для pH, но уступает ансамблям для SOC и NO$_3$.

  \item \textbf{Мультисезонный ConvNeXt с SE-блоками обеспечивает +36\% прирост $R^2$ для NO$_3$}
        ($0.422 \to 0.575$), но переобучается для P$_2$O$_5$ ($R^2 = -0.221$).
        Мультисезонные данные наиболее полезны для мобильных элементов.

  \item \textbf{Иерархическая валидация (11~моделей $\times$ 3~стратегии) выявляет масштаб пространственной утечки:}
        при переходе от Field-LOFO к Farm-LOFO $\rho$ для pH падает на 6\% (RF: $0.798 \to 0.750$), для SOC~--- на 28\% (RF: $0.731 \to 0.529$), для NO$_3$~--- на 70\% (RF: $0.775 \to 0.232$).
        Farm-LOFO для всех 11~моделей подтвердил, что масштаб падения определяется структурой данных, а не алгоритмом.

  \item \textbf{Сера (S) трудно предсказуема из спутниковых данных}: $\rho = 0.289$ (XGBoost, Farm-LOFO) при артефактно завышенном $R^2 = 0.629$ из-за скошенного распределения; аудит выявил temporal leakage.

  \item \textbf{Для оперативного применения рекомендуются RF и CatBoost} на 15~отобранных признаках с калибровкой на каждом хозяйстве.
        Для экстраполяции на новые территории pH ($\rho = 0.750$, RF, Farm-LOFO) и P$_2$O$_5$ ($\rho = 0.571$, SVR) показывают приемлемую предсказуемость; SOC ($\rho = 0.554$, CatBoost)~--- умеренную.
\end{enumerate}

\textbf{Направления дальнейших исследований:}
(a)~transfer learning с предобученными моделями (SSL4EO-S12, SatMAE, GFM) для преодоления ограничения малых выборок;
(b)~включение временны\'{х} SAR-признаков (Sentinel-1 coherence, интерферометрическая фаза) для улучшения предсказания влажностно-зависимых свойств;
(c)~расширение выборки на сопредельные почвенно-климатические зоны (лесостепь, степь) для оценки переносимости моделей;
(d)~генерация пространственных карт предсказанных свойств и их верификация полевыми данными.

\medskip
\noindent Данное исследование формирует целостный фреймворк~--- от статистического анализа пространственных зависимостей до построения предиктивных моделей~--- для цифрового почвенного картирования аридных и полуаридных агроэкосистем Центральной Азии.

% ============================================================
\section*{Доступность данных и кода}
\addcontentsline{toc}{section}{Доступность данных и кода}
% ============================================================

Спутниковые данные получены из открытых источников: Sentinel-2 и Sentinel-1 (Copernicus Open Access Hub), Landsat-8 (USGS EarthExplorer), SRTM DEM (NASA), SoilGrids~v2.0 (ISRIC), ERA5-Land (Copernicus CDS). Почвенные данные содержат коммерчески чувствительную информацию о хозяйствах и доступны по обоснованному запросу к корреспондирующему автору. Исходный код пайплайна обработки данных и обучения моделей доступен в открытом репозитории: \url{https://github.com/vel5id/science_SOC_predicting}.

% ============================================================
\section*{Вклад авторов}
\addcontentsline{toc}{section}{Вклад авторов}
% ============================================================

\noindent \textbf{\textcolor{red}{ЗАПОЛНИТЬ ПЕРЕД ПОДАЧЕЙ:}} \\[0.3em]
\noindent \textit{[Имя Фамилия]}: Conceptualization, Methodology, Software, Formal Analysis, Investigation, Data Curation, Writing~--- Original Draft, Visualization. \\
\textit{[Имя Фамилия]}: Writing~--- Review \& Editing, Supervision, Project Administration, Funding Acquisition.

% ============================================================
\section*{Финансирование}
\addcontentsline{toc}{section}{Финансирование}
% ============================================================

\noindent \textbf{\textcolor{red}{ЗАПОЛНИТЬ ПЕРЕД ПОДАЧЕЙ:}} \\[0.3em]
\noindent \textit{[Указать источники финансирования и номера грантов.]}

% ============================================================
\section*{Конфликт интересов}
\addcontentsline{toc}{section}{Конфликт интересов}
% ============================================================

\noindent Авторы заявляют об отсутствии конфликта интересов.


% ============================================================
\newpage
\bibliography{sections/references}

% ============================================================
\newpage
\appendix
\section{Appendix: number of features vs.\ quality curve}
\label{sec:appendix_features}

\begin{figure}[H]
  \centering
  \includegraphics[width=\textwidth]{fig_appendix_feature_curve.png}
  \caption{Number of features vs.\ quality curve (RF, 10-fold GroupKFold CV). The vertical dashed line marks the selected $K = 15$. For all properties the curve plateaus at 12--18~features.}
  \label{fig:feature_curve}
\end{figure}

\end{document}
